\input texinfo    @c -*-texinfo-*-
@comment %**start of header (This is for running Texinfo on a region.)
@setfilename readline.info
@settitle GNU Readline Library
@comment %**end of header (This is for running Texinfo on a region.)
@synindex vr fn
@setchapternewpage odd

@ignore
last change: Thu Mar 21 16:06:39 EST 1996
@end ignore

@set EDITION 2.1
@set VERSION 2.1
@set UPDATED 21 March 1996
@set UPDATE-MONTH March 1996

@ifinfo
This document describes the GNU Readline Library, a utility which aids
in the consistency of user interface across discrete programs that need
to provide a command line interface.

Copyright (C) 1988, 1991 Free Software Foundation, Inc.

Permission is granted to make and distribute verbatim copies of
this manual provided the copyright notice and this permission notice
pare preserved on all copies.

@ignore
Permission is granted to process this file through TeX and print the
results, provided the printed document carries copying permission
notice identical to this one except for the removal of this paragraph
(this paragraph not being relevant to the printed manual).
@end ignore

Permission is granted to copy and distribute modified versions of this
manual under the conditions for verbatim copying, provided that the entire
resulting derived work is distributed under the terms of a permission
notice identical to this one.

Permission is granted to copy and distribute translations of this manual
into another language, under the above conditions for modified versions,
except that this permission notice may be stated in a translation approved
by the Foundation.
@end ifinfo

@titlepage  
@title GNU Readline Library
@subtitle Edition @value{EDITION}, for @code{Readline Library} Version @value{VERSION}.
@subtitle @value{UPDATE-MONTH}
@author Brian Fox, Free Software Foundation
@author Chet Ramey, Case Western Reserve University

@page
This document describes the GNU Readline Library, a utility which aids
in the consistency of user interface across discrete programs that need
to provide a command line interface.

Published by the Free Software Foundation @*
675 Massachusetts Avenue, @*
Cambridge, MA 02139 USA

Permission is granted to make and distribute verbatim copies of
this manual provided the copyright notice and this permission notice
are preserved on all copies.

Permission is granted to copy and distribute modified versions of this
manual under the conditions for verbatim copying, provided that the entire
resulting derived work is distributed under the terms of a permission
notice identical to this one.

Permission is granted to copy and distribute translations of this manual
into another language, under the above conditions for modified versions,
except that this permission notice may be stated in a translation approved
by the Foundation.

@vskip 0pt plus 1filll
Copyright @copyright{} 1989, 1991 Free Software Foundation, Inc.
@end titlepage

@ifinfo
@node Top
@top GNU Readline Library

This document describes the GNU Readline Library, a utility which aids
in the consistency of user interface across discrete programs that need
to provide a command line interface.

@menu
* Command Line Editing::	   GNU Readline User's Manual.
* Programming with GNU Readline::  GNU Readline Programmer's Manual.
* Concept Index::		   Index of concepts described in this manual.
* Function and Variable Index::	   Index of externally visible functions
				   and variables.
@end menu
@end ifinfo

@include rluser.tex
@include rltech.tex

@node Concept Index
@unnumbered Concept Index
@printindex cp

@node Function and Variable Index
@unnumbered Function and Variable Index
@printindex fn

@contents
@bye
