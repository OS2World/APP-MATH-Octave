@ignore
This file documents the user interface to the GNU History library.

Copyright (C) 1988, 1991, 1996 Free Software Foundation, Inc.
Authored by Brian Fox and Chet Ramey.

Permission is granted to make and distribute verbatim copies of this manual
provided the copyright notice and this permission notice are preserved on
all copies.

Permission is granted to process this file through Tex and print the
results, provided the printed document carries copying permission notice
identical to this one except for the removal of this paragraph (this
paragraph not being relevant to the printed manual).

Permission is granted to copy and distribute modified versions of this
manual under the conditions for verbatim copying, provided also that the
GNU Copyright statement is available to the distributee, and provided that
the entire resulting derived work is distributed under the terms of a
permission notice identical to this one.

Permission is granted to copy and distribute translations of this manual
into another language, under the above conditions for modified versions.
@end ignore

@node Using History Interactively
@chapter Using History Interactively

@ifset BashFeatures
This chapter describes how to use the GNU History Library interactively,
from a user's standpoint.  It should be considered a user's guide.  For
information on using the GNU History Library in your own programs,
see the GNU Readline Library Manual.
@end ifset
@ifclear BashFeatures
This chapter describes how to use the GNU History Library interactively,
from a user's standpoint.  It should be considered a user's guide.  For
information on using the GNU History Library in your own programs,
@pxref{Programming with GNU History}.
@end ifclear

@ifset BashFeatures
@menu
* Bash History Facilities::	How Bash lets you manipulate your command
				history.
* History Interaction::		What it feels like using History as a user.
@end menu
@end ifset
@ifclear BashFeatures
@menu
* History Interaction::		What it feels like using History as a user.
@end menu
@end ifclear

@ifset BashFeatures
@node Bash History Facilities
@section Bash History Facilities
@cindex command history
@cindex history list

When the @samp{-o history} option to the @code{set} builtin
is enabled (@pxref{The Set Builtin}),
the shell provides access to the @var{command history},
the list of commands previously typed.  The text of the last
@code{HISTSIZE}
commands (default 500) is saved in a history list.  The shell
stores each command in the history list prior to parameter and
variable expansion
but after history expansion is performed, subject to the
values of the shell variables
@code{HISTIGNORE} and @code{HISTCONTROL}.
When the shell starts up, the history is initialized from the
file named by the @code{HISTFILE} variable (default @file{~/.bash_history}).
@code{HISTFILE} is truncated, if necessary, to contain no more than
the number of lines specified by the value of the @code{HISTFILESIZE}
variable.  When an interactive shell exits, the last
@code{HISTSIZE} lines are copied from the history list to @code{HISTFILE}.
If the @code{histappend} shell option is set (@pxref{Bash Builtins}),
the lines are appended to the history file,
otherwise the history file is overwritten.
If @code{HISTFILE}
is unset, or if the history file is unwritable, the history is
not saved.  After saving the history, the history file is truncated
to contain no more than @code{$HISTFILESIZE}
lines.  If @code{HISTFILESIZE} is not set, no truncation is performed.

The builtin command @code{fc} (@pxref{Korn Shell Builtins})
may be used to list or edit and re-execute a portion of
the history list.  The @code{history} builtin (@pxref{C Shell Builtins})
can be used to display or modify the history list and
manipulate the history file.
When using the command-line editing, search commands
are available in each editing mode that provide access to the
history list.

The shell allows control over which commands are saved on the history
list.  The @code{HISTCONTROL} and @code{HISTIGNORE}
variables may be set to cause the shell to save only a subset of the
commands entered.
The @code{cmdhist}
shell option, if enabled, causes the shell to attempt to save each
line of a multi-line command in the same history entry, adding
semicolons where necessary to preserve syntactic correctness.
The @code{lithist}
shell option causes the shell to save the command with embedded newlines
instead of semicolons.
@xref{Bash Builtins} for a description of @code{shopt}.
@end ifset

@node History Interaction
@section Interactive History Expansion
@cindex history expansion

The History library provides a history expansion feature that is similar
to the history expansion provided by @code{csh}.  This section
describes the syntax used to manipulate the history information.

History expansions introduce words from the history list into
the input stream, making it easy to repeat commands, insert the
arguments to a previous command into the current input line, or
fix errors in previous commands quickly.

History expansion takes place in two parts.  The first is to determine
which line from the previous history should be used during substitution.
The second is to select portions of that line for inclusion into the
current one.  The line selected from the previous history is called the
@dfn{event}, and the portions of that line that are acted upon are
called @dfn{words}.  Various @dfn{modifiers} are available to manipulate
the selected words.  The line is broken into words in the same fashion
that Bash does, so that several English (or Unix) words
surrounded by quotes are considered as one word.
History expansions are introduced by the appearance of the
history expansion character, which is @samp{!} by default.
@ifset BashFeatures
Only @samp{\} and @samp{'} may be used to escape the history expansion
character.
@end ifset

@ifset BashFeatures
Several shell options settable with the @code{shopt}
builtin (@pxref{Bash Builtins}) may be used to tailor
the behavior of history expansion.  If the
@code{histverify} shell option is enabled, and Readline
is being used, history substitutions are not immediately passed to
the shell parser.
Instead, the expanded line is reloaded into the Readline
editing buffer for further modification.
If Readline is being used, and the @code{histreedit}
shell option is enabled, a failed history expansion will be
reloaded into the Readline editing buffer for correction.
The @samp{-p} option to the @code{history} builtin command
may be used to see what a history expansion will do before using it.
The @samp{-s} option to the @code{history} builtin may be used to
add commands to the end of the history list without actually executing
them, so that they are available for subsequent recall.

The shell allows control of the various characters used by the
history expansion mechanism with the @code{histchars} variable.
@end ifset

@menu
* Event Designators::	How to specify which history line to use.
* Word Designators::	Specifying which words are of interest.
* Modifiers::		Modifying the results of substitution.
@end menu

@node Event Designators
@subsection Event Designators
@cindex event designators

An event designator is a reference to a command line entry in the
history list.
@cindex history events

@table @asis

@item @code{!}
Start a history substitution, except when followed by a space, tab,
the end of the line, @key{=} or @key{(}.

@item @code{!@var{n}}
Refer to command line @var{n}.

@item @code{!-@var{n}}
Refer to the command @var{n} lines back.

@item @code{!!}
Refer to the previous command.  This is a synonym for @samp{!-1}.

@item @code{!@var{string}}
Refer to the most recent command starting with @var{string}.

@item @code{!?@var{string}[?]}
Refer to the most recent command containing @var{string}.  The trailing
@samp{?} may be omitted if the @var{string} is followed immediately by
a newline.

@item @code{^@var{string1}^@var{string2}^}
Quick Substitution.  Repeat the last command, replacing @var{string1}
with @var{string2}.  Equivalent to
@code{!!:s/@var{string1}/@var{string2}/}.

@item @code{!#}
The entire command line typed so far.

@end table

@node Word Designators
@subsection Word Designators

Word designators are used to select desired words from the event.
A @samp{:} separates the event specification from the word designator.  It
can be omitted if the word designator begins with a @samp{^}, @samp{$},
@samp{*}, @samp{-}, or @samp{%}.  Words are numbered from the beginning
of the line, with the first word being denoted by 0 (zero).  Words are
inserted into the current line separated by single spaces.

@table @code

@item 0 (zero)
The @code{0}th word.  For many applications, this is the command word.

@item @var{n}
The @var{n}th word.

@item ^
The first argument; that is, word 1.

@item $
The last argument.

@item %
The word matched by the most recent @samp{?@var{string}?} search.

@item @var{x}-@var{y}
A range of words; @samp{-@var{y}} abbreviates @samp{0-@var{y}}.

@item *
All of the words, except the @code{0}th.  This is a synonym for @samp{1-$}.
It is not an error to use @samp{*} if there is just one word in the event;
the empty string is returned in that case.

@item @var{x}*
Abbreviates @samp{@var{x}-$}

@item @var{x}-
Abbreviates @samp{@var{x}-$} like @samp{@var{x}*}, but omits the last word.

@end table

If a word designator is supplied without an event specification, the
previous command is used as the event.

@node Modifiers
@subsection Modifiers

After the optional word designator, you can add a sequence of one or more
of the following modifiers, each preceded by a @samp{:}.

@table @code

@item h
Remove a trailing pathname component, leaving only the head.

@item t
Remove all leading  pathname  components, leaving the tail.

@item r
Remove a trailing suffix of the form @samp{.@var{suffix}}, leaving
the basename.

@item e
Remove all but the trailing suffix.

@item p
Print the new command but do not execute it.

@ifset BashFeatures
@item q
Quote the substituted words, escaping further substitutions.

@item x
Quote the substituted words as with @samp{q},
but break into words at spaces, tabs, and newlines.
@end ifset

@item s/@var{old}/@var{new}/
Substitute @var{new} for the first occurrence of @var{old} in the
event line.  Any delimiter may be used in place of @samp{/}.
The delimiter may be quoted in @var{old} and @var{new}
with a single backslash.  If @samp{&} appears in @var{new},
it is replaced by @var{old}.  A single backslash will quote
the @samp{&}.  The final delimiter is optional if it is the last
character on the input line.

@item &
Repeat the previous substitution.

@item g
Cause changes to be applied over the entire event line.  Used in
conjunction with @samp{s}, as in @code{gs/@var{old}/@var{new}/},
or with @samp{&}.

@end table
