@ifclear version
@defcodeindex fl
@defcodeindex op
@end ifclear

@node Reporting bugs
@section Reporting bugs

@cindex reporting bugs
@cindex bugs, reporting

@ifset version
(A copy of this chapter is in the file @file{kpathsea/BUGS}.)
@end ifset

@flindex tex-k@@mail.tug.org @r{(bug address)}
@cindex bug address
If you have problems or suggestions, please report them to
@email{tex-k@@mail.tug.org} using the bug checklist below.

Please report bugs in the documentation; not only factual errors or
inconsistent behavior, but unclear or incomplete explanations, typos,
wrong fonts, @dots{}

@menu
* Bug checklist::       What to include in a good bug report.
* Mailing lists::       Joining the bugs or announcements mailing lists.
* Debugging::           Analyzing runtime problems.
* Logging::             Recording searches.
* Common problems::     When things go wrong.
@end menu


@node Bug checklist
@subsection Bug checklist

@cindex checklist for bug reports
@cindex bug checklist

Before reporting a bug, please check below to be sure it isn't already
known (@pxref{Common problems}).

Bug reports should be sent via electronic mail to
@email{tex-k@@mail.tug.org}, or by postal mail to 135 Center Hill Road /
Plymouth, MA 02360 / USA.

The general principle is that a good bug report includes all the
information necessary for reproduction.  Therefore, to enable
investigation, your report should include the following:

@itemize @bullet
@item
@cindex version numbers, determining
The version number(s) of the program(s) involved, and of Kpathsea
itself.  You can get the former by giving a sole option @samp{--version}
to the program, and the latter by running @samp{kpsewhich --version}.
The @file{NEWS} and @file{ChangeLog} files also contain the version
number.

@item
@pindex uname
The hardware, operating system (including version number), compiler, and
@code{make} program you are using (the output of @code{uname -a} is a
start on the first two, though often incomplete).  If the bug involves
the X window system, include X version and supplier information as well
(examples: X11R6 from MIT; X11R4 from HP; OpenWindows 3.3 bundled with
SunOS 4.1.4).

@item
@flindex config.log
Any options you gave to @code{configure}.  This is recorded in the
@file{config.status} files.

@cindex configuration bugs
@flindex config.status
If you are reporting a bug in @samp{configure} itself, it's probably
system-dependent, and it will be unlikely the maintainers can do
anything useful if you merely report that thus-and-such is broken.
Therefore, you need to do some additional work: for some bugs, you can
look in the file @file{config.log} where the test that failed should
appear, along with the compiler invocation and source program in
question.  You can then compile it yourself by hand, and discover why
the test failed.  Other @samp{configure} bugs do not involve the
compiler; in that case, the only recourse is to inspect the
@code{configure} shell script itself, or the Autoconf macros that
generated @code{configure}.

@item
The log of all debugging output, if the bug is in path searching.  You
can get this by setting the environment variable @code{KPATHSEA_DEBUG}
to @samp{-1} before running the program.  Please look at the log
yourself to make sure the behavior is really a bug before reporting it;
perhaps ``old'' environment variable settings are causing files not to
be found, for example.

@item
The contents of any input files necessary to reproduce the bug.  For
bugs in DVI-reading programs, for example, this generally means a DVI
file (and any EPS or other files it uses)---@TeX{} source files are
helpful, but the DVI file is necessary, because that's the actual
program input.

@cindex shar@r{, recommended}
GNU @code{shar}, available from @url{ftp://prep.ai.mit.edu/pub/gnu} is
a convenient way of packaging multiple (possibly binary) files for
electronic mail.  If you feel your input files are too big to send by
email, you can ftp them to @url{ftp://ftp.tug.org/incoming} (that
directory is writable, but not readable).

@item
@cindex context diff
@cindex sending patches
@flindex ChangeLog @r{entry}
If you are sending a patch (do so if you can!), please do so in the form
of a context diff (@samp{diff -c}) against the original distribution
source.  Any other form of diff is either not as complete or harder for
me to understand.  Please also include a @file{ChangeLog} entry.

@item
@cindex stack trace
@cindex debugger
@cindex crashes, reporting
@cindex core dumps, reporting
@cindex null pointers, dereferencing
@pindex gdb@r{, recommended}
If the bug involved is an actual crash (i.e., core dump), it is easy and
useful to include a stack trace from a debugger (I recommend the GNU
debugger GDB, available from @url{ftp://prep.ai.mit.edu/pub/gnu}).  If
the cause is apparent (a @code{NULL} value being dereferenced, for
example), please send the details along.  If the program involved is
@TeX{} or Metafont, and the crash is happening at apparently-sound code,
however, the bug may well be in the compiler, rather than in the program
or the library (@pxref{TeX or Metafont failing,, @TeX{} or Metafont
failing}).

@item
Any additional information that will be helpful in reproducing,
diagnosing, or fixing the bug.
@end itemize


@node Mailing lists
@subsection Mailing lists

@cindex mailing lists
@cindex bug mailing list
@cindex announcement mailing list

Web2c and Kpathsea in general are discussed on the mailing list
@email{tex-k@@mail.tug.org}.
@flindex tex-k-request@@mail.tug.org
To join, email @email{tex-k-request@@mail.tug.org} with a line
consisting of

@example
subscribe @var{you}@@@var{your.preferred.email.address}
@end example

@noindent in the body of the message.

You do not need to join to submit a report, nor will it affect whether
you get a response.  There is no Usenet newsgroup equivalent (if you can
be the one to set this up, email @samp{tex-k-request}).  Traffic on the
list is fairly light, and is mainly bug reports and enhancement requests
to the software.  The best way to decide if you want to join or not is
read some of the archives from @url{ftp://ftp.tug.org/mail/archives/tex-k/}.

Be aware that large data files are sometimes included in bug reports.
If this is a problem for you, do not join the list.

@flindex tex-archive@@math.utah.edu
@cindex announcement mailing list
If you only want announcements of new releases, not bug reports and
discussion, join @email{tex-archive@@math.utah.edu} (via mail to
@email{tex-archive-request@@math.utah.edu}).

@cindex @TeX{} help mailing list
@cindex La@TeX{} help mailing list
@cindex Usenet @TeX{} newsgroup
@cindex newsgroup for @TeX{}
@cindex help, mailing list for general @TeX{}
@flindex info-tex@@shsu.edu
@flindex comp.text.tex
If you are looking for general @TeX{} help, such as how to use La@TeX{},
please use the mailing list @email{info-tex@@shsu.edu} mailing list,
which is gatewayed to the @samp{comp.text.tex} Usenet newsgroup (or post
to the newsgroup; the gateway is bidirectional).


@node Debugging
@subsection Debugging

@cindex debugging
@cindex runtime debugging
@cindex options for debugging

@vindex kpathsea_debug
@flindex debug.h
Kpathsea provides a number of runtime debugging options, detailed below
by their names and corresponding numeric values.  When the files you
expect aren't being found, the thing to do is enable these options and
examine the output.

You can set these with some runtime argument (e.g., @samp{-d}) to the
program; in that case, you should use the numeric values described in
the program's documentation (which, for Dvipsk and Xdvik, are different
than those below).  It's best to give the @samp{-d} (or whatever) option
first, for maximal output.  Dvipsk and Xdvik have additional
program-specific debugging options as well.

@vindex KPATHSEA_DEBUG
@vindex kpathsea_debug
You can also set the environment variable @code{KPATHSEA_DEBUG}; in this
case, you should use the numbers below.  If you run the program under a
debugger and set the variable @code{kpathsea_debug}, also use the numbers
below.

@kindex -1 @r{debugging value}
In any case, by far the simplest value to use is @samp{-1}, which will
turn on all debugging output.  This is usually better than guessing
which particular values will yield the output you need.

@cindex debugging output
@cindex standard error and debugging output
Debugging output always goes to standard error, so you can redirect it
easily.  For example, in Bourne-compatible shells:
@example
dvips -d -1 @dots{} 2>/tmp/debug
@end example

@cindex Kpsewhich, and debugging
It is sometimes helpful to run the standalone Kpsewhich utility
(@pxref{Invoking kpsewhich}), instead of the original program.

@cindex numeric debugging values
In any case, you can @emph{not} use the @emph{names} below; you must
always use somebody's numbers.  (Sorry.)  To set more than one
option, just sum the corresponding numbers.

@vtable @code
@item KPSE_DEBUG_STAT @r{(1)}
Report @samp{stat}(2) calls. This is useful for verifying that your
directory structure is not forcing Kpathsea to do many additional file
tests (@pxref{Slow path searching}, and @pxref{Subdirectory
expansion}). If you are using an up-to-date @file{ls-R} database
(@pxref{Filename database}), this should produce no output unless a
nonexistent file that must exist is searched for.

@item KPSE_DEBUG_HASH @r{(2)}
Report lookups in all hash tables: @file{ls-R} and @file{aliases}
(@pxref{Filename database}); font aliases (@pxref{Fontmap}); and config
file values (@pxref{Config files}).  Useful when expected values are not
being found, e.g.., file searches are looking at the disk instead of
using @file{ls-R}.

@item KPSE_DEBUG_FOPEN @r{(4)}
@findex fopen@r{, redefined}
Report file openings and closings. Especially useful when your system's
file table is full, for seeing which files have been opened but never
closed. In case you want to set breakpoints in a debugger: this works by
redefining @samp{fopen} (@samp{fclose}) to be @samp{kpse_fopen_trace}
(@samp{kpse_fclose_trace}).

@item KPSE_DEBUG_PATHS @r{(8)}
@tindex kpse_format_info_type
Report general path information for each file type Kpathsea is asked to
search. This is useful when you are trying to track down how a
particular path got defined---from @file{texmf.cnf}, @file{config.ps},
an environment variable, the compile-time default, etc.  This is the
contents of the @code{kpse_format_info_type} structure defined in
@file{tex-file.h}.

@item KPSE_DEBUG_EXPAND @r{(16)}
Report the directory list corresponding to each path element Kpathsea
searches. This is only relevant when Kpathsea searches the disk, since
@file{ls-R} searches don't look through directory lists in this way.

@item KPSE_DEBUG_SEARCH @r{(32)}
Report on each file search: the name of the file searched for, the path
searched in, whether or not the file must exist (when drivers search for
@file{cmr10.vf}, it need not exist), and whether or not we are
collecting all occurrences of the file in the path (as with, e.g.,
@file{texmf.cnf} and @file{texfonts.map}), or just the first (as with
most lookups).  This can help you correlate what Kpathsea is doing with
what is in your input file.

@item KPSE_DEBUG_VARS @r{(64)}
Report the value of each variable Kpathsea looks up.  This is useful for
verifying that variables do indeed obtain their correct values.

@item GSFTOPK_DEBUG @r{(128)}
Activates debugging printout specific to @code{gsftopk} program.

@item MAKETEX_DEBUG @r{(512)}
If you use the optional @code{mktex} programs instead of the
traditional shell scripts, this will report the name of the site file
(@file{mktex.cnf} by default) which is read, directories created by
@code{mktexdir}, the full path of the @file{ls-R} database built by
@code{mktexlsr}, font map searches, @code{MT_FEATURES} in effect,
parameters from @code{mktexnam}, filenames added by
@code{mktexupd}, and some subsidiary commands run by the programs.

@item MAKETEX_FINE_DEBUG @r{(1024)}
When the optional @code{mktex} programs are used, this will print
additional debugging info from functions internal to these programs.
@end vtable

@cindex @samp{kdebug:}
@vindex hash_summary_only @r{variable for debugging}
@cindex hash table buckets, printing
Debugging output from Kpathsea is always written to standard error, and
begins with the string @samp{kdebug:}. (Except for hash table buckets,
which just start with the number, but you can only get that output
running under a debugger. See comments at the @code{hash_summary_only}
variable in @file{kpathsea/db.c}.)


@node Logging
@subsection Logging

@cindex log file

@cindex logging successful searches
@cindex recording successful searches
@cindex usage patterns, finding
@cindex disk usage, reducing
Kpathsea can record the time and filename found for each successful
search.  This may be useful in finding good candidates for deletion when
your filesystem is full, or in discovering usage patterns
at your site.

@vindex TEXMFLOG
To do this, define the environment or config file variable
@code{TEXMFLOG}.  The value is the name of the file to append the
information to.  The file is created if it doesn't exist, and appended
to if it does.

@cindex epoch, seconds since
@findex time @r{system call}
Each successful search turns into one line in the log file: two words
separated by a space. The first word is the time of the search, as the
integer number of seconds since ``the epoch'', i.e., UTC midnight 1
January 1970 (more precisely, the result of the @code{time} system
call). The second word is the filename.

For example, after @code{setenv TEXMFLOG /tmp/log}, running Dvips on
@file{story.dvi} appends the following lines:

@example
774455887 /usr/local/share/texmf/dvips/config.ps
774455887 /usr/local/share/texmf/dvips/psfonts.map
774455888 /usr/local/share/texmf/dvips/texc.pro
774455888 /usr/local/share/texmf/fonts/pk/ljfour/public/cm/cmbx10.600pk
774455889 /usr/local/share/texmf/fonts/pk/ljfour/public/cm/cmsl10.600pk
774455889 /usr/local/share/texmf/fonts/pk/ljfour/public/cm/cmr10.600pk
774455889 /usr/local/share/texmf/dvips/texc.pro
@end example

@cindex privacy, semblance of
@noindent Only filenames that are absolute are recorded, to preserve
some semblance of privacy.


@node Common problems
@subsection Common problems

@cindex common problems
@cindex problems, common
@cindex FAQ, Kpathsea

Here are some common problems with configuration, compilation, linking,
execution, @dots{}

@menu
* Unable to find files::        If your program can't find fonts (or whatever).
* Slow path searching::         If it takes forever to find anything.
* Unable to generate fonts::    If mktexpk fails.
* TeX or Metafont failing::     Likely compiler bugs.

* Empty Makefiles::            	When configure produces empty makefiles.
* XtStrings::                   When _XtStrings is undefined.
* dlopen::                      When dlopen is undefined.
* ShellWidgetClass::            For dynamic linking troubles under OpenWindows.
* Pointer combination warnings::  For old compilers that don't grok char *.
@end menu

@node Unable to find files
@subsubsection Unable to find files

@cindex unable to find files
@cindex files, unable to find

If a program complains it cannot find fonts (or other input files), any
of several things might be wrong.  In any case, you may find the
debugging options helpful.  @xref{Debugging}.

@itemize @bullet
@item
Perhaps you simply haven't installed all the necessary files; the basic
fonts and input files are distributed separately from the programs.
@xref{unixtex.ftp}.

@item
@flindex /etc/profile
@cindex environment variables, old
You have (perhaps unknowingly) told Kpathsea to use search paths that
don't reflect where the files actually are.  One common cause is having
environment variables set from a previous installation, thus overriding
what you carefully set in @file{texmf.cnf} (@pxref{Supported file
formats}).  System @file{/etc/profile} or other files such may be the
culprit.

@item
@cindex symbolic links not found
@cindex leaf directories wrongly guessed
Your files reside in a directory that is only pointed to via a symbolic
link, in a leaf directory and is not listed in @file{ls-R}.

Unfortunately, Kpathsea's subdirectory searching has an irremediable
deficiency: If a directory @var{d} being searched for subdirectories
contains plain files and symbolic links to other directories, but no
true subdirectories, @var{d} will be considered a leaf directory, i.e.,
the symbolic links will not be followed.  @xref{Subdirectory expansion}.

You can work around this problem by creating an empty dummy subdirectory
in @var{d}. Then @var{d} will no longer be a leaf, and the symlinks will
be followed.

The directory immediately followed by the @samp{//} in the path
specification, however, is always searched for subdirectories, even if
it is a leaf.  Presumably you would not have asked for the directory to
be searched for subdirectories if you didn't want it to be.

@item
If the fonts (or whatever) don't already exist, @code{mktexpk} (or
@code{mktexmf} or @code{mktextfm}) will try to create them.  If
these rather complicated shell scripts fail, you'll eventually get an
error message saying something like @samp{Can't find font
@var{fontname}}. The best solution is to fix (or at least report) the
bug in @code{mktexpk}; the workaround is to generate the necessary
fonts by hand with Metafont, or to grab them from a CTAN site
(@pxref{unixtex.ftp}).

@item
There is a bug in the library. @xref{Reporting bugs}.
@end itemize


@node Slow path searching
@subsubsection Slow path searching

@cindex excessive startup time
@cindex slow startup time
@cindex startup time, excessive

If your program takes an excessively long time to find fonts or other
input files, but does eventually succeed, here are some possible culprits:

@itemize @bullet
@item
Most likely, you just have a lot of directories to search, and that
takes a noticeable time. The solution is to create and maintain a
separate @file{ls-R} file that lists all the files in your main @TeX{}
hierarchy.  @xref{Filename database}.  Kpathsea always uses @file{ls-R}
if it's present; there's no need to recompile or reconfigure any of the
programs.

@item
Your recursively-searched directories (e.g.,
@file{/usr/local/share/texmf/fonts//}), contain a mixture of files and
directories. This prevents Kpathsea from using a useful optimization
(@pxref{Subdirectory expansion}).

It is best to have only directories (and perhaps a @file{README}) in the
upper levels of the directory structure, and it's very important to have
@emph{only} files, and no subdirectories, in the leaf directories where
the dozens of TFM, PK, or whatever files reside.
@end itemize

In any case, you may find the debugging options helpful in determining
precisely when the disk or network is being pounded.  @xref{Debugging}.


@node Unable to generate fonts
@subsubsection Unable to generate fonts

@cindex unable to generate fonts
@cindex font generation failures

This can happen if either @code{mktexpk} hasn't been installed
properly, or if the local installation of Metafont isn't correct.

@cindex Metafont installation
If @code{mf} is a command not found by @code{mktexpk}, then you need
to install Metafont (@pxref{unixtex.ftp}).

@cindex Metafont using the wrong resolution
@cindex resolution, wrong
If Metafont runs, but generates fonts at the wrong resolution, you need
to be sure the @samp{M} and @samp{D} lines in your Dvips configuration
file match (@pxref{Config files,,, dvips, Dvips}).  For example, if
@code{mktexpk} is generating 300@dmn{dpi} fonts, but you need
600@dmn{dpi} fonts, you should have:
@example
M ljfour
D 600
@end example

@flindex .2602gf
@flindex 2602gf
@cindex Metafont making too-large fonts
@cindex proof mode
@cindex online Metafont display, spurious
If Metafont runs but generates fonts at a resolution of 2602@dmn{dpi}
(and prints out the name of each character as well as just a character
number, and maybe tries to display the characters), then your Metafont
base file probably hasn't been made properly.  (It's using the default
@code{proof} mode, instead of an actual device mode.)  To make a proper
@file{plain.base}, assuming the local mode definitions are contained in
a file @file{modes.mf}, run the following command (assuming Unix):

@example
inimf "plain; input modes; dump"
@end example

@noindent
@flindex plain.base
Then copy the @file{plain.base} file from the current directory to where
the base files are stored on your system
(@file{/usr/local/share/texmf/web2c} by default), and make a link
(either hard or soft) from @file{plain.base} to @file{mf.base} in that
directory.
@xref{inimf invocation,,, web2c, Web2c}.


@node TeX or Metafont failing
@subsubsection @TeX{} or Metafont failing

@cindex @TeX{} failures
@cindex Metafont failures
@cindex compiler bugs
If @TeX{} or Metafont get a segmentation fault or otherwise fail while
running a normal input file, the problem is usually a compiler bug
(unlikely as that may sound).  Even if the trip and trap tests are
passed, problems may lurk.  Optimization occasionally causes trouble in
programs other than @TeX{} and Metafont themselves, too.

Insufficient swap space may also cause core dumps or other erratic
behavior.

@cindex optimization caveat
For a workaround, if you enabled any optimization flags, it's best to
omit optimization entirely.  In any case, the way to find the facts is
to run the program under the debugger and see where it's failing.

@cindex GNU C compiler bugs
@cindex system C compiler bugs
Also, if you have trouble with a system C compiler, I advise trying the
GNU C compiler. And vice versa, unfortunately; but in that case I also
recommend reporting a bug to the GCC mailing list; see @ref{Bugs,,, gcc,
Using and Porting GNU CC}.

@cindex compiler bugs, finding
To report compiler bugs effectively requires perseverance and
perspicacity: you must find the miscompiled line, and that usually
involves delving backwards in time from the point of error, checking
through @TeX{}'s (or whatever program's) data structures.  Things are
not helped by all-too-common bugs in the debugger itself.  Good luck.

@cindex ANSI C
@cindex HP-UX, compiling on
@cindex compiling on HP-UX
One known cause of trouble is the way arrays are handled.  Some of the
Pascal arrays have a lower index other than 0, and the C code will take
the pointer to the allocated memory, subtract the lower index, and use
the resulting pointer for the array.  While this trick often works, ANSI
C doesn't guarantee that it will.  It it known to fail on HP-UX 10
mchines when the native compiler is used, unless the @samp{+u} compiler
switch was specified.  Using GCC will work on this platform as well.

@node Empty Makefiles
@subsubsection Empty Makefiles

@cindex Makefiles, empty
@pindex sed @r{error from @code{configure}}
@pindex configure @r{error from @code{sed}}
@cindex NetBSD @code{configure} error
@cindex FreeBSD @code{configure} error
@cindex Mach10 @code{configure} error
@cindex AIX 4.1 @code{configure} error
@cindex NeXT @code{sed} error

On some systems (NetBSD, FreeBSD, AIX 4.1, and Mach10), @code{configure}
may fail to properly create the Makefiles. Instead, you get an error
which looks something like this:

@example
prompt$ ./configure
@dots{}
creating Makefile
sed: 1: "\\@@^ac_include make/pat ...": \ can not be used as a string delimiter
@end example

So far as I know, the bug here is in @code{/bin/sh} on these systems. I
don't have access to a machine running any of them, so if someone can
find a workaround that avoids the quoting bug, I'd be most
grateful. (Search for @code{ac_include} in the @code{configure} script
to get to the problematic code.)

It should work to run @code{bash configure}, instead of using
@code{/bin/sh}. You can get Bash from
@url{ftp://prep.ai.mit.edu/pub/gnu} and mirrors.

Another possible cause (reported for NeXT) is a bug in the @code{sed}
command.  In that case the error may look like this:

@example
Unrecognized command: \@@^ac_include make/paths.make@@r make/paths.make
@end example

In this case, installing GNU @code{sed} should solve the problem.  You
can get GNU @code{sed} from the same places as Bash.

@ignore
@node wchar_t
@subsubsection @code{wchar_t}

@vindex FOIL_X_WCHAR_T
@tindex wchar_t

The upshot of all the following is that if you get error messages
regarding @code{wchar_t}, try defining @code{NO_FOIL_X_WCHAR_T} (for
Web2c) or @code{FOIL_X_WCHAR_T} (for everything else), as in:

@example
make XCFLAGS=-DNO_FOIL_X_WCHAR_T @var{other-make-options}
@end example

@flindex Xlib.h
@flindex stddef.h
@code{wchar_t} has caused infinite trouble. None of my code ever uses
@code{wchar_t}; all I want to do is include X header files and various
system header files, possibly compiling with GCC. This seems an
impossible task! The basic problem is that the X11 header
@file{<Xlib.h>} and GCC's @file{<stddef.h>} have conflicting definitions
for @code{wchar_t}.

The particulars: @file{<X11/Xlib.h>} from MIT X11R5 defines
@code{wchar_t} if @code{X_WCHAR} is defined, which is defined if
@code{X_NOT_STDC_ENV} is defined, and we define @emph{that} if
@code{STDC_HEADERS} is not defined (@samp{configure} decides if
@code{STDC_HEADERS} gets defined).  But when compiling with GCC on SunOS
4.1.x, @code{STDC_HEADERS} is not defined (@file{string.h} doesn't
declare the @samp{mem}* functions), so we do get X's
@code{wchar_t}---and we also get GCC's @code{wchar_t} from its
@file{<stddef.h>}.  Conflict.

On the other hand, SunOS 4.1.1 with some other X configurations actually
needs GCC to define @code{wchar_t}, and fails otherwise.

My current theory is to define @code{wchar_t} to a nonsense symbol
before the X include files are read; that way its definition (if any)
will be ignored by other system include files.  Going along with that,
define @code{X_WCHAR} to tell X not to use @file{<stddef.h>}, that we've
already included, but instead to make its own definition.

But this is not the end of the story. The X11 include files distributed
with DG/UX 5.4.2 for the Aviion have been modified to include
@file{<_int_wchar_t.h>} if @code{X_WCHAR}, so our @code{#define} will
not have any typedef to change---but the uses of @code{wchar_t} in the X
include files will be changed to reference this undefined symbol. So
there's nothing to foil in this case. I don't know how to detect this
automatically, so it's up to you to define @code{NO_FOIL_X_WCHAR_T}
yourself.
@end ignore

@node XtStrings
@subsubsection @code{XtStrings}

@findex XtStrings
You may find that linking X programs results in an error from the linker
that @samp{XtStrings} is undefined, something like this:

@example
gcc -o virmf @dots{}
@dots{}/x11.c:130: undefined reference to `XtStrings'
@end example

This generally happens because of a mismatch between the X include files
with which you compiled and the X libraries with which you linked;
often, the include files are from MIT and the libraries from Sun.

The solution is to use the same X distribution for compilation and
linking.  Probably @samp{configure} was unable to guess the proper
directories from your installation.  You can use the @code{configure}
options @samp{--x-includes=@var{path}} and
@samp{--x-libraries=@var{path}} to explicitly specify them.


@node dlopen
@subsubsection @code{dlopen}

@cindex static linking and @code{dlsym}
@flindex dlopen
@flindex dlsym
@flindex dlclose
@flindex wcstombs
@flindex libdl.a
(This section adapted from the file @file{dlsym.c} in the X distribution.)

The @code{Xlib} library uses the standard C function @code{wcstombs}.
Under SunOS 4.1, @code{wcstombs} uses the @samp{dlsym} interface defined
in @file{libdl.so}.  Unfortunately, the SunOS 4.1 distribution does not
include a static @samp{libdl.a} library.

As a result, if you try to link an X program statically under SunOS, you
may get undefined references to @code{dlopen}, @code{dlsym}, and
@code{dlclose}.  One workaround is to include these definitions
when you link:

@example
void *dlopen() @{ return 0; @}
void *dlsym()  @{ return 0; @}
int dlclose()  @{ return -1; @}
@end example

@flindex dlsym.c
@noindent These are contained in the @file{dlsym.c} file in the MIT X
distribution.


@node ShellWidgetClass
@subsubsection @code{ShellWidgetClass}

@cindex dynamic linking problems with OpenWin libraries
@cindex OpenWin libraries, dynamic linking problems
@findex get_wmShellWidgetClass
@findex get_applicationShellWidgetClass

@flindex comp.sys.sun.admin @r{FAQ}
@cindex FAQ, @t{comp.sys.sun.admin}
(This section adapted from the @t{comp.sys.sun.admin} FAQ.)

If you are linking with Sun's OpenWindows libraries in SunOS 4.1.x, you
may get undefined symbols @code{_get_wmShellWidgetClass} and
@code{_get_applicationShellWidgetClass} when linking. This problem does
not arise using the standard MIT X libraries under SunOS.

@findex Xmu @r{library problems}
The cause is bugs in the @code{Xmu} shared library as shipped from Sun.
There are several fixes:

@itemize @bullet

@item Install the free MIT distribution from @samp{ftp.x.org} and mirrors.

@item Get the OpenWindows patches listed below.

@item Statically link the @code{Xmu} library into the executable.

@item Avoid using @code{Xmu} at all. If you are compiling
Metafont, see @ref{Online Metafont graphics,,, web2c, Web2c}. If you are
compiling Xdvi, see the @code{-DNOTOOL} option in @file{xdvik/INSTALL}.

@item Ignore the errors. The binary runs fine regardless.

@end itemize

@cindex Sun OpenWin patches
@cindex patches, Sun OpenWin
Here is the information for getting the two patches:

@display
Patch ID: 100512-02
Bug ID's: 1086793, 1086912, 1074766
Description: 4.1.x OpenWindows 3.0 @code{libXt} jumbo patch

Patch ID: 100573-03
Bug ID: 1087332
Description: 4.1.x OpenWindows 3.0 undefined symbols when using shared @code{libXmu}.
@end display

@cindex static linking
The way to statically link with @code{libXmu} depends on whether you are
using a Sun compiler (e.g., @code{cc}) or @code{gcc}. If the latter,
alter the @code{x_libs} Make variable to include

@opindex -static
@opindex -dynamic
@example
-static -lXmu -dynamic
@end example

@opindex -Bstatic
@opindex -Bdynamic
If you are using the Sun compiler, use @samp{-Bstatic} and @samp{-Bdynamic}.


@node Pointer combination warnings
@subsubsection Pointer combination warnings

@cindex warnings, pointer combinations
@cindex pointer combination warnings
@cindex illegal pointer combination warnings
@pindex cc @r{warnings}
When compiling with old C compilers, you may get some warnings about
``illegal pointer combinations''.  These are spurious; just ignore them.
I decline to clutter up the source with casts to get rid of them.

@c This isn't worth including any more, OSF 1.x is too old.
@c The other XtInherit problem (R4 Xlib on Suns) should never come up,
@c but the answer from the X faq is included anyway.
@c
@c @node XtInherit
@c @subsubsection @code{XtInherit}
@c 
@c @findex XtInherit @r{bug on OSF/1}
@c @cindex OSF/1 loader bug and @code{XtInherit}
@c @cindex Alpha OSF/1 loader bug and @code{XtInherit}
@c 
@c On DEC OSF/1 1.x systems, the loader has a bug that manifests itself in
@c the following error (all on one line, but for the sake of the paper
@c width it's broken here):
@c 
@c @example
@c xdvik/xdvi: /sbin/loader: Fatal Error: search_for_undefineds: 
@c      symbol _XtInherit should not have any relocation entry
@c @end example
@c 
@c @noindent According to Michael Rickabaugh @code{<mjr@@quarry.enet.dec.com>}:
@c 
@c @quotation
@c This is a bug fixed in DEC OSF/1 2.0.
@c 
@c If you know how, installing @file{/sbin/loader} from a 2.0 system onto a
@c 1.3 system will work.  Make sure that @file{/usr} is @emph{not} mounted
@c when you do this.  (If you forget about umounting @code{/usr}, it is
@c possible most of your filesystems will become corrupted.)
@c 
@c Otherwise, I suggest getting a later CD and running
@c @file{/usr/sbin/installupdate}.
@c @end quotation
@c 
@c Alternatively, you may be able to use the freely available X11 libraries
@c that come with the MIT distribution (on @file{ftp.x.org}, for example).
@c 
@c Linking statically, perhaps only with some of the X libraries, may also
@c work.
@c 
@c The Sun XtInherit weirdness, from the comp.windows.x FAQ:
@c Subject: 126)! What are these problems with "*_XtInherit* not found" on the Sun?
@c When I link a X program that I wrote on a SunOS 4.0.3 or 4.1 machine I get the 
@c error "ld.so: symbol not found *_XtInherit*".
@c 
@c      What you are seeing is a side-effect of a kludge in the R4 libXt.a to 
@c get Sun shared libraries working.  Apparently, you can't share a function that 
@c is both called and compared, as *_XtInherit* is. This was handled by putting 
@c *_XtInherit* in the same file as a function that is always used, thereby 
@c guaranteeing that it would be loaded -- that is, in Initialize.c, where 
@c XtToolkitInitialize() and XtInitialize() reside. These routines would normally
@c be called.
@c 
@c      You are probably seeing this error because your program is not a normal
@c Xt-based program and does not call XtToolkitInitialize() anywhere. 
@c      1) it may be a program that uses Xt functions but never opens a 
@c connection to the X server.  [OSF/Motif's 1.1.0 UIL had this problem; it called
@c XtMalloc() and other Xt functions.] The solution is to add the call to your 
@c program; the function does not have to be executed, just linked in.
@c      2) alternatively, your program doesn't need any Xt functions and is
@c correct in not calling XtToolkitInitialize() -- it may be an Xlib or XView 
@c program. In this case, you can remove -lXt from your link command. 
@c 
@c      It should not be necessary to link the shared libraries statically,
@c although this will certainly solve the problem.

@c * Empty Makefiles::             If configure gives you sed errors.
@c * wchar_t::                     For wchar_t difficulties.
