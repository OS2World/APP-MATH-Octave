% Modified by Klaus Gebhardt, 1997
\input texinfo.tex      @c -*-texinfo-*-

@setfilename oct-faq.
@settitle Frequently asked questions about Octave (with answers)

@setchapternewpage off

@titlepage
@title Octave FAQ
@subtitle Frequently asked questions about Octave
@subtitle February 14, 1998
@sp 1
@author John W. Eaton
@page
@end titlepage

@ifinfo
@node Top, What is Octave?, (dir), (dir)
@top
@unnumbered Preface
@cindex FAQ for Octave, latest version

@macro email{text}
(\text\)
@end macro

@macro url{text}
(\text\)
@end macro
@end ifinfo

This is a list of frequently asked questions (FAQ) for Octave users.

Some information in this FAQ was written for earlier versions of
Octave and may now be obsolete.

I'm looking for new questions (@emph{with} answers), better answers,
or both.  Please send suggestions to bug-octave@@bevo.che.wisc.edu.
If you have general questions about Octave, or need help for something
that is not covered by the Octave manual or the FAQ, please use the
help-octave@@bevo.che.wisc.edu mailing list.

This FAQ is intended to supplement, not replace, the Octave manual.
Before posting a question to the help-octave mailing list, you should
first check to see if the topic is covered in the manual.

@menu
* What is Octave?::             
* Version 2.0::               
* Octave Features::             
* Documentation::               
* Getting Octave::              
* Installation::                
* Common problems::             
* Getting additional help::     
* Bug reports::                 
* MATLAB compatibility::        
* Index::                       
@end menu

@node What is Octave?, Version 2.0, Top, Top
@chapter What is Octave?

Octave is a high-level interactive language, primarily intended for
numerical computations that is mostly compatible with
@sc{Matlab}.@footnote{@sc{Matlab} is a registered trademark of The MathWorks,
Inc.}

Octave can do arithmetic for real and complex scalars and matrices,
solve sets of nonlinear algebraic equations, integrate functions over
finite and infinite intervals, and integrate systems of ordinary
differential and differential-algebraic equations.

Octave uses the GNU readline library to handle reading and editing
input.  By default, the line editing commands are similar to the
cursor movement commands used by GNU Emacs, and a vi-style line
editing interface is also available.  At the end of each session, the
command history is saved, so that commands entered during previous
sessions are not lost.

The Octave distribution includes a 200+ page Texinfo manual.  Access
to the complete text of the manual is available via the help command
at the Octave prompt.

Two and three dimensional plotting is fully supported using gnuplot.

The underlying numerical solvers are currently standard Fortran ones
like Lapack, Linpack, Odepack, the Blas, etc., packaged in a library
of C++ classes.  If possible, the Fortran subroutines are compiled
with the system's Fortran compiler, and called directly from the C++
functions.  If that's not possible, you can still compile Octave if
you have the free Fortran to C translator f2c.

Octave is also free software; you can redistribute it and/or modify it
under the terms of the GNU General Public License as published by the
Free Software Foundation.

@node Version 2.0, Octave Features, What is Octave?, Top
@chapter What's new in version 2.0 of Octave

Version 2.0.10 of Octave was released February 6, 1998.  Most bugs
reported since version 2.0 was release have been fixed, and a number of
new features have been added.  Octave is now much more compatible with
@sc{Matlab}.

A list of user-visible changes in recent versions of Octave may be found
in the file NEWS, distributed in both source and binary releases of
Octave.

@node Octave Features, Documentation, Version 2.0, Top
@chapter What features are unique to Octave?

@menu
* Command and variable name completion::  
* Command history::             
* Data structures::             
* Short-circuit boolean operators::  
* Increment and decrement operators::  
* Unwind-protect::              
* Variable-length argument lists::  
* Variable-length return lists::  
* Built-in ODE and DAE solvers::  
@end menu

@node Command and variable name completion, Command history, Octave Features, Octave Features
@section Command and variable name completion

@cindex Command completion
@cindex Function name completion
@cindex Variable name completion
@cindex Name completion

Typing a TAB character (ASCII code 9) on the command line causes Octave
to attempt to complete variable, function, and file names.  Octave uses
the text before the cursor as the initial portion of the name to
complete.

For example, if you type @samp{fu} followed by TAB at the Octave prompt,
Octave will complete the rest of the name @samp{function} on the command
line (unless you have other variables or functions defined that begin
with the characters @samp{fu}).  If there is more than one possible
completion, Octave will ring the terminal bell to let you know that your
initial sequence of characters is not enough to specify a unique name.
To complete the name, you may either edit the initial character sequence
(usually adding more characters until completion is possible) or type
another TAB to cause Octave to display the list of possible completions.

@node Command history, Data structures, Command and variable name completion, Octave Features
@section Command history

@cindex Command history
@cindex History

When running interactively, Octave saves the commands you type in an
internal buffer so that you can recall and edit them.  Emacs and vi
editing modes are available with Emacs keybindings enabled by default.

When Octave exits, the current command history is saved to the file
@file{~/.octave_hist}, and each time Octave starts, it inserts the
contents of the @file{~/.octave_hist} file in the history list so that
it is easy to begin working where you left off.

@node Data structures, Short-circuit boolean operators, Command history, Octave Features
@section Data structures

@cindex Data structures
@cindex Structures

Octave includes a limited amount of support for organizing data in
structures.  The current implementation uses an associative array
with indices limited to strings, but the syntax is more like C-style
structures.  Here are some examples of using data structures in Octave.

@itemize @bullet
@item Elements of structures can be of any value type.

@example
@group
octave:1> x.a = 1; x.b = [1, 2; 3, 4]; x.c = "string";
octave:2> x.a
x.a = 1
octave:3> x.b
x.b =

  1  2
  3  4

octave:4> x.c
x.c = string
@end group
@end example

@item Structures may be copied.

@example
@group
octave:1> y = x
y =
@{
  a = 1
  b =

    1  2
    3  4

  c = string
  s =

    0.00000  0.00000  0.00000
    0.00000  5.46499  0.00000
    0.00000  0.00000  0.36597

  u =

    -0.40455  -0.91451
    -0.91451   0.40455

  v =

    -0.57605   0.81742
    -0.81742  -0.57605
@}
@end group
@end example

@item Structure elements may reference other structures.

@example
@group
octave:1> x.b.d = 3
x.b.d = 3
octave:2> x.b
ans =
@{
  d = 3
@}
octave:3> x.b.d
ans = 3
@end group
@end example

@item Functions can return structures.

@example
@group
octave:1> function y = f (x)
> y.re = real (x);
> y.im = imag (x);
> endfunction

octave:2> f (rand + rand*I);
ans =
@{
  im = 0.18033
  re = 0.19069
@}
@end group
@end example

@item Function return lists can include structure elements, and they may
be indexed like any other variable.

@example
@group
octave:1> [x.u, x.s(2:3,2:3), x.v] = svd ([1, 2; 3, 4]);
octave:2> x
x =
@{
  s =

    0.00000  0.00000  0.00000
    0.00000  5.46499  0.00000
    0.00000  0.00000  0.36597

  u =

    -0.40455  -0.91451
    -0.91451   0.40455

  v =

    -0.57605   0.81742
    -0.81742  -0.57605
@}
@end group
@end example

@item You can also use the function @code{is_struct} to determine
whether a given value is a data structure.  For example

@example
is_struct (x)
@end example

@noindent
returns 1 if the value of the variable @var{x} is a data structure.
@end itemize

This feature should be considered experimental, but you should expect it
to work.  Suggestions for ways to improve it are welcome.

@node Short-circuit boolean operators, Increment and decrement operators, Data structures, Octave Features
@section Short-circuit boolean operators

@cindex Boolean operators, short-circuit
@cindex Logical operators, short-circuit
@cindex Short-circuit boolean operators
@cindex Operators, boolean

Octave's @samp{&&} and @samp{||} logical operators are evaluated in
a short-circuit fashion (like the corresponding operators in the C
language) and work differently than the element by element operators
@samp{&} and @samp{|}.

@node Increment and decrement operators, Unwind-protect, Short-circuit boolean operators, Octave Features
@section Increment and decrement operators

@cindex Increment operators
@cindex Decrement operators
@cindex Operators, increment
@cindex Operators, decrement

Octave includes the C-like increment and decrement operators @samp{++}
and @samp{--} in both their prefix and postfix forms.

For example, to pre-increment the variable @var{x}, you would write
@code{++@var{x}}.  This would add one to @var{x} and then return the new
value of @var{x} as the result of the expression.  It is exactly the
same as the expression @code{@var{x} = @var{x} + 1}.

To post-increment a variable @var{x}, you would write @code{@var{x}++}.
This adds one to the variable @var{x}, but returns the value that
@var{x} had prior to incrementing it.  For example, if @var{x} is equal
to 2, the result of the expression @code{@var{x}++} is 2, and the new
value of @var{x} is 3.

For matrix and vector arguments, the increment and decrement operators
work on each element of the operand.

It is not currently possible to increment index expressions.  For
example, you might expect that the expression @code{@var{v}(4)++} would
increment the fourth element of the vector @var{v}, but instead it
results in a parse error.  This problem may be fixed in a future
release of Octave.

@node Unwind-protect, Variable-length argument lists, Increment and decrement operators, Octave Features
@section Unwind-protect

@cindex Unwind-protect

Octave supports a limited form of exception handling modelled after the
unwind-protect form of Lisp.  The general form of an
@code{unwind_protect} block looks like this:

@example
@group
unwind_protect
  @var{body}
unwind_protect_cleanup
  @var{cleanup}
end_unwind_protect
@end group
@end example

@noindent
Where @var{body} and @var{cleanup} are both optional and may contain any
Octave expressions or commands.  The statements in @var{cleanup} are 
guaranteed to be executed regardless of how control exits @var{body}.

The @code{unwind_protect} statement is often used to reliably restore
the values of global variables that need to be temporarily changed.

@node Variable-length argument lists, Variable-length return lists, Unwind-protect, Octave Features
@section Variable-length argument lists

@cindex Variable-length argument lists
@cindex Argument lists, variable-length

Octave has a real mechanism for handling functions that take an
unspecified number of arguments, so it is no longer necessary to place
an upper bound on the number of optional arguments that a function can
accept.

Here is an example of a function that uses the new syntax to print a
header followed by an unspecified number of values:

@example
@group
function foo (heading, ...)
  disp (heading);
  va_start ();
  while (--nargin)
    disp (va_arg ());
  endwhile
endfunction
@end group
@end example

Calling @code{va_start()} positions an internal pointer to the first
unnamed argument and allows you to cycle through the arguments more than
once.  It is not necessary to call @code{va_start()} if you do not plan
to cycle through the arguments more than once.

The function @code{va_arg()} returns the value of the next available
argument and moves the internal pointer to the next argument.  It is an
error to call @code{va_arg()} when there are no more arguments
available.

It is also possible to use the keyword @var{all_va_args} to pass all
unnamed arguments to another function.

@node Variable-length return lists, Built-in ODE and DAE solvers, Variable-length argument lists, Octave Features
@section Variable-length return lists

@cindex Variable-length return lists
@cindex Return lists, variable-length

Octave also has a real mechanism for handling functions that return an
unspecified number of values, so it is no longer necessary to place an
upper bound on the number of outputs that a function can produce.

Here is an example of a function that uses the new syntax to produce
@samp{N} values:

@example
@group
function [...] = foo (n)
  for i = 1:n
    vr_val (i);
  endfor
endfunction
@end group
@end example

@node Built-in ODE and DAE solvers,  , Variable-length return lists, Octave Features
@section Built-in ODE and DAE solvers

@cindex DASSL
@cindex LSODE

Octave includes LSODE and DASSL for solving systems of stiff ordinary
differential and differential-algebraic equations.  These functions are
built in to the interpreter.

@node Documentation, Getting Octave, Octave Features, Top
@chapter What documentation exists for Octave?

@cindex Octave, documentation

The Octave distribution includes a 220+ page manual that is also
distributed under the terms of the GNU GPL.

The Octave manual is intended to be a complete reference for Octave, but
it is not a finished document.  If you have problems using it, or find
that some topic is not adequately explained, indexed, or
cross-referenced, please send a bug report to bug-octave@@bevo.che.wisc.edu.

Because the Octave manual is written using Texinfo, the complete text of
the Octave manual is also available on line using the GNU Info system
via the GNU Emacs, info, or xinfo programs, or by using the @samp{help -i} 
command to start the GNU info browser directly from the Octave prompt.

It is also possible to use your favorite WWW browser to read the Octave
manual by converting the Texinfo source to HTML using the
@code{texi2html} program.

@node Getting Octave, Installation, Documentation, Top
@chapter Obtaining Source Code

@cindex Source code

@menu
* Octave for Unix::             
* Octave for other platforms::  
* latest versions::             
@end menu

@node Octave for Unix, Octave for other platforms, Getting Octave, Getting Octave
@section How do I get a copy of Octave for Unix?

You can get Octave from a friend who has a copy, by anonymous FTP, or by
ordering a tape or CD-ROM from the Free Software Foundation (FSF).

@cindex Octave, ordering
@cindex Octave, getting a copy

Octave was not developed by the FSF, but the FSF does distribute Octave,
and the developers of Octave support the efforts of the FSF by
encouraging users of Octave to order Octave on CD-ROM directly from
the FSF.

The FSF is a nonprofit organization that distributes software and
manuals to raise funds for more GNU development.  Buying a CD-ROM from
the FSF contributes directly to paying staff to develop GNU software.
CD-ROMs cost $240 if an organization is buying, or $60 if an individual
is buying.

@cindex FSF [Free Software Foundation]
@cindex GNU [GNU's not unix]

For more information about ordering from the FSF, contact
gnu@@gnu.org, phone (617) 542-5942 or anonymous ftp the file
@file{/pub/gnu/GNUinfo/ORDERS} from ftp.gnu.org.

@cindex FSF, contact <gnu@@gnu.org>
@cindex GNUware, anonymous FTP sites

If you are on the Internet, you can copy the latest distribution
version of Octave from the file @file{/pub/octave/octave-M.N.tar.gz}, on
the host @file{ftp.che.wisc.edu}.  This tar file has been compressed
with GNU gzip, so be sure to use binary mode for the transfer.  @samp{M}
and @samp{N} stand for version numbers; look at a listing of the
directory through ftp to see what version is available.  After you
unpack the distribution, be sure to look at the files @file{README} and
@file{INSTALL}.

Binaries for several popular systems are also available.  If you would
like help out by making binaries available for other systems, please
contact bug-octave@@bevo.che.wisc.edu.

A list of user-visible changes since the last release is available in
the file @file{NEWS}.  The file @file{ChangeLog} in the source
distribution contains a more detailed record of changes made since the
last release.

@node Octave for other platforms, latest versions, Octave for Unix, Getting Octave
@section How do I get a copy of Octave for (some other platform)?

@cindex VMS support
@cindex VAX
@cindex MS-DOS support
@cindex Windows support
@cindex DJGPP
@cindex EMX
@cindex OS/2 support

Octave currently runs on Unix-like systems, OS/2, and Windows NT/95
(using the gnu-win32 tools from Cygnus Support).  It should be possible
to make Octave work on other systems as well.  If you are interested in
porting Octave to other systems, please contact
bug-octave@@bevo.che.wisc.edu.

@node latest versions,  , Octave for other platforms, Getting Octave
@section What is the latest version of Octave

@cindex Octave, version date

The latest version of Octave is 2.0.10, released February 6, 1998.

@node Installation, Common problems, Getting Octave, Top
@chapter Installation Issues and Problems

@cindex Octave, building 

Octave requires approximately 125MB of disk storage to unpack and
compile from source (significantly less if you don't compile with
debugging symbols or create shared libraries).  Once installed, Octave
requires approximately 65MB of disk space (again, considerably less if
you don't build shared libraries or the binaries and libraries do not
include debugging symbols).

@menu
* What else do I need?::        
* Other C++ compilers?::        
@end menu

@node What else do I need?, Other C++ compilers?, Installation, Installation
@section What else do I need?

@cindex GNU gcc
@cindex GNU g++
@cindex libg++
@cindex GNU Make
@cindex Flex
@cindex GNU Bison

To compile Octave, you will need a recent version of GNU Make.  You
will also need g++ 2.7.2 or later.  Version 2.8.0 or egcs 1.0.x should
work.  Later versions may work, but C++ is still evolving, so don't be
too surprised if you run into some trouble.

It is no longer necessary to have libg++, but you do need to have the
GNU implementation of libstdc++.  If you are using g++ 2.7.2,
libstdc++ is distributed along with libg++, but for later versions,
libstdc++ is distributed separately.  For egcs, libstdc++ is included
with the compiler distribution.

<em>You must have gnu make to compile octave</em>.  Octave's Makefiles
use features of GNU Make that are not present in other versions of make.
GNU Make is very portable and easy to install.

@node Other C++ compilers?,  , What else do I need?, Installation
@section Can I compile Octave with another C++ compiler?

Currently, Octave can only be compiled with the GNU C++ compiler.  It
would be nice to make it possible to compile Octave with other C++
compilers, but the maintainers do not have sufficient time to devote to
this.  If you are interested in working to make Octave portable to other
compilers, please contact bug-octave@@bevo.che.wisc.edu.

@node Common problems, Getting additional help, Installation, Top
@chapter Common problems

This list is probably far too short.  Feel free to suggest additional
questions (preferably with answers!)

@itemize @bullet
@item
Octave takes a long time to find symbols.

Octave is probably spending this time recursively searching directories
for function files.  Check the value of your LOADPATH.  For those
elements that end in @samp{//}, do any name a very large directory tree?
Does it contain directories that have a mixture of files and
directories?  In order for the recursive directory searching code to
work efficiently, directories that are to be searched recursively should
have either function files only, or subdirectories only, but not a
mixture of both.  Check to make sure that Octave's standard set of
function files is installed this way.
@end itemize

@node Getting additional help, Bug reports, Common problems, Top
@chapter Getting additional help

@cindex Additional help
@cindex Mailing lists, help-octave

The mailing list

@example
help-octave@@bevo.che.wisc.edu
@end example

@noindent
is available for questions related to using, installing, and porting
Octave that are not adequately answered by the Octave manual or by this
document.

If you would like to join the discussion and receive all messages sent
to the list, please send a short note to

@example
@group
help-octave-request@@bevo.che.wisc.edu
            ^^^^^^^
@end group
@end example

@strong{Please do not} send requests to be added or removed from the the
mailing list, or other administrative trivia to the list itself.

An archive of old postings to the help-octave mailing list is maintained
on ftp.che.wisc.edu in the directory @file{/pub/octave/MAILING-LISTS}.

@node Bug reports, MATLAB compatibility, Getting additional help, Top
@chapter I think I have found a bug in Octave.

@cindex Bug in Octave, newly found

``I think I have found a bug in Octave, but I'm not sure.  How do I know,
and who should I tell?''

@cindex Manual, for Octave

First, see the section on bugs and bug reports in the Octave manual.
The Octave manual is included in the Octave distribution.

When you report a bug, make sure to describe the type of computer you
are using, the version of the operating system it is running, and the
version of Octave that you are using.  Also provide enough code so that
the Octave maintainers can duplicate your bug.

If you have Octave working at all, the easiest way to do this is to use
the Octave function @code{bug_report}.  When you execute this function,
Octave will prompt you for a subject and then invoke the editor on a
file that already contains all the configuration information.  When you
exit the editor, Octave will mail the bug report for you.

@cindex Octave bug report
@cindex Mailing lists, bug-octave

If for some reason you cannot use Octave's @code{bug_report} function,
mail your bug report to "bug-octave@@bevo.che.wisc.edu".  Your message
needs to include enough information to allow the maintainers of Octave
to fix the bug.  Please read the section on bugs and bug reports in the
Octave manual for a list of things that should be included in every bug
report.

@node MATLAB compatibility, Index, Bug reports, Top
@chapter Porting programs from @sc{Matlab} to Octave

@cindex @sc{Matlab} compatibility
@cindex Compatibility with @sc{Matlab}

``I wrote some code for @sc{Matlab}, and I want to get it running under
Octave.  Is there anything I should watch out for?''

The differences between Octave and @sc{Matlab} typically fall into one of
three categories:

@enumerate
@item
Irrelevant.

@item
Known differences, perhaps configurable with a user preference variable.

@item
Unknown differences.
@end enumerate

The first category, irrelevant differences, do not affect computations
and most likely do not affect the execution of function files.

The differences of the second category are usually because the authors
of Octave decided on a better (subjective) implementation that the way
@sc{Matlab} does it, and so introduced ``user preference variables'' so that
you can customize Octave's behavior to be either @sc{Matlab}-compatible or
to use Octave's new features.  To make Octave more @sc{Matlab}-compatible,
put the following statements in your @file{~/.octaverc} file, or use the
command line option @samp{--traditional}, which implies all of these
settings.  Note that this list may not be complete, because some new
variables may have been introduced since this document was last updated.

@example
@group
  PS1 = ">> "
  PS2 = ""
  beep_on_error = 1.0
  default_eval_print_flag = 0.0
  default_save_format = "mat-binary"
  define_all_return_values = 1.0
  do_fortran_indexing = 1.0
  empty_list_elements_ok = 1.0
  fixed_point_format = 1.0
  implicit_num_to_str_ok = 1.0
  implicit_str_to_num_ok = 1.0
  ok_to_lose_imaginary_part = 1.0
  page_screen_output = 0.0
  prefer_column_vectors = 0.0
  prefer_zero_one_indexing = 1.0
  print_empty_dimensions = 0.0
  treat_neg_dim_as_zero = 1.0
  warn_function_name_clash = 0.0
  whitespace_in_literal_matrix = "traditional"
@end group
@end example

Some other known differences are:

@itemize @bullet
@item
The Octave plotting functions are mostly compatible with the ones from
@sc{Matlab} 3.x, but not from @sc{Matlab} 4.x.
@end itemize

The third category of differences is (hopefully) shrinking.  If you find
a difference between Octave behavior and @sc{Matlab}, then you should send a
description of this difference (with code illustrating the difference,
if possible) to bug-octave@@bevo.che.wisc.edu.

An archive of old postings to the Octave mailing lists is maintained
on ftp.che.wisc.edu in the directory @file{/pub/octave/MAILING-LISTS}.

@node Index,  , MATLAB compatibility, Top
@appendix Concept Index

@printindex cp

@page
@contents
@bye
