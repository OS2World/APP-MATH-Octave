@c Copyright (C) 1996, 1997 John W. Eaton
@c Written by Kurt Hornik <Kurt.Hornik@ci.tuwien.ac.at> on 1996/05/14
@c This is part of the Octave manual.
@c For copying conditions, see the file gpl.tex.

@node Audio Processing, System Utilities, Image Processing, Top
@chapter Audio Processing

Octave provides a few functions for dealing with audio data.  An audio
`sample' is a single output value from an A/D converter, i.e., a small
integer number (usually 8 or 16 bits), and audio data is just a series
of such samples.  It can be characterized by three parameters:  the
sampling rate (measured in samples per second or Hz, e.g. 8000 or
44100), the number of bits per sample (e.g. 8 or 16), and the number of
channels (1 for mono, 2 for stereo, etc.).

There are many different formats for representing such data.  Currently,
only the two most popular, @emph{linear encoding} and @emph{mu-law
encoding}, are supported by Octave.  There is an excellent FAQ on audio
formats by Guido van Rossum <guido@@cwi.nl> which can be found at any
FAQ ftp site, in particular in the directory
@file{/pub/usenet/news.answers/audio-fmts} of the archive site
@code{rtfm.mit.edu}.

Octave simply treats audio data as vectors of samples (non-mono data are
not supported yet).  It is assumed that audio files using linear
encoding have one of the extensions @file{lin} or @file{raw}, and that
files holding data in mu-law encoding end in @file{au}, @file{mu}, or
@file{snd}.

@deftypefn {Function File} {} lin2mu (@var{x})
If the vector @var{x} represents mono audio data in 8- or 16-bit
linear encoding, @code{lin2mu (@var{x})} is the corresponding mu-law
encoding.
@end deftypefn

@deftypefn {Function File} {} mu2lin (@var{x}, @var{bps})
If the vector @var{x} represents mono audio data in mu-law encoding,
@code{mu2lin} converts it to linear encoding.  The optional argument
@var{bps} specifies whether the input data uses 8 bit per sample
(default) or 16 bit.
@end deftypefn

@deftypefn {Function File} {} loadaudio (@var{name}, @var{ext}, @var{bps})
Loads audio data from the file @file{@var{name}.@var{ext}} into the
vector @var{x}.  

The extension @var{ext} determines how the data in the audio file is
interpreted;  the extensions @file{lin} (default) and @file{raw}
correspond to linear, the extensions @file{au}, @file{mu}, or @file{snd}
to mu-law encoding.

The argument @var{bps} can be either 8 (default) or 16, and specifies
the number of bits per sample used in the audio file.
@end deftypefn

@deftypefn {Function File} {} saveaudio (@var{name}, @var{x}, @var{ext}, @var{bps})
Saves a vector @var{x} of audio data to the file
@file{@var{name}.@var{ext}}.  The optional parameters @var{ext} and
@var{bps} determine the encoding and the number of bits per sample used
in the audio file (see @code{loadaudio});  defaults are @file{lin} and
8, respectively.
@end deftypefn

The following functions for audio I/O require special A/D hardware and
operating system support.  It is assumed that audio data in linear
encoding can be played and recorded by reading from and writing to
@file{/dev/dsp}, and that similarly @file{/dev/audio} is used for mu-law
encoding.  These file names are system-dependent.  Improvements so that
these functions will work without modification on a wide variety of
hardware are welcome.

@deftypefn {Function File} {} playaudio (@var{name}, @var{ext})
@deftypefnx {Function File} {} playaudio (@var{x})
Plays the audio file @file{@var{name}.@var{ext}} or the audio data
stored in the vector @var{x}.
@end deftypefn

@deftypefn {Function File} {} record (@var{sec}, @var{sampling_rate})
Records @var{sec} seconds of audio input into the vector @var{x}.  The
default value for @var{sampling_rate} is 8000 samples per second, or
8kHz.  The program waits until the user types @key{RET} and then
immediately starts to record.
@end deftypefn

@deftypefn {Function File} {} setaudio (@var{type})
@deftypefnx {Function File} {} setaudio (@var{type}, @var{value})
Set or display various properties of your mixer hardware.

For example, if @code{vol} corresponds to the volume property, you can
set it to 50 (percent) by @code{setaudio ("vol", 50)}.

This is an simple experimental program to control the audio hardware
settings.  It assumes that there is a @code{mixer} program which can be
used as @code{mixer @var{type} @var{value}}, and simply executes
@code{system ("mixer @var{type} @var{value}")}.  Future releases might
get rid of this assumption by using the @code{fcntl} interface.
@end deftypefn
