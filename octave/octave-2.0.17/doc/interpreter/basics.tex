@c Copyright (C) 1996, 1997 John W. Eaton
@c This is part of the Octave manual.
@c For copying conditions, see the file gpl.tex.

@node Getting Started, Data Types, Introduction, Top
@chapter Getting Started

This chapter explains some of Octave's basic features, including how to
start an Octave session, get help at the command prompt, edit the
command line, and write Octave programs that can be executed as commands
from your shell.

@menu
* Invoking Octave::             
* Quitting Octave::             
* Getting Help::                
* Command Line Editing::        
* Errors::                      
* Executable Octave Programs::  
* Comments::                    
@end menu

@node Invoking Octave, Quitting Octave, Getting Started, Getting Started
@section Invoking Octave

Normally, Octave is used interactively by running the program
@samp{octave} without any arguments.  Once started, Octave reads
commands from the terminal until you tell it to exit.

You can also specify the name of a file on the command line, and Octave
will read and execute the commands from the named file and then exit
when it is finished.

You can further control how Octave starts by using the command-line
options described in the next section, and Octave itself can remind you
of the options available.  Type @samp{octave --help} to display all
available options and briefly describe their use (@samp{octave -h} is a
shorter equivalent).

@menu
* Command Line Options::        
* Startup Files::               
@end menu

@node Command Line Options, Startup Files, Invoking Octave, Invoking Octave
@subsection Command Line Options
@cindex Octave command options
@cindex command options
@cindex options, Octave command

Here is a complete list of all the command line options that Octave
accepts.

@table @code
@item --debug
@itemx -d
@cindex @code{--debug}
@cindex @code{-d}
Enter parser debugging mode.  Using this option will cause Octave's
parser to print a lot of information about the commands it reads, and is
probably only useful if you are actually trying to debug the parser.

@item --echo-commands
@itemx -x
@cindex @code{--echo-commands}
@cindex @code{-x}
Echo commands as they are executed.

@item --exec-path @var{path}
@cindex @code{--exec-path @var{path}}
Specify the path to search for programs to run.  The value of @var{path}
specified on the command line will override any value of
@code{OCTAVE_EXEC_PATH} found in the environment, but not any commands
in the system or user startup files that set the built-in variable
@code{EXEC_PATH}.

@item --help
@itemx -h
@itemx -?
@cindex @code{--help}
@cindex @code{-h}
@cindex @code{-?}
Print short help message and exit.

@item --info-file @var{filename}
@cindex @code{--info-file @var{filename}}
Specify the name of the info file to use.  The value of @var{filename}
specified on the command line will override any value of
@code{OCTAVE_INFO_FILE} found in the environment, but not any commands
in the system or user startup files that set the built-in variable
@code{INFO_FILE}.

@item --info-program @var{program}
@cindex @code{--info-program @var{program}}
Specify the name of the info program to use.  The value of @var{program}
specified on the command line will override any value of
@code{OCTAVE_INFO_PROGRAM} found in the environment, but not any
commands in the system or user startup files that set the built-in
variable @code{INFO_PROGRAM}.

@item --interactive
@itemx -i
@cindex @code{--interactive}
@cindex @code{-i}
Force interactive behavior.  This can be useful for running Octave via a
remote shell command or inside an Emacs shell buffer.  For another way
to run Octave within Emacs, see @ref{Emacs}.

@item --no-init-file
@cindex @code{--no-init-file}
Don't read the @file{~/.octaverc} or @file{.octaverc} files.

@item --no-line-editing
@cindex @code{--no-line-editing}
Disable command-line editing.

@item --no-site-file
@cindex @code{--no-site-file}
Don't read the site-wide @file{octaverc} file.

@item --norc
@itemx -f
@cindex @code{--norc}
@cindex @code{-f}
Don't read any of the system or user initialization files at startup.
This is equivalent to using both of the options @code{--no-init-file}
and @code{--no-site-file}.

@item --path @var{path}
@itemx -p @var{path}
@cindex @code{--path @var{path}}
@cindex @code{-p @var{path}}
Specify the path to search for function files.  The value of @var{path}
specified on the command line will override any value of
@code{OCTAVE_PATH} found in the environment, but not any commands in the
system or user startup files that set the built-in variable @code{LOADPATH}.

@item --silent
@itemx --quiet
@itemx -q
@cindex @code{--silent}
@cindex @code{--quiet}
@cindex @code{-q}
Don't print the usual greeting and version message at startup.

@item --traditional
@itemx --braindead
@cindex @code{--traditional}
@cindex @code{--braindead}
Set initial values for user-preference variables to the following
values for compatibility with @sc{Matlab}.

@example
PS1                           = ">> "
PS2                           = ""
beep_on_error                 = 1
default_save_format           = "mat-binary"
define_all_return_values      = 1
do_fortran_indexing           = 1
crash_dumps_octave_core       = 0
empty_list_elements_ok        = 1
implicit_str_to_num_ok        = 1
ok_to_lose_imaginary_part     = 1
page_screen_output            = 0
prefer_column_vectors         = 0
print_empty_dimensions        = 0
treat_neg_dim_as_zero         = 1
warn_function_name_clash      = 0
whitespace_in_literal_matrix  = "traditional"
@end example

@item --verbose
@itemx -V
@cindex @code{--verbose}
@cindex @code{-V}
Turn on verbose output.

@item --version
@itemx -v
@cindex @code{--version}
@cindex @code{-v}
Print the program version number and exit.

@item @var{file}
Execute commands from @var{file}.
@end table

Octave also includes several built-in variables that contain information
about the command line, including the number of arguments and all of the
options.

@defvr {Built-in Variable} argv
The command line arguments passed to Octave are available in this
variable.  For example, if you invoked Octave using the command

@example
octave --no-line-editing --silent
@end example

@noindent
@code{argv} would be a string vector with the elements
@code{--no-line-editing} and @code{--silent}.

If you write an executable Octave script, @code{argv} will contain the
list of arguments passed to the script.  @pxref{Executable Octave Programs}.
@end defvr

@defvr {Built-in Variable} program_invocation_name
@defvrx {Built-in Variable} program_name
When Octave starts, the value of the built-in variable
@code{program_invocation_name} is automatically set to the name that was
typed at the shell prompt to run Octave, and the value of
@code{program_name} is automatically set to the final component of
@code{program_invocation_name}.  For example, if you typed
@samp{@value{OCTAVEHOME}/bin/octave} to start Octave,
@code{program_invocation_name} would have the value
@code{"@value{OCTAVEHOME}/bin/octave"}, and @code{program_name} would
have the value @code{"octave"}.

If executing a script from the command line (e.g., @code{octave foo.m})
or using an executable Octave script, the program name is set to the
name of the script.  @xref{Executable Octave Programs} for an example of
how to create an executable Octave script.
@end defvr

Here is an example of using these variables to reproduce Octave's
command line.

@example
printf ("%s", program_name);
for i = 1:nargin
  printf (" %s", argv(i,:));
endfor
printf ("\n");
@end example

@noindent
@xref{Index Expressions} for an explanation of how to properly index
arrays of strings and substrings in Octave, and @xref{Defining Functions} 
for information about the variable @code{nargin}.

@node Startup Files,  , Command Line Options, Invoking Octave
@subsection Startup Files
@cindex initialization
@cindex startup

When Octave starts, it looks for commands to execute from the following
files:

@cindex startup files

@table @code
@item @var{octave-home}/share/octave/site/m/startup/octaverc
Where @var{octave-home} is the directory in which all of Octave is
installed (the default is @file{@value{OCTAVEHOME}}).  This file is
provided so that changes to the default Octave environment can be made
globally for all users at your site for all versions of Octave you have
installed.  Some care should be taken when making changes to this file,
since all users of Octave at your site will be affected.

@item @var{octave-home}/share/octave/@var{version}/m/startup/octaverc
Where @var{octave-home} is the directory in which all of Octave is
installed (the default is @file{@value{OCTAVEHOME}}), and @var{version}
is the version number of Octave.  This file is provided so that changes
to the default Octave environment can be made globally for all users for
a particular version of Octave.  Some care should be taken when making
changes to this file, since all users of Octave at your site will be
affected.

@item ~/.octaverc
@cindex @code{~/.octaverc}
This file is normally used to make personal changes to the default
Octave environment.

@item .octaverc
@cindex @code{.octaverc}
This file can be used to make changes to the default Octave environment
for a particular project.  Octave searches for this file in the current
directory after it reads @file{~/.octaverc}.  Any use of the @code{cd}
command in the @file{~/.octaverc} file will affect the directory that
Octave searches for the file @file{.octaverc}.

If you start Octave in your home directory, commands from from the file
@file{~/.octaverc} will only be executed once.
@end table

A message will be displayed as each of the startup files is read if you
invoke Octave with the @code{--verbose} option but without the
@code{--silent} option.

Startup files may contain any valid Octave commands, including function
definitions.

@node Quitting Octave, Getting Help, Invoking Octave, Getting Started
@section Quitting Octave
@cindex exiting octave
@cindex quitting octave

@deftypefn {Built-in Function} {} exit (@var{status})
@deftypefnx {Built-in Function} {} quit (@var{status})
Exit the current Octave session.  If the optional integer value
@var{status} is supplied, pass that value to the operating system as the
Octave's exit status.
@end deftypefn

@deftypefn {Built-in Function} {} atexit (@var{fcn})
Register function to be called when Octave exits.  For example,

@example
@group
function print_flops_at_exit ()
  printf ("\n%s\n", system ("fortune"));
  fflush (stdout);
endfunction
atexit ("print_flops_at_exit");
@end group
@end example

@noindent
will print a message when Octave exits.
@end deftypefn

@node Getting Help, Command Line Editing, Quitting Octave, Getting Started
@section Commands for Getting Help
@cindex on-line help
@cindex help, on-line

The entire text of this manual is available from the Octave prompt
via the command @kbd{help -i}.  In addition, the documentation for
individual user-written functions and variables is also available via
the @kbd{help} command.  This section describes the commands used for
reading the manual and the documentation strings for user-supplied
functions and variables.  @xref{Function Files}, for more information
about how to document the functions you write.

@deffn {Command} help
Octave's @code{help} command can be used to print brief usage-style
messages, or to display information directly from an on-line version of
the printed manual, using the GNU Info browser.  If invoked without any
arguments, @code{help} prints a list of all the available operators,
functions, and built-in variables.  If the first argument is @code{-i},
the @code{help} command searches the index of the on-line version of
this manual for the given topics.

For example, the command @kbd{help help} prints a short message
describing the @code{help} command, and @kbd{help -i help} starts the
GNU Info browser at this node in the on-line version of the manual.

Once the GNU Info browser is running, help for using it is available
using the command @kbd{C-h}.
@end deffn

The help command can give you information about operators, but not the
comma and semicolons that are used as command separators.  To get help
for those, you must type @kbd{help comma} or @kbd{help semicolon}.

@defvr {Built-in Variable} INFO_FILE
The variable @code{INFO_FILE} names the location of the Octave info file.
The default value is @code{"@var{octave-home}/info/octave.info"}, where
@var{octave-home} is the directory where all of Octave is installed.
@end defvr

@defvr {Built-in Variable} INFO_PROGRAM
The variable @code{INFO_PROGRAM} names the info program to run.  Its
initial value is
@code{"@var{octave-home}/libexec/octave/@var{version}/exec/@var{arch}/info"},
where @var{octave-home} is the directory where all of Octave is
installed, @var{version} is the Octave version number, and @var{arch} is
the machine type.  The value of @code{INFO_PROGRAM} can be overridden by
the environment variable @code{OCTAVE_INFO_PROGRAM}, or the command line
argument @code{--info-program NAME}, or by setting the value of the
built-in variable @code{INFO_PROGRAM} in a startup script.
@end defvr

@defvr {Built-in Variable} suppress_verbose_help_message
If the value of @code{suppress_verbose_help_message} is nonzero, Octave
will not add additional help information to the end of the output from
the @code{help} command and usage messages for built-in commands.
@end defvr

@node Command Line Editing, Errors, Getting Help, Getting Started
@section Command Line Editing
@cindex command-line editing
@cindex editing the command line

Octave uses the GNU readline library to provide an extensive set of
command-line editing and history features.  Only the most common
features are described in this manual.  Please see The GNU Readline
Library manual for more information.

To insert printing characters (letters, digits, symbols, etc.), simply
type the character.  Octave will insert the character at the cursor and
advance the cursor forward.

Many of the command-line editing functions operate using control
characters.  For example, the character @kbd{Control-a} moves the cursor
to the beginning of the line.  To type @kbd{C-a}, hold down @key{CTRL}
and then press @key{a}.  In the following sections, control characters
such as @kbd{Control-a} are written as @kbd{C-a}.

Another set of command-line editing functions use Meta characters.  On
some terminals, you type @kbd{M-u} by holding down @key{META} and
pressing @key{u}.  If your terminal does not have a @key{META} key, you
can still type Meta charcters using two-character sequences starting
with @kbd{ESC}.  Thus, to enter @kbd{M-u}, you could type
@key{ESC}@key{u}.  The @kbd{ESC} character sequences are also allowed on
terminals with real Meta keys.  In the following sections, Meta
characters such as @kbd{Meta-u} are written as @kbd{M-u}.

@menu
* Cursor Motion::               
* Killing and Yanking::         
* Commands For Text::           
* Commands For Completion::     
* Commands For History::        
* Customizing the Prompt::      
* Diary and Echo Commands::     
@end menu

@node Cursor Motion, Killing and Yanking, Command Line Editing, Command Line Editing
@subsection Cursor Motion

The following commands allow you to position the cursor.

@table @kbd
@item C-b
Move back one character.

@item C-f
Move forward one character.

@item @key{DEL}
Delete the character to the left of the cursor.

@item C-d
Delete the character underneath the cursor.

@item M-f
Move forward a word.

@item M-b
Move backward a word.

@item C-a
Move to the start of the line.

@item C-e
Move to the end of the line.

@item C-l
Clear the screen, reprinting the current line at the top.

@item C-_
@itemx C-/
Undo the last thing that you did.  You can undo all the way back to an
empty line.

@item M-r
Undo all changes made to this line.  This is like typing the `undo'
command enough times to get back to the beginning.
@end table

The above table describes the most basic possible keystrokes that you need
in order to do editing of the input line.  On most terminals, you can
also use the arrow keys in place of @kbd{C-f} and @kbd{C-b} to move
forward and backward.

Notice how @kbd{C-f} moves forward a character, while @kbd{M-f} moves
forward a word.  It is a loose convention that control keystrokes
operate on characters while meta keystrokes operate on words.

There is also a function available so that you can clear the screen from
within Octave programs.

@cindex clearing the screen

@deftypefn {Built-in Function} {} clc ()
@deftypefnx {Built-in Function} {} home ()
Clear the terminal screen and move the cursor to the upper left corner.
@end deftypefn

@node Killing and Yanking, Commands For Text, Cursor Motion, Command Line Editing
@subsection Killing and Yanking

@dfn{Killing} text means to delete the text from the line, but to save
it away for later use, usually by @dfn{yanking} it back into the line.
If the description for a command says that it `kills' text, then you can
be sure that you can get the text back in a different (or the same)
place later.

Here is the list of commands for killing text.

@table @kbd
@item C-k
Kill the text from the current cursor position to the end of the line.

@item M-d
Kill from the cursor to the end of the current word, or if between
words, to the end of the next word.

@item M-@key{DEL}
Kill from the cursor to the start of the previous word, or if between
words, to the start of the previous word. 

@item C-w
Kill from the cursor to the previous whitespace.  This is different than
@kbd{M-@key{DEL}} because the word boundaries differ.
@end table

And, here is how to @dfn{yank} the text back into the line.  Yanking
means to copy the most-recently-killed text from the kill buffer.

@table @kbd
@item C-y
Yank the most recently killed text back into the buffer at the cursor.

@item M-y
Rotate the kill-ring, and yank the new top.  You can only do this if
the prior command is @kbd{C-y} or @kbd{M-y}.
@end table

When you use a kill command, the text is saved in a @dfn{kill-ring}.
Any number of consecutive kills save all of the killed text together, so
that when you yank it back, you get it in one clean sweep.  The kill
ring is not line specific; the text that you killed on a previously
typed line is available to be yanked back later, when you are typing
another line.

@node Commands For Text, Commands For Completion, Killing and Yanking, Command Line Editing
@subsection Commands For Changing Text

The following commands can be used for entering characters that would
otherwise have a special meaning (e.g., @kbd{TAB}, @kbd{C-q}, etc.), or
for quickly correcting typing mistakes.

@table @kbd
@item C-q
@itemx C-v
Add the next character that you type to the line verbatim.  This is
how to insert things like @kbd{C-q} for example.

@item M-@key{TAB}
Insert a tab character.

@item C-t
Drag the character before the cursor forward over the character at the
cursor, also moving the cursor forward.  If the cursor is at the end of
the line, then transpose the two characters before it.

@item M-t
Drag the word behind the cursor past the word in front of the cursor
moving the cursor over that word as well.

@item M-u
Uppercase the characters following the cursor to the end of the current
(or following) word, moving the cursor to the end of the word.

@item M-l
Lowecase the characters following the cursor to the end of the current
(or following) word, moving the cursor to the end of the word.

@item M-c
Uppercase the character following the cursor (or the beginning of the
next word if the cursor is between words), moving the cursor to the end
of the word.
@end table

@node Commands For Completion, Commands For History, Commands For Text, Command Line Editing
@subsection Letting Readline Type For You
@cindex command completion

The following commands allow Octave to complete command and variable
names for you.

@table @kbd
@item @key{TAB}
Attempt to do completion on the text before the cursor.  Octave can
complete the names of commands and variables.

@item M-?
List the possible completions of the text before the cursor.
@end table

@defvr {Built-in Variable} completion_append_char
The value of @code{completion_append_char} is used as the character to
append to successful command-line completion attempts.  The default
value is @code{" "} (a single space).
@end defvr

@deftypefn {Built-in Function} {} completion_matches (@var{hint})
Generate possible completions given @var{hint}.

This function is provided for the benefit of programs like Emacs which
might be controlling Octave and handling user input.  The current
command number is not incremented when this function is called.  This is
a feature, not a bug.
@end deftypefn

@node Commands For History, Customizing the Prompt, Commands For Completion, Command Line Editing
@subsection Commands For Manipulating The History
@cindex command history
@cindex input history
@cindex history of commands

Octave normally keeps track of the commands you type so that you can
recall previous commands to edit or execute them again.  When you exit
Octave, the most recent commands you have typed, up to the number
specified by the variable @code{history_size}, are saved in a file.
When Octave starts, it loads an initial list of commands from the file
named by the variable @code{history_file}.

Here are the commands for simple browsing and searching the history
list.

@table @kbd
@item @key{LFD}
@itemx @key{RET}
Accept the line regardless of where the cursor is.  If this line is
non-empty, add it to the history list.  If this line was a history
line, then restore the history line to its original state.

@item C-p
Move `up' through the history list.

@item C-n
Move `down' through the history list.

@item M-<
Move to the first line in the history.

@item M->
Move to the end of the input history, i.e., the line you are entering!

@item C-r
Search backward starting at the current line and moving `up' through
the history as necessary.  This is an incremental search.

@item C-s
Search forward starting at the current line and moving `down' through
the history as necessary.
@end table

On most terminals, you can also use the arrow keys in place of @kbd{C-p}
and @kbd{C-n} to move through the history list.

In addition to the keyboard commands for moving through the history
list, Octave provides three functions for viewing, editing, and
re-running chunks of commands from the history list.

@deffn {Command} history options
If invoked with no arguments, @code{history} displays a list of commands
that you have executed.  Valid options are:

@table @code
@item -w @var{file}
Write the current history to the file @var{file}.  If the name is
omitted, use the default history file (normally @file{~/.octave_hist}).

@item -r @var{file}
Read the file @var{file}, replacing the current history list with its
contents.  If the name is omitted, use the default history file
(normally @file{~/.octave_hist}).

@item @var{N}
Only display the most recent @var{N} lines of history.

@item -q
Don't number the displayed lines of history.  This is useful for cutting
and pasting commands if you are using the X Window System.
@end table

For example, to display the five most recent commands that you have
typed without displaying line numbers, use the command
@kbd{history -q 5}.
@end deffn

@deffn {Command} edit_history options
If invoked with no arguments, @code{edit_history} allows you to edit the
history list using the editor named by the variable @code{EDITOR}.  The
commands to be edited are first copied to a temporary file.  When you
exit the editor, Octave executes the commands that remain in the file.
It is often more convenient to use @code{edit_history} to define functions 
rather than attempting to enter them directly on the command line.
By default, the block of commands is executed as soon as you exit the
editor.  To avoid executing any commands, simply delete all the lines
from the buffer before exiting the editor.

The @code{edit_history} command takes two optional arguments specifying
the history numbers of first and last commands to edit.  For example,
the command

@example
edit_history 13
@end example

@noindent
extracts all the commands from the 13th through the last in the history
list.  The command

@example
edit_history 13 169
@end example

@noindent
only extracts commands 13 through 169.  Specifying a larger number for
the first command than the last command reverses the list of commands
before placing them in the buffer to be edited.  If both arguments are
omitted, the previous command in the history list is used.
@end deffn

@deffn {Command} run_history
Similar to @code{edit_history}, except that the editor is not invoked,
and the commands are simply executed as they appear in the history list.
@end deffn

@defvr {Built-in Variable} EDITOR
A string naming the editor to use with the @code{edit_history} command.
If the environment variable @code{EDITOR} is set when Octave starts, its
value is used as the default.  Otherwise, @code{EDITOR} is set to
@code{"emacs"}.
@end defvr

@defvr {Built-in Variable} history_file
This variable specifies the name of the file used to store command
history.  The default value is @code{"~/.octave_hist"}, but may be
overridden by the environment variable @code{OCTAVE_HISTFILE}.
@end defvr

@defvr {Built-in Variable} history_size
This variable specifies how many entries to store in the history file.
The default value is @code{1024}, but may be overridden by the
environment variable @code{OCTAVE_HISTSIZE}.
@end defvr

@defvr {Built-in Variable} saving_history
If the value of @code{saving_history} is nonzero, command entered
on the command line are saved in the file specified by the variable
@code{history_file}.
@end defvr

@node Customizing the Prompt, Diary and Echo Commands, Commands For History, Command Line Editing
@subsection Customizing the Prompt
@cindex prompt customization
@cindex customizing the prompt

The following variables are available for customizing the appearance of
the command-line prompts.  Octave allows the prompt to be customized by
inserting a number of backslash-escaped special characters that are
decoded as follows:

@table @samp
@item \t
The time.

@item \d
The date.

@item \n
Begins a new line by printing the equivalent of a carriage return
followed by a line feed.

@item \s
The name of the program (usually just @samp{octave}).

@item \w
The current working directory.

@item \W
The basename of the current working directory.

@item \u
The username of the current user.

@item \h
The hostname, up to the first `.'.

@item \H
The hostname.

@item \#
The command number of this command, counting from when Octave starts.

@item \!
The history number of this command.  This differs from @samp{\#} by the
number of commands in the history list when Octave starts.

@item \$
If the effective UID is 0, a @samp{#}, otherwise a @samp{$}.

@item \nnn
The character whose character code in octal is @var{nnn}.

@item \\
A backslash.
@end table

@defvr {Built-in Variable} PS1
The primary prompt string.  When executing interactively, Octave
displays the primary prompt @code{PS1} when it is ready to read a
command.

The default value of @code{PS1} is @code{"\s:\#> "}.  To change it, use a
command like

@example
octave:13> PS1 = "\\u@@\\H> "
@end example

@noindent
which will result in the prompt @samp{boris@@kremvax> } for the user
@samp{boris} logged in on the host @samp{kremvax.kgb.su}.  Note that two
backslashes are required to enter a backslash into a string.
@xref{Strings}.
@end defvr

@defvr {Built-in Variable} PS2
The secondary prompt string, which is printed when Octave is
expecting additional input to complete a command.  For example, when
defining a function over several lines, Octave will print the value of
@code{PS1} at the beginning of each line after the first.  The default
value of @code{PS2} is @code{"> "}.
@end defvr

@defvr {Built-in Variable} PS4
If Octave is invoked with the @code{--echo-input} option, the value of
@code{PS4} is printed before each line of input that is echoed.  The
default value of @code{PS4} is @code{"+ "}.  @xref{Invoking Octave}, for
a description of @code{--echo-input}.
@end defvr

@node Diary and Echo Commands,  , Customizing the Prompt, Command Line Editing
@subsection Diary and Echo Commands
@cindex diary of commands and output
@cindex command and ouput logs
@cindex logging commands and output
@cindex echoing executing commands
@cindex command echoing

Octave's diary feature allows you to keep a log of all or part of an
interactive session by recording the input you type and the output that
Octave produces in a separate file.

@deffn {Command} diary options
Create a list of all commands @emph{and} the output they produce, mixed
together just as you see them on your terminal.  Valid options are:

@table @code
@item on
Start recording your session in a file called @file{diary} in your
current working directory.

@item off
Stop recording your session in the diary file.

@item @var{file}
Record your session in the file named @var{file}.
@end table

Without any arguments, @code{diary} toggles the current diary state.
@end deffn

Sometimes it is useful to see the commands in a function or script as
they are being evaluated.  This can be especially helpful for debugging
some kinds of problems.

@deffn {Command} echo options
Control whether commands are displayed as they are executed.  Valid
options are:

@table @code
@item on
Enable echoing of commands as they are executed in script files.

@item off
Disable echoing of commands as they are executed in script files.

@item on all
Enable echoing of commands as they are executed in script files and
functions.

@item off all
Disable echoing of commands as they are executed in script files and
functions.
@end table

@noindent
If invoked without any arguments, @code{echo} toggles the current echo
state.
@end deffn

@defvr {Built-in Variable} echo_executing_commands
This variable may also be used to control the echo state.  It may be
the sum of the following values:

@table @asis
@item 1
Echo commands read from script files.

@item 2
Echo commands from functions.

@item 4
Echo commands read from command line.
@end table

More than one state can be active at once.  For example, a value of 3 is
equivalent to the command @kbd{echo on all}.

The value of @code{echo_executing_commands} is set by the @kbd{echo}
command and the command line option @code{--echo-input}.
@end defvr

@node Errors, Executable Octave Programs, Command Line Editing, Getting Started
@section How Octave Reports Errors
@cindex error messages
@cindex messages, error

Octave reports two kinds of errors for invalid programs.

A @dfn{parse error} occurs if Octave cannot understand something you
have typed.  For example, if you misspell a keyword,

@example
octave:13> functon y = f (x) y = x^2; endfunction
@end example

@noindent
Octave will respond immediately with a message like this:

@example
parse error:

  functon y = f (x) y = x^2; endfunction
          ^
@end example

@noindent
For most parse errors, Octave uses a caret (@samp{^}) to mark the point
on the line where it was unable to make sense of your input.  In this
case, Octave generated an error message because the keyword
@code{function} was misspelled.  Instead of seeing @samp{function f},
Octave saw two consecutive variable names, which is invalid in this
context.  It marked the error at @code{y} because the first name by
itself was accepted as valid input.

Another class of error message occurs at evaluation time.  These
errors are called @dfn{run-time errors}, or sometimes
@dfn{evaluation errors} because they occur when your program is being
@dfn{run}, or @dfn{evaluated}.  For example, if after correcting the
mistake in the previous function definition, you type

@example
octave:13> f ()
@end example

@noindent
Octave will respond with

@example
@group
error: `x' undefined near line 1 column 24
error: evaluating expression near line 1, column 24
error: evaluating assignment expression near line 1, column 22
error: called from `f'
@end group
@end example

This error message has several parts, and gives you quite a bit of
information to help you locate the source of the error.  The messages
are generated from the point of the innermost error, and provide a
traceback of enclosing expressions and function calls.

In the example above, the first line indicates that a variable named
@samp{x} was found to be undefined near line 1 and column 24 of some
function or expression.  For errors occurring within functions, lines
are counted from the beginning of the file containing the function
definition.  For errors occurring at the top level, the line number
indicates the input line number, which is usually displayed in the
prompt string.

The second and third lines in the example indicate that the error
occurred within an assignment expression, and the last line of the error
message indicates that the error occurred within the function @code{f}.
If the function @code{f} had been called from another function, for
example, @code{g}, the list of errors would have ended with one more
line:

@example
error: called from `g'
@end example

These lists of function calls usually make it fairly easy to trace the
path your program took before the error occurred, and to correct the
error before trying again.

@node Executable Octave Programs, Comments, Errors, Getting Started
@section Executable Octave Programs
@cindex executable scripts
@cindex scripts
@cindex self contained programs
@cindex program, self contained
@cindex @samp{#!}

Once you have learned Octave, you may want to write self-contained
Octave scripts, using the @samp{#!} script mechanism.  You can do this
on GNU systems and on many Unix systems @footnote{The @samp{#!}
mechanism works on Unix systems derived from Berkeley Unix, System V
Release 4, and some System V Release 3 systems.}

For example, you could create a text file named @file{hello}, containing
the following lines:

@example
@group
#! @var{octave-interpreter-name} -qf
# a sample Octave program
printf ("Hello, world!\n");
@end group
@end example

@noindent
(where @var{octave-interpreter-name} should be replaced with the full
file name for your Octave binary).  After making this file executable
(with the @code{chmod} command), you can simply type:

@example
hello
@end example

@noindent
at the shell, and the system will arrange to run Octave as if you had
typed:

@example
octave hello
@end example

The line beginning with @samp{#!} lists the full file name of an
interpreter to be run, and an optional initial command line argument to
pass to that interpreter.  The operating system then runs the
interpreter with the given argument and the full argument list of the
executed program.  The first argument in the list is the full file name
of the Octave program. The rest of the argument list will either be
options to Octave, or data files, or both.  The @samp{-qf} option is
usually specified in stand-alone Octave programs to prevent them from
printing the normal startup message, and to keep them from behaving
differently depending on the contents of a particular user's
@file{~/.octaverc} file.  @xref{Invoking Octave}.  Note that some
operating systems may place a limit on the number of characters that are
recognized after @samp{#!}.

Self-contained Octave scripts are useful when you want to write a
program which users can invoke without knowing that the program is
written in the Octave language.

If you invoke an executable Octave script with command line arguments,
the arguments are available in the built-in variable @code{argv}.
@xref{Command Line Options}.  For example, the following program will
reproduce the command line that is used to execute it.

@example
@group
#! /bin/octave -qf
printf ("%s", program_name);
for i = 1:nargin
  printf (" %s", argv(i,:));
endfor
printf ("\n");
@end group
@end example

@node Comments,  , Executable Octave Programs, Getting Started
@section Comments in Octave Programs
@cindex @samp{#}
@cindex @samp{%}
@cindex comments
@cindex use of comments
@cindex documenting Octave programs
@cindex programs

A @dfn{comment} is some text that is included in a program for the sake
of human readers, and that is not really part of the program.  Comments
can explain what the program does, and how it works.  Nearly all
programming languages have provisions for comments, because programs are
typically hard to understand without them.

In the Octave language, a comment starts with either the sharp sign
character, @samp{#}, or the percent symbol @samp{%} and continues to the
end of the line.  The Octave interpreter ignores the rest of a
line following a sharp sign or percent symbol.  For example, we could
have put the following into the function @code{f}:

@example
@group
function xdot = f (x, t)

# usage: f (x, t)
#
# This function defines the right hand
# side functions for a set of nonlinear
# differential equations.

  r = 0.25;
  @dots{}
endfunction
@end group
@end example

The @code{help} command (@pxref{Getting Help}) is able to find the first
block of comments in a function (even those that are composed directly
on the command line).  This means that users of Octave can use the same
commands to get help for built-in functions, and for functions that you
have defined.  For example, after defining the function @code{f} above,
the command @kbd{help f} produces the output

@example
@group
 usage: f (x, t)

 This function defines the right hand
 side functions for a set of nonlinear
 differential equations.
@end group
@end example

Although it is possible to put comment lines into keyboard-composed
throw-away Octave programs, it usually isn't very useful, because the
purpose of a comment is to help you or another person understand the
program at a later time.

