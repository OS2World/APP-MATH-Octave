@c Copyright (C) 1996, 1997 John W. Eaton
@c This is part of the Octave manual.
@c For copying conditions, see the file gpl.tex.

@node Nonlinear Equations, Quadrature, Linear Algebra, Top
@chapter Nonlinear Equations
@cindex nonlinear equations
@cindex equations, nonlinear

Octave can solve sets of nonlinear equations of the form
@iftex
@tex
$$
 f (x) = 0
$$
@end tex
@end iftex
@ifinfo

@example
F (x) = 0
@end example
@end ifinfo

@noindent
using the function @code{fsolve}, which is based on the @sc{Minpack}
subroutine @code{hybrd}.

@deftypefn {Loadable Function} {[@var{x}, @var{info}] =} fsolve (@var{fcn}, @var{x0})
Given @var{fcn}, the name of a function of the form @code{f (@var{x})}
and an initial starting point @var{x0}, @code{fsolve} solves the set of
equations such that @code{f(@var{x}) == 0}.
@end deftypefn

@deftypefn {Loadable Function} {} fsolve_options (@var{opt}, @var{val})
When called with two arguments, this function allows you set options
parameters for the function @code{fsolve}.  Given one argument,
@code{fsolve_options} returns the value of the corresponding option.  If
no arguments are supplied, the names of all the available options and
their current values are displayed.
@end deftypefn

Here is a complete example.  To solve the set of equations
@iftex
@tex
$$
 \eqalign{-2x^2 + 3xy + 4\sin(y) - 6 &= 0\cr
           3x^2 - 2xy^2 + 3\cos(x) + 4 &= 0}
$$
@end tex
@end iftex
@ifinfo

@example
-2x^2 + 3xy   + 4 sin(y) = 6
 3x^2 - 2xy^2 + 3 cos(x) = -4
@end example
@end ifinfo

@noindent
you first need to write a function to compute the value of the given
function.  For example:

@example
function y = f (x)
  y(1) = -2*x(1)^2 + 3*x(1)*x(2)   + 4*sin(x(2)) - 6;
  y(2) =  3*x(1)^2 - 2*x(1)*x(2)^2 + 3*cos(x(1)) + 4;
endfunction
@end example

Then, call @code{fsolve} with a specified initial condition to find the
roots of the system of equations.  For example, given the function
@code{f} defined above,

@example
[x, info] = fsolve ("f", [1; 2])
@end example

@noindent
results in the solution

@example
x =

  0.57983
  2.54621

info = 1
@end example

A value of @code{info = 1} indicates that the solution has converged.

The function @code{perror} may be used to print English messages
corresponding to the numeric error codes.  For example,

@example
@group
perror ("fsolve", 1)
     @print{} solution converged to requested tolerance
@end group
@end example


