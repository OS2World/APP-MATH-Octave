@c Copyright (C) 1996, 1997 John W. Eaton
@c This is part of the Octave manual.
@c For copying conditions, see the file gpl.tex.

@node System Utilities, Tips, Audio Processing, Top
@chapter System Utilities

This chapter describes the functions that are available to allow you to
get information about what is happening outside of Octave, while it is
still running, and use this information in your program.  For example,
you can get information about environment variables, the current time,
and even start other programs from the Octave prompt.

@menu
* Timing Utilities::            
* Filesystem Utilities::        
* Controlling Subprocesses::    
* Process ID Information::      
* Environment Variables::       
* Current Working Directory::   
* Password Database Functions::  
* Group Database Functions::    
* System Information::          
@end menu

@node Timing Utilities, Filesystem Utilities, System Utilities, System Utilities
@section Timing Utilities

Octave's core set of functions for manipulating time values are
patterned after the corresponding functions from the standard C library.
Several of these functions use a data structure for time that includes
the following elements:

@table @code
@item usec
Microseconds after the second (0-999999).

@item sec
Seconds after the minute (0-61).  This number can be 61 to account
for leap seconds.

@item min
Minutes after the hour (0-59).

@item hour
Hours since midnight (0-23).

@item mday
Day of the month (1-31).

@item mon
Months since January (0-11).

@item year
Years since 1900.

@item wday
Days since Sunday (0-6).

@item yday
Days since January 1 (0-365).

@item isdst
Daylight Savings Time flag.

@item zone
Time zone.
@end table

@noindent
In the descriptions of the following functions, this structure is
referred to as a @var{tm_struct}.

@deftypefn {Loadable Function} {} time ()
Return the current time as the number of seconds since the epoch.  The
epoch is referenced to 00:00:00 CUT (Coordinated Universal Time) 1 Jan
1970.  For example, on Monday February 17, 1997 at 07:15:06 CUT, the
value returned by @code{time} was 856163706.
@end deftypefn

@deftypefn {Function File} {} ctime (@var{t})
Convert a value returned from @code{time} (or any other nonnegative
integer), to the local time and return a string of the same form as
@code{asctime}.  The function @code{ctime (time)} is equivalent to
@code{asctime (localtime (time))}.  For example,

@example
@group
ctime (time ())
     @result{} "Mon Feb 17 01:15:06 1997"
@end group
@end example
@end deftypefn

@deftypefn {Loadable Function} {} gmtime (@var{t})
Given a value returned from time (or any nonnegative integer),
return a time structure corresponding to CUT.  For example,

@example
@group
gmtime (time ())
     @result{} @{
           usec = 0
           year = 97
           mon = 1
           mday = 17
           sec = 6
           zone = CST
           min = 15
           wday = 1
           hour = 7
           isdst = 0
           yday = 47
         @}
@end group
@end example
@end deftypefn

@deftypefn {Loadable Function} {} localtime (@var{t})
Given a value returned from time (or any nonnegative integer),
return a time structure corresponding to the local time zone.

@example
@group
localtime (time ())
     @result{} @{
           usec = 0
           year = 97
           mon = 1
           mday = 17
           sec = 6
           zone = CST
           min = 15
           wday = 1
           hour = 1
           isdst = 0
           yday = 47
         @}
@end group
@end example
@end deftypefn

@deftypefn {Loadable Function} {} mktime (@var{tm_struct})
Convert a time structure corresponding to the local time to the number
of seconds since the epoch.  For example,

@example
@group
mktime (localtime (time ())
     @result{} 856163706
@end group
@end example
@end deftypefn

@deftypefn {Function File} {} asctime (@var{tm_struct})
Convert a time structure to a string using the following five-field
format: Thu Mar 28 08:40:14 1996.  For example,

@example
@group
asctime (localtime (time ())
     @result{} "Mon Feb 17 01:15:06 1997\n"
@end group
@end example

This is equivalent to @code{ctime (time ())}.
@end deftypefn

@deftypefn {Loadable Function} {} strftime (@var{tm_struct})
Format a time structure in a flexible way using @samp
with one of the modifiers described below.  Unknown field specifiers are
copied as normal characters.  All other characters are copied to the
output without change.  For example,

@example
@group
strftime ("%r (%Z) %A %e %B %Y", localtime (time ())
     @result{} "01:15:06 AM (CST) Monday 17 February 1997"
@end group
@end example

Octave's @code{strftime} function supports a superset of the ANSI C
field specifiers.

@noindent
Literal character fields:

@table @code
@item %
% character.

@item n
Newline character.

@item t
Tab character.
@end table

@noindent
Numeric modifiers (a nonstandard extension):

@table @code
@item - (dash)
Do not pad the field.

@item _ (underscore)
Pad the field with spaces.
@end table

@noindent
Time fields:

@table @code
@item %H
Hour (00-23).

@item %I
Hour (01-12).

@item %k
Hour (0-23).

@item %l
Hour (1-12).

@item %M
Minute (00-59).

@item %p
Locale's AM or PM.

@item %r
Time, 12-hour (hh:mm:ss [AP]M).

@item %R
Time, 24-hour (hh:mm).

@item %s
Time in seconds since 00:00:00, Jan 1, 1970 (a nonstandard extension).

@item %S
Second (00-61).

@item %T
Time, 24-hour (hh:mm:ss).

@item %X
Locale's time representation (%H:%M:%S).

@item %Z
Time zone (EDT), or nothing if no time zone is determinable.
@end table

@noindent
Date fields:

@table @code
@item %a
Locale's abbreviated weekday name (Sun-Sat).

@item %A
Locale's full weekday name, variable length (Sunday-Saturday).

@item %b
Locale's abbreviated month name (Jan-Dec).

@item %B
Locale's full month name, variable length (January-December).

@item %c
Locale's date and time (Sat Nov 04 12:02:33 EST 1989).

@item %C
Century (00-99).

@item %d
Day of month (01-31).

@item %e
Day of month ( 1-31).

@item %D
Date (mm/dd/yy).

@item %h
Same as %b.

@item %j
Day of year (001-366).

@item %m
Month (01-12).

@item %U
Week number of year with Sunday as first day of week (00-53).

@item %w
Day of week (0-6).

@item %W
Week number of year with Monday as first day of week (00-53).

@item %x
Locale's date representation (mm/dd/yy).

@item %y
Last two digits of year (00-99).

@item %Y
Year (1970-).
@end table
@end deftypefn

Most of the remaining functions described in this section are not
patterned after the standard C library.  Some are available for
compatiblity with @sc{Matlab} and others are provided because they are
useful.

@deftypefn {Function File} {} clock ()
Return a vector containing the current year, month (1-12), day (1-31),
hour (0-23), minute (0-59) and second (0-61).  For example,

@example
@group
clock ()
     @result{} [ 1993, 8, 20, 4, 56, 1 ]
@end group
@end example

The function clock is more accurate on systems that have the
@code{gettimeofday} function.
@end deftypefn

@deftypefn {Function File} {} date ()
Return the date as a character string in the form DD-MMM-YY.  For
example,

@example
@group
date ()
     @result{} "20-Aug-93"
@end group
@end example
@end deftypefn

@deftypefn {Function File} {} etime (@var{t1}, @var{t2})
Return the difference (in seconds) between two time values returned from
@code{clock}.  For example:

@example
t0 = clock ();
# many computations later...
elapsed_time = etime (clock (), t0);
@end example

@noindent
will set the variable @code{elapsed_time} to the number of seconds since
the variable @code{t0} was set.
@end deftypefn

@deftypefn {Built-in Function} {[@var{total}, @var{user}, @var{system}] =} cputime ();
Return the CPU time used by your Octave session.  The first output is
the total time spent executing your process and is equal to the sum of
second and third outputs, which are the number of CPU seconds spent
executing in user mode and the number of CPU seconds spent executing in
system mode, respectively.  If your system does not have a way to report
CPU time usage, @code{cputime} returns 0 for each of its output values.
Note that because Octave used some CPU time to start, it is reasonable
to check to see if @code{cputime} works by checking to see if the total
CPU time used is nonzero.
@end deftypefn

@deftypefn {Function File} {} is_leap_year (@var{year})
Return 1 if the given year is a leap year and 0 otherwise.  If no
arguments are provided, @code{is_leap_year} will use the current year.
For example,

@example
@group
is_leap_year (2000)
     @result{} 1
@end group
@end example
@end deftypefn

@deftypefn {Function File} {} tic ()
@deftypefnx {Function File} {} toc ()
These functions set and check a wall-clock timer.  For example,

@example
tic ();
# many computations later...
elapsed_time = toc ();
@end example

@noindent
will set the variable @code{elapsed_time} to the number of seconds since
the most recent call to the function @code{tic}.

If you are more interested in the CPU time that your process used, you
should use the @code{cputime} function instead.  The @code{tic} and
@code{toc} functions report the actual wall clock time that elapsed
between the calls.  This may include time spent processing other jobs or
doing nothing at all.  For example,

@example
@group
tic (); sleep (5); toc ()
     @result{} 5
t = cputime (); sleep (5); cputime () - t
     @result{} 0
@end group
@end example

@noindent
(This example also illustrates that the CPU timer may have a fairly
coarse resolution.)
@end deftypefn

@deftypefn {Built-in Function} {} pause (@var{seconds})
Suspend the execution of the program.  If invoked without any arguments,
Octave waits until you type a character.  With a numeric argument, it
pauses for the given number of seconds.  For example, the following
statement prints a message and then waits 5 seconds before clearing the
screen.

@example
@group
fprintf (stderr, "wait please...\n");
pause (5);
clc;
@end group
@end example
@end deftypefn

@deftypefn {Built-in Function} {} sleep (@var{seconds})
Suspend the execution of the program for the given number of seconds.
@end deftypefn

@deftypefn {Built-in Function} {} usleep (@var{microseconds})
Suspend the execution of the program for the given number of
microseconds.  On systems where it is not possible to sleep for periods
of time less than one second, @code{usleep} will pause the execution for
@code{round (@var{microseconds} / 1e6)} seconds.
@end deftypefn

@node Filesystem Utilities, Controlling Subprocesses, Timing Utilities, System Utilities
@section Filesystem Utilities

Octave includes the following functions for renaming and deleting files,
creating, deleting, and reading directories, and for getting information
about the status of files.

@deftypefn {Built-in Function} {[@var{err}, @var{msg}] =} rename (@var{old}, @var{new})
Change the name of file @var{old} to @var{new}.

If successful, @var{err} is 0 and @var{msg} is an empty string.
Otherwise, @var{err} is nonzero and @var{msg} contains a
system-dependent error message.
@end deftypefn

@deftypefn {Built-in Function} {[@var{err}, @var{msg}] =} unlink (@var{file})
Delete @var{file}.

If successful, @var{err} is 0 and @var{msg} is an empty string.
Otherwise, @var{err} is nonzero and @var{msg} contains a
system-dependent error message.
@end deftypefn

@deftypefn {Built-in Function} {[@var{files}, @var{err}, @var{msg}] =} readdir (@var{dir})
Return names of the files in the directory @var{dir} as an array of
strings.  If an error occurs, return an empty matrix in @var{files}.

If successful, @var{err} is 0 and @var{msg} is an empty string.
Otherwise, @var{err} is nonzero and @var{msg} contains a
system-dependent error message.
@end deftypefn

@deftypefn {Built-in Function} {[@var{err}, @var{msg}] =} mkdir (@var{dir})
Create a directory named @var{dir}.

If successful, @var{err} is 0 and @var{msg} is an empty string.
Otherwise, @var{err} is nonzero and @var{msg} contains a
system-dependent error message.
@end deftypefn

@deftypefn {Built-in Function} {[@var{err}, @var{msg}] =} rmdir (@var{dir})
Remove the directory named @var{dir}.

If successful, @var{err} is 0 and @var{msg} is an empty string.
Otherwise, @var{err} is nonzero and @var{msg} contains a
system-dependent error message.
@end deftypefn

@deftypefn {Built-in Function} {[@var{err}, @var{msg}] =} mkfifo (@var{name})
Create a FIFO special file.

If successful, @var{err} is 0 and @var{msg} is an empty string.
Otherwise, @var{err} is nonzero and @var{msg} contains a
system-dependent error message.
@end deftypefn

@c XXX FIXME XXX -- this needs to be explained, but I don't feel up to
@c it just now...

@deftypefn {Built-in Function} {} umask (@var{mask})
Set the permission mask for file creation.  The parameter @var{mask} is
interpreted as an octal number.
@end deftypefn

@deftypefn {Built-in Function} {[@var{info}, @var{err}, @var{msg}] =} stat (@var{file})
@deftypefnx {Built-in Function} {[@var{info}, @var{err}, @var{msg}] =} lstat (@var{file})
Return a structure @var{s} containing the following information about
@var{file}.

@table @code
@item dev
ID of device containing a directory entry for this file.

@item ino
File number of the file.

@item modestr
File mode, as a string of ten letters or dashes as would be returned by
@kbd{ls -l}.

@item nlink
Number of links.

@item uid
User ID of file's owner.

@item gid
Group ID of file's group.

@item rdev
ID of device for block or character special files.

@item size
Size in bytes.

@item atime
Time of last access in the same form as time values returned from
@code{time}.  @xref{Timing Utilities}.

@item mtime
Time of last modification in the same form as time values returned from
@code{time}.  @xref{Timing Utilities}.

@item ctime
Time of last file status change in the same form as time values returned from
@code{time}.  @xref{Timing Utilities}.

@item blksize
Size of blocks in the file.

@item blocks
Number of blocks allocated for file.
@end table

If the call is successful @var{err} is 0 and @var{msg} is an empty
string.  If the file does not exist, or some other error occurs, @var{s}
is an empty matrix, @var{err} is @minus{}1, and @var{msg} contains the
corresponding system error message.

If @var{file} is a symbolic link, @code{stat} will return information
about the actual file the is referenced by the link.  Use @code{lstat}
if you want information about the symbolic link itself.

For example,

@example
@group
[s, err, msg] = stat ("/vmlinuz")
     @result{} s =
        @{
          atime = 855399756
          rdev = 0
          ctime = 847219094
          uid = 0
          size = 389218
          blksize = 4096
          mtime = 847219094
          gid = 6
          nlink = 1
          blocks = 768
          modestr = -rw-r--r--
          ino = 9316
          dev = 2049
        @}
     @result{} err = 0
     @result{} msg = 
@end group
@end example
@end deftypefn

@deftypefn {Built-in Function} {} glob (@var{pattern})
Given an array of strings in @var{pattern}, return the list of file
names that any of them, or an empty string if no patterns match.  Tilde
expansion is performed on each of the patterns before looking for
matching file names.  For example,

@example
@group
glob ("/vm*")
     @result{} "/vmlinuz"
@end group
@end example

Note that multiple values are returned in a string matrix with the fill
character set to ASCII NUL.
@end deftypefn

@deftypefn {Built-in Function} {} fnmatch (@var{pattern}, @var{string})
Return 1 or zero for each element of @var{string} that matches any of
the elements of the string array @var{pattern}, using the rules of
filename pattern matching.  For example,

@example
@group
fnmatch ("a*b", ["ab"; "axyzb"; "xyzab"])
     @result{} [ 1; 1; 0 ]
@end group
@end example
@end deftypefn

@deftypefn {Built-in Function} {} file_in_path (@var{path}, @var{file})
Return the absolute name name of @var{file} if it can be found in
@var{path}.  The value of @var{path} should be a colon-separated list of
directories in the format described for the built-in variable
@code{LOADPATH}.

If the file cannot be found in the path, an empty matrix is returned.
For example,

@example
file_in_path (LOADPATH, "nargchk.m")
     @result{} "@value{OCTAVEHOME}/share/octave/2.0/m/general/nargchk.m"
@end example
@end deftypefn

@deftypefn {Built-in Function} {} tilde_expand (@var{string})
Performs tilde expansion on @var{string}.  If @var{string} begins with a
tilde character, (@samp{~}), all of the characters preceding the first
slash (or all characters, if there is no slash) are treated as a
possible user name, and the tilde and the following characters up to the
slash are replaced by the home directory of the named user.  If the
tilde is followed immediately by a slash, the tilde is replaced by the
home directory of the user running Octave.  For example,

@example
@group
tilde_expand ("~joeuser/bin")
     @result{} "/home/joeuser/bin"
tilde_expand ("~/bin")
     @result{} "/home/jwe/bin"
@end group
@end example
@end deftypefn

@node Controlling Subprocesses, Process ID Information, Filesystem Utilities, System Utilities
@section Controlling Subprocesses

Octave includes some high-level commands like @code{system} and
@code{popen} for starting subprocesses.  If you want to run another
program to perform some task and then look at its output, you will
probably want to use these functions.

Octave also provides several very low-level Unix-like functions which
can also be used for starting subprocesses, but you should probably only
use them if you can't find any way to do what you need with the
higher-level functions.

@deftypefn {Built-in Function} {} system (@var{string}, @var{return_output}, @var{type})
Execute a shell command specified by @var{string}.  The second argument is optional.
If @var{type} is @code{"async"}, the process is started in the
background and the process id of the child process is returned
immediately.  Otherwise, the process is started, and Octave waits until
it exits.  If @var{type} argument is omitted, a value of @code{"sync"}
is assumed.

If two input arguments are given (the actual value of
@var{return_output} is irrelevant) and the subprocess is started
synchronously, or if @var{system} is called with one input argument and
one or more output arguments, the output from the command is returned.
Otherwise, if the subprocess is executed synchronously, it's output is
sent to the standard output.  To send the output of a command executed
with @var{system} through the pager, use a command like

@example
disp (system (cmd, 1));
@end example

@noindent
or

@example
printf ("%s\n", system (cmd, 1));
@end example

The @code{system} function can return two values.  The first is any
output from the command that was written to the standard output stream,
and the second is the output status of the command.  For example,

@example
[output, status] = system ("echo foo; exit 2");
@end example

@noindent
will set the variable @code{output} to the string @samp{foo}, and the
variable @code{status} to the integer @samp{2}.
@end deftypefn

@deftypefn {Built-in Function} {fid =} popen (@var{command}, @var{mode})
Start a process and create a pipe.  The name of the command to run is
given by @var{command}.  The file identifier corresponding to the input
or output stream of the process is returned in @var{fid}.  The argument
@var{mode} may be

@table @code
@item "r"
The pipe will be connected to the standard output of the process, and
open for reading.

@item "w"
The pipe will be connected to the standard input of the process, and
open for writing.
@end table

For example,

@example
@group
fid = popen ("ls -ltr / | tail -3", "r");
while (isstr (s = fgets (fid)))
  fputs (stdout, s);
endwhile
     @print{} drwxr-xr-x  33 root  root  3072 Feb 15 13:28 etc
     @print{} drwxr-xr-x   3 root  root  1024 Feb 15 13:28 lib
     @print{} drwxrwxrwt  15 root  root  2048 Feb 17 14:53 tmp
@end group
@end example
@end deftypefn

@deftypefn {Built-in Function} {} pclose (@var{fid})
Close a file identifier that was opened by @code{popen}.  You may also
use @code{fclose} for the same purpose.
@end deftypefn

@deftypefn {Built-in Function} {[@var{in}, @var{out}, @var{pid}] =} popen2 (@var{command}, @var{args})
Start a subprocess with two-way communication.  The name of the process
is given by @var{command}, and @var{args} is an array of strings
containing options for the command.  The file identifiers for the input
and output streams of the subprocess are returned in @var{in} and
@var{out}.  If execution of the command is successful, @var{pid}
contains the process ID of the subprocess.  Otherwise, @var{pid} is
@minus{}1.

For example,

@example
@group
[in, out, pid] = popen2 ("sort", "-nr");
fputs (in, "these\nare\nsome\nstrings\n");
fclose (in);
while (isstr (s = fgets (out)))
  fputs (stdout, s);
endwhile
fclose (out);
     @print{} are
     @print{} some
     @print{} strings
     @print{} these
@end group
@end example
@end deftypefn

@defvr {Built-in Variable} EXEC_PATH
The variable @code{EXEC_PATH} is a colon separated list of directories
to search when executing subprograms.  Its initial value is taken from
the environment variable @code{OCTAVE_EXEC_PATH} (if it exists) or
@code{PATH}, but that value can be overridden by the command line
argument @code{--exec-path PATH}, or by setting the value of
@code{EXEC_PATH} in a startup script.  If the value of @code{EXEC_PATH}
begins (ends) with a colon, the directories

@example
@group
@var{octave-home}/libexec/octave/site/exec/@var{arch}
@var{octave-home}/libexec/octave/@var{version}/exec/@var{arch}
@end group
@end example

@noindent
are prepended (appended) to @code{EXEC_PATH}, where @var{octave-home}
is the top-level directory where all of Octave is installed
(the default value is @file{@value{OCTAVEHOME}}).  If you don't specify
a value for @code{EXEC_PATH} explicitly, these special directories are
prepended to your shell path.
@end defvr

In most cases, the following functions simply decode their arguments and
make the corresponding Unix system calls.  For a complete example of how
they can be used, look at the definition of the function @code{popen2}.

@deftypefn {Built-in Function} {[@var{pid}, @var{msg}] =} fork ()
Create a copy of the current process.

Fork can return one of the following values:

@table @asis
@item > 0
You are in the parent process.  The value returned from @code{fork} is
the process id of the child process.  You should probably arrange to
wait for any child processes to exit.

@item 0
You are in the child process.  You can call @code{exec} to start another
process.  If that fails, you should probably call @code{exit}.

@item < 0
The call to @code{fork} failed for some reason.  You must take evasive
action.  A system dependent error message will be waiting in @var{msg}.
@end table
@end deftypefn

@deftypefn {Built-in Function} {[@var{err}, @var{msg}] =} exec (@var{file}, @var{args})
Replace current process with a new process.  Calling @code{exec} without
first calling @code{fork} will terminate your current Octave process and
replace it with the program named by @var{file}.  For example,

@example
exec ("ls" "-l")
@end example

@noindent
will run @code{ls} and return you to your shell prompt.

If successful, @code{exec} does not return.  If @code{exec} does return,
@var{err} will be nonzero, and @var{msg} will contain a system-dependent
error message.
@end deftypefn

@deftypefn {Built-in Function} {[@var{file_ids}, @var{err}, @var{msg}] =} pipe ()
Create a pipe and return the vector @var{file_ids}, which corresponding
to the reading and writing ends of the pipe.

If successful, @var{err} is 0 and @var{msg} is an empty string.
Otherwise, @var{err} is nonzero and @var{msg} contains a
system-dependent error message.
@end deftypefn

@deftypefn {Built-in Function} {[@var{fid}, @var{msg}] =} dup2 (@var{old}, @var{new})
Duplicate a file descriptor.

If successful, @var{fid} is greater than zero and contains the new file
ID.  Otherwise, @var{fid} is negative and @var{msg} contains a
system-dependent error message.
@end deftypefn

@deftypefn {Built-in Function} {[@var{pid}, @var{msg}] =} waitpid (@var{pid}, @var{options})
Wait for process @var{pid} to terminate.  The @var{pid} argument can be:

@table @asis
@item @minus{}1
Wait for any child process.

@item 0
Wait for any child process whose process group ID is equal to that of
the Octave interpreter process.

@item > 0
Wait for termination of the child process with ID @var{pid}.
@end table

The @var{options} argument can be:

@table @asis
@item 0
Wait until signal is received or a child process exits (this is the
default if the @var{options} argument is missing).

@item 1
Do not hang if status is not immediately available.

@item 2
Report the status of any child processes that are stopped, and whose
status has not yet been reported since they stopped.

@item 3
Implies both 1 and 2.
@end table

If the returned value of @var{pid} is greater than 0, it is the process
ID of the child process that exited.  If an error occurs, @var{pid} will
be less than zero and @var{msg} will contain a system-dependent error
message.
@end deftypefn

@deftypefn {Built-in Function} {[@var{err}, @var{msg}] =} fcntl (@var{fid}, @var{request}, @var{arg})
Change the properties of the open file @var{fid}.  The following values
may be passed as @var{request}:

@vtable @code
@item F_DUPFD
Return a duplicate file descriptor.

@item F_GETFD
Return the file descriptor flags for @var{fid}.

@item F_SETFD
Set the file descriptor flags for @var{fid}.

@item F_GETFL
Return the file status flags for @var{fid}.  The following codes may be
returned (some of the flags may be undefined on some systems).

@vtable @code
@item O_RDONLY
Open for reading only.

@item O_WRONLY
Open for writing only.

@item O_RDWR
Open for reading and writing.

@item O_APPEND
Append on each write.

@item O_NONBLOCK
Nonblocking mode.

@item O_SYNC
Wait for writes to complete.

@item O_ASYNC
Asynchronous I/O.
@end vtable

@item F_SETFL
Set the file status flags for @var{fid} to the value specified by
@var{arg}.  The only flags that can be changed are @code{O_APPEND} and
@code{O_NONBLOCK}.
@end vtable

If successful, @var{err} is 0 and @var{msg} is an empty string.
Otherwise, @var{err} is nonzero and @var{msg} contains a
system-dependent error message.
@end deftypefn

@node Process ID Information, Environment Variables, Controlling Subprocesses, System Utilities
@section Process, Group, and User IDs

@deftypefn {Built-in Function} {} getpgrp ()
Return the process group id of the current process.
@end deftypefn

@deftypefn {Built-in Function} {} getpid ()
Return the process id of the current process.
@end deftypefn

@deftypefn {Built-in Function} {} getppid ()
Return the process id of the parent process.
@end deftypefn

@deftypefn {Built-in Function} {} geteuid ()
Return the effective user id of the current process.
@end deftypefn

@deftypefn {Built-in Function} {} getuid ()
Return the real user id of the current process.
@end deftypefn

@deftypefn {Built-in Function} {} getegid ()
Return the effective group id of the current process.
@end deftypefn

@deftypefn {Built-in Function} {} getgid ()
Return the real group id of the current process.
@end deftypefn

@node Environment Variables, Current Working Directory, Process ID Information, System Utilities
@section Environment Variables

@deftypefn {Built-in Function} {} getenv (@var{var})
Return the value of the environment variable @var{var}.  For example,

@example
getenv ("PATH")
@end example

@noindent
returns a string containing the value of your path.
@end deftypefn

@deftypefn {Built-in Function} {} putenv (@var{var}, @var{value})
Set the value of the environment variable @var{var} to @var{value}.
@end deftypefn

@node Current Working Directory, Password Database Functions, Environment Variables, System Utilities
@section Current Working Directory

@deffn {Command} cd dir
@deffnx {Command} chdir dir
Change the current working directory to @var{dir}.  For example,

@example
cd ~/octave
@end example

@noindent
Changes the current working directory to @file{~/octave}.  If the
directory does not exist, an error message is printed and the working
directory is not changed.
@end deffn

@deftypefn {Built-in Function} {} pwd ()
Return the current working directory.
@end deftypefn

@deffn {Command} ls options
@deffnx {Command} dir options
List directory contents.  For example,

@example
ls -l
     @print{} total 12
     @print{} -rw-r--r--   1 jwe  users  4488 Aug 19 04:02 foo.m
     @print{} -rw-r--r--   1 jwe  users  1315 Aug 17 23:14 bar.m
@end example

The @code{dir} and @code{ls} commands are implemented by calling your
system's directory listing command, so the available options may vary
from system to system.
@end deffn

@node Password Database Functions, Group Database Functions, Current Working Directory, System Utilities
@section Password Database Functions

Octave's password database functions return information in a structure
with the following fields.

@table @code
@item name
The user name.

@item passwd
The encrypted password, if available.

@item uid
The numeric user id.

@item gid
The numeric group id.

@item gecos
The GECOS field.

@item dir
The home directory.

@item shell
The initial shell.
@end table

In the descriptions of the following functions, this data structure is
referred to as a @var{pw_struct}.

@deftypefn {Loadable Function} {@var{pw_struct} = } getpwent ()
Return a structure containing an entry from the password database,
opening it if necessary. Once the end of the data has been reached,
@code{getpwent} returns 0.
@end deftypefn

@deftypefn {Loadable Function} {@var{pw_struct} = } getpwuid (@var{uid}).
Return a structure containing the first entry from the password database
with the user ID @var{uid}.  If the user ID does not exist in the
database, @code{getpwuid} returns 0.
@end deftypefn

@deftypefn {Loadable Function} {@var{pw_struct} = } getpwnam (@var{name})
Return a structure containing the first entry from the password database
with the user name @var{name}.  If the user name does not exist in the
database, @code{getpwname} returns 0.
@end deftypefn

@deftypefn {Loadable Function} {} setpwent ()
Return the internal pointer to the beginning of the password database.
@end deftypefn

@deftypefn {Loadable Function} {} endpwent ()
Close the password database.
@end deftypefn

@node Group Database Functions, System Information, Password Database Functions, System Utilities
@section Group Database Functions

Octave's group database functions return information in a structure
with the following fields.

@table @code
@item name
The user name.

@item passwd
The encrypted password, if available.

@item gid
The numeric group id.

@item mem
The members of the group.
@end table

In the descriptions of the following functions, this data structure is
referred to as a @var{grp_struct}.

@deftypefn {Loadable Function} {@var{grp_struct} =} getgrent ()
Return an entry from the group database, opening it if necessary.
Once the end of the data has been reached, @code{getgrent} returns 0.
@end deftypefn

@deftypefn {Loadable Function} {@var{grp_struct} =} getgrgid (@var{gid}).
Return the first entry from the group database with the group ID
@var{gid}.  If the group ID does not exist in the database,
@code{getgrgid} returns 0.
@end deftypefn

@deftypefn {Loadable Function} {@var{grp_struct} =} getgrnam (@var{name})
Return the first entry from the group database with the group name
@var{name}.  If the group name does not exist in the database,
@code{getgrname} returns 0.
@end deftypefn

@deftypefn {Loadable Function} {} setgrent ()
Return the internal pointer to the beginning of the group database.
@end deftypefn

@deftypefn {Loadable Function} {} endgrent ()
Close the group database.
@end deftypefn

@node System Information,  , Group Database Functions, System Utilities
@section System Information

@deftypefn {Built-in Function} {} computer ()
Print or return a string of the form @var{cpu}-@var{vendor}-@var{os}
that identifies the kind of computer Octave is running on.  If invoked
with an output argument, the value is returned instead of printed.  For
example,

@example
@group
computer ()
     @print{} i586-pc-linux-gnu

x = computer ()
     @result{} x = "i586-pc-linux-gnu"
@end group
@end example
@end deftypefn

@deftypefn {Built-in Function} {} isieee ()
Return 1 if your computer claims to conform to the IEEE standard for
floating point calculations.
@end deftypefn

@deftypefn {Built-in Function} {} version ()
Return Octave's version number as a string.  This is also the value of
the built-in variable @code{OCTAVE_VERSION}.
@end deftypefn

@defvr {Built-in Variable} OCTAVE_VERSION
The version number of Octave, as a string.
@end defvr

@deftypefn {Built-in Function} {} octave_config_info ()
Return a structure containing configuration and installation
information.
@end deftypefn

@deftypefn {Loadable Function} {} getrusage ()
Return a structure containing a number of statistics about the current
Octave process.  Not all fields are available on all systems.  If it is
not possible to get CPU time statistics, the CPU time slots are set to
zero.  Other missing data are replaced by NaN.  Here is a list of all
the possible fields that can be present in the structure returned by
@code{getrusage}:

@table @code
@item 
@item idrss
Unshared data size.

@item inblock
Number of block input operations.

@item isrss
Unshared stack size.

@item ixrss
Shared memory size.

@item majflt
Number of major page faults.

@item maxrss
Maximum data size.

@item minflt
Number of minor page faults.

@item msgrcv
Number of messages received.

@item msgsnd
Number of messages sent.

@item nivcsw
Number of involuntary context switches.

@item nsignals
Number of signals received.

@item nswap
Number of swaps.

@item nvcsw
Number of voluntary context switches.

@item oublock
Number of block output operations.

@item stime
A structure containing the system CPU time used.  The structure has the
elements @code{sec} (seconds) @code{usec} (microseconds).

@item utime
A structure containing the user CPU time used.  The structure has the
elements @code{sec} (seconds) @code{usec} (microseconds).
@end table
@end deftypefn
