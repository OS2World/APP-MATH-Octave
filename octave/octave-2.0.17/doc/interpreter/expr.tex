@c Copyright (C) 1996, 1997 John W. Eaton
@c This is part of the Octave manual.
@c For copying conditions, see the file gpl.tex.

@node Expressions, Evaluation, Variables, Top
@chapter Expressions
@cindex expressions

Expressions are the basic building block of statements in Octave.  An
expression evaluates to a value, which you can print, test, store in a
variable, pass to a function, or assign a new value to a variable with
an assignment operator.

An expression can serve as a statement on its own.  Most other kinds of
statements contain one or more expressions which specify data to be
operated on.  As in other languages, expressions in Octave include
variables, array references, constants, and function calls, as well as
combinations of these with various operators.

@menu
* Index Expressions::           
* Calling Functions::           
* Arithmetic Ops::              
* Comparison Ops::              
* Boolean Expressions::         
* Assignment Ops::              
* Increment Ops::               
* Operator Precedence::         
@end menu

@node Index Expressions, Calling Functions, Expressions, Expressions
@section Index Expressions

@opindex (
@opindex )

An @dfn{index expression} allows you to reference or extract selected
elements of a matrix or vector.

Indices may be scalars, vectors, ranges, or the special operator
@samp{:}, which may be used to select entire rows or columns.

Vectors are indexed using a single expression.  Matrices require two
indices unless the value of the built-in variable
@code{do_fortran_indexing} is nonzero, in which case matrices may
also be indexed by a single expression.

@defvr {Built-in Variable} do_fortran_indexing
If the value of @code{do_fortran_indexing} is nonzero, Octave allows 
you to select elements of a two-dimensional matrix using a single index
by treating the matrix as a single vector created from the columns of
the matrix.  The default value is 0. 
@end defvr

Given the matrix

@example
a = [1, 2; 3, 4]
@end example

@noindent
all of the following expressions are equivalent

@example
@group
a (1, [1, 2])
a (1, 1:2)
a (1, :)
@end group
@end example

@noindent
and select the first row of the matrix.

A special form of indexing may be used to select elements of a matrix or
vector.  If the indices are vectors made up of only ones and zeros, the
result is a new matrix whose elements correspond to the elements of the
index vector that are equal to one.  For example,

@example
@group
a = [1, 2; 3, 4];
a ([1, 0], :)
@end group
@end example

@noindent
selects the first row of the matrix @code{a}.

This operation can be useful for selecting elements of a matrix based on
some condition, since the comparison operators return matrices of ones
and zeros.

This special zero-one form of indexing leads to a conflict with the
standard indexing operation.  For example, should the following
statements

@example
@group
a = [1, 2; 3, 4];
a ([1, 1], :)
@end group
@end example

@noindent
return the original matrix, or the matrix formed by selecting the first
row twice?  Although this conflict is not likely to arise very often in
practice, you may select the behavior you prefer by setting the built-in
variable @code{prefer_zero_one_indexing}.

@defvr {Built-in Variable} prefer_zero_one_indexing
If the value of @code{prefer_zero_one_indexing} is nonzero, Octave
will perform zero-one style indexing when there is a conflict with the
normal indexing rules.  @xref{Index Expressions}.  For example, given a
matrix

@example
a = [1, 2, 3, 4]
@end example

@noindent
with @code{prefer_zero_one_indexing} is set to nonzero, the
expression

@example
a ([1, 1, 1, 1])
@end example

@noindent
results in the matrix @code{[ 1, 2, 3, 4 ]}.  If the value of
@code{prefer_zero_one_indexing} set to 0, the result would be
the matrix @code{[ 1, 1, 1, 1 ]}.

In the first case, Octave is selecting each element corresponding to a
@samp{1} in the index vector.  In the second, Octave is selecting the
first element multiple times.

The default value for @code{prefer_zero_one_indexing} is 0.
@end defvr

Finally, indexing a scalar with a vector of ones can be used to create a
vector the same size as the index vector, with each element equal to
the value of the original scalar.  For example, the following statements

@example
@group
a = 13;
a ([1, 1, 1, 1])
@end group
@end example

@noindent
produce a vector whose four elements are all equal to 13.

Similarly, indexing a scalar with two vectors of ones can be used to
create a matrix.  For example the following statements

@example
@group
a = 13;
a ([1, 1], [1, 1, 1])
@end group
@end example

@noindent
create a 2 by 3 matrix with all elements equal to 13.

This is an obscure notation and should be avoided.  It is better to
use the function @code{ones} to generate a matrix of the appropriate
size whose elements are all one, and then to scale it to produce the
desired result.  @xref{Special Utility Matrices}.

@defvr {Built-in Variable} prefer_column_vectors
If @code{prefer_column_vectors} is nonzero, operations like

@example
for i = 1:10
  a (i) = i;
endfor
@end example

@noindent
(for @code{a} previously  undefined) produce column vectors.  Otherwise, row
vectors are preferred.  The default value is 1.

If a variable is already defined to be a vector (a matrix with a single
row or column), the original orientation is respected, regardless of the
value of @code{prefer_column_vectors}.
@end defvr

@defvr {Built-in Variable} resize_on_range_error
If the value of @code{resize_on_range_error} is nonzero, expressions
like

@example
for i = 1:10
  a (i) = sqrt (i);
endfor
@end example

@noindent
(for @code{a} previously undefined) result in the variable @code{a}
being resized to be just large enough to hold the new value.  New
elements that have not been given a value are set to zero.  If the value
of @code{resize_on_range_error} is 0, an error message is printed and
control is returned to the top level.  The default value is 1.
@end defvr

Note that it is quite inefficient to create a vector using a loop like
the one shown in the example above.  In this particular case, it would
have been much more efficient to use the expression

@example
a = sqrt (1:10);
@end example

@noindent
thus avoiding the loop entirely.  In cases where a loop is still
required, or a number of values must be combined to form a larger
matrix, it is generally much faster to set the size of the matrix first,
and then insert elements using indexing commands.  For example, given a
matrix @code{a},

@example
@group
[nr, nc] = size (a);
x = zeros (nr, n * nc);
for i = 1:n
  x(:,(i-1)*n+1:i*n) = a;
endfor
@end group
@end example

@noindent
is considerably faster than

@example
@group
x = a;
for i = 1:n-1
  x = [x, a];
endfor
@end group
@end example

@noindent
particularly for large matrices because Octave does not have to
repeatedly resize the result.

@node Calling Functions, Arithmetic Ops, Index Expressions, Expressions
@section Calling Functions

A @dfn{function} is a name for a particular calculation.  Because it has
a name, you can ask for it by name at any point in the program.  For
example, the function @code{sqrt} computes the square root of a number.

A fixed set of functions are @dfn{built-in}, which means they are
available in every Octave program.  The @code{sqrt} function is one of
these.  In addition, you can define your own functions.
@xref{Functions and Scripts}, for information about how to do this.

@cindex arguments in function call
The way to use a function is with a @dfn{function call} expression,
which consists of the function name followed by a list of
@dfn{arguments} in parentheses. The arguments are expressions which give
the raw materials for the calculation that the function will do.  When
there is more than one argument, they are separated by commas.  If there
are no arguments, you can omit the parentheses, but it is a good idea to
include them anyway, to clearly indicate that a function call was
intended.  Here are some examples:

@example
@group
sqrt (x^2 + y^2)      # @r{One argument}
ones (n, m)           # @r{Two arguments}
rand ()               # @r{No arguments}
@end group
@end example

Each function expects a particular number of arguments.  For example, the
@code{sqrt} function must be called with a single argument, the number
to take the square root of:

@example
sqrt (@var{argument})
@end example

Some of the built-in functions take a variable number of arguments,
depending on the particular usage, and their behavior is different
depending on the number of arguments supplied.

Like every other expression, the function call has a value, which is
computed by the function based on the arguments you give it.  In this
example, the value of @code{sqrt (@var{argument})} is the square root of
the argument.  A function can also have side effects, such as assigning
the values of certain variables or doing input or output operations.

Unlike most languages, functions in Octave may return multiple values.
For example, the following statement

@example
[u, s, v] = svd (a)
@end example

@noindent
computes the singular value decomposition of the matrix @code{a} and
assigns the three result matrices to @code{u}, @code{s}, and @code{v}.

The left side of a multiple assignment expression is itself a list of
expressions, and is allowed to be a list of variable names or index
expressions.  See also @ref{Index Expressions}, and @ref{Assignment Ops}.

@menu
* Call by Value::               
* Recursion::                   
@end menu

@node Call by Value, Recursion, Calling Functions, Calling Functions
@subsection Call by Value

In Octave, unlike Fortran, function arguments are passed by value, which
means that each argument in a function call is evaluated and assigned to
a temporary location in memory before being passed to the function.
There is currently no way to specify that a function parameter should be
passed by reference instead of by value.  This means that it is
impossible to directly alter the value of function parameter in the
calling function.  It can only change the local copy within the function
body.  For example, the function

@example
@group
function f (x, n)
  while (n-- > 0)
    disp (x);
  endwhile
endfunction
@end group
@end example

@noindent
displays the value of the first argument @var{n} times.  In this
function, the variable @var{n} is used as a temporary variable without
having to worry that its value might also change in the calling
function.  Call by value is also useful because it is always possible to
pass constants for any function parameter without first having to
determine that the function will not attempt to modify the parameter.

The caller may use a variable as the expression for the argument, but
the called function does not know this: it only knows what value the
argument had.  For example, given a function called as

@example
@group
foo = "bar";
fcn (foo)
@end group
@end example

@noindent
you should not think of the argument as being ``the variable
@code{foo}.''  Instead, think of the argument as the string value,
@code{"bar"}.

Even though Octave uses pass-by-value semantics for function arguments,
values are not copied unnecessarily.  For example,

@example
@group
x = rand (1000);
f (x);
@end group
@end example

@noindent
does not actually force two 1000 by 1000 element matrices to exist
@emph{unless} the function @code{f} modifies the value of its
argument.  Then Octave must create a copy to avoid changing the
value outside the scope of the function @code{f}, or attempting (and
probably failing!) to modify the value of a constant or the value of a
temporary result.

@node Recursion,  , Call by Value, Calling Functions
@subsection Recursion
@cindex factorial function

With some restrictions@footnote{Some of Octave's function are
implemented in terms of functions that cannot be called recursively.
For example, the ODE solver @code{lsode} is ultimately implemented in a
Fortran subroutine that cannot be called recursively, so @code{lsode}
should not be called either directly or indirectly from within the
user-supplied function that @code{lsode} requires.  Doing so will result
in undefined behavior.}, recursive function calls are allowed.  A
@dfn{recursive function} is one which calls itself, either directly or
indirectly.  For example, here is an inefficient@footnote{It would be
much better to use @code{prod (1:n)}, or @code{gamma (n+1)} instead,
after first checking to ensure that the value @code{n} is actually a
positive integer.} way to compute the factorial of a given integer:

@example
@group
function retval = fact (n)
  if (n > 0)
    retval = n * fact (n-1);
  else
    retval = 1;
  endif
endfunction
@end group
@end example

This function is recursive because it calls itself directly.  It
eventually terminates because each time it calls itself, it uses an
argument that is one less than was used for the previous call.  Once the
argument is no longer greater than zero, it does not call itself, and
the recursion ends.

The built-in variable @code{max_recursion_depth} specifies a limit to
the recursion depth and prevents Octave from recursing infinitely.

@defvr max_recursion_depth
Limit the number of times a function may be called recursively.
If the limit is exceeded, an error message is printed and control
returns to the top level.

The default value is 256.
@end defvr

@node Arithmetic Ops, Comparison Ops, Calling Functions, Expressions
@section Arithmetic Operators
@cindex arithmetic operators
@cindex operators, arithmetic
@cindex addition
@cindex subtraction
@cindex multiplication
@cindex matrix multiplication
@cindex division
@cindex quotient
@cindex negation
@cindex unary minus
@cindex exponentiation
@cindex transpose
@cindex Hermitian operator
@cindex transpose, complex-conjugate
@cindex complex-conjugate transpose

The following arithmetic operators are available, and work on scalars
and matrices.

@table @code
@item @var{x} + @var{y}
@opindex +
Addition.  If both operands are matrices, the number of rows and columns
must both agree.  If one operand is a scalar, its value is added to
all the elements of the other operand.

@item @var{x} .+ @var{y}
@opindex .+
Element by element addition.  This operator is equivalent to @code{+}.

@item @var{x} - @var{y}
@opindex -
Subtraction.  If both operands are matrices, the number of rows and
columns of both must agree.

@item @var{x} .- @var{y}
Element by element subtraction.  This operator is equivalent to @code{-}.

@item @var{x} * @var{y}
@opindex *
Matrix multiplication.  The number of columns of @var{x} must agree
with the number of rows of @var{y}.

@item @var{x} .* @var{y}
@opindex .*
Element by element multiplication.  If both operands are matrices, the
number of rows and columns must both agree.

@item @var{x} / @var{y}
@opindex /
Right division.  This is conceptually equivalent to the expression

@example
(inverse (y') * x')'
@end example

@noindent
but it is computed without forming the inverse of @var{y'}.

If the system is not square, or if the coefficient matrix is singular,
a minimum norm solution is computed.

@item @var{x} ./ @var{y}
@opindex ./
Element by element right division.

@item @var{x} \ @var{y}
@opindex \
Left division.  This is conceptually equivalent to the expression

@example
inverse (x) * y
@end example

@noindent
but it is computed without forming the inverse of @var{x}.

If the system is not square, or if the coefficient matrix is singular,
a minimum norm solution is computed.

@item @var{x} .\ @var{y}
@opindex .\
Element by element left division.  Each element of @var{y} is divided
by each corresponding element of @var{x}.

@item @var{x} ^ @var{y}
@itemx @var{x} ** @var{y}
@opindex **
@opindex ^
Power operator.  If @var{x} and @var{y} are both scalars, this operator
returns @var{x} raised to the power @var{y}.  If @var{x} is a scalar and
@var{y} is a square matrix, the result is computed using an eigenvalue
expansion.  If @var{x} is a square matrix. the result is computed by
repeated multiplication if @var{y} is an integer, and by an eigenvalue
expansion if @var{y} is not an integer.  An error results if both
@var{x} and @var{y} are matrices.

The implementation of this operator needs to be improved.

@item @var{x} .^ @var{y}
@item @var{x} .** @var{y}
@opindex .**
@opindex .^
Element by element power operator.  If both operands are matrices, the
number of rows and columns must both agree.

@item -@var{x}
@opindex -
Negation.

@item +@var{x}
@opindex +
Unary plus.  This operator has no effect on the operand.

@item @var{x}'
@opindex '
Complex conjugate transpose.  For real arguments, this operator is the
same as the transpose operator.  For complex arguments, this operator is
equivalent to the expression

@example
conj (x.')
@end example

@item @var{x}.'
@opindex .'
Transpose.
@end table

Note that because Octave's element by element operators begin with a
@samp{.}, there is a possible ambiguity for statements like

@example
1./m
@end example

@noindent
because the period could be interpreted either as part of the constant
or as part of the operator.  To resolve this conflict, Octave treats the
expression as if you had typed

@example
(1) ./ m
@end example

@noindent
and not

@example
(1.) / m
@end example

@noindent
Although this is inconsistent with the normal behavior of Octave's
lexer, which usually prefers to break the input into tokens by
preferring the longest possible match at any given point, it is more
useful in this case.

@defvr {Built-in Variable} warn_divide_by_zero
If the value of @code{warn_divide_by_zero} is nonzero, a warning
is issued when Octave encounters a division by zero.  If the value is
0, the warning is omitted.  The default value is 1.
@end defvr

@node Comparison Ops, Boolean Expressions, Arithmetic Ops, Expressions
@section Comparison Operators
@cindex comparison expressions
@cindex expressions, comparison
@cindex relational operators
@cindex operators, relational
@cindex less than operator
@cindex greater than operator
@cindex equality operator
@cindex tests for equality
@cindex equality, tests for

@dfn{Comparison operators} compare numeric values for relationships
such as equality.  They are written using
@emph{relational operators}.

All of Octave's comparison operators return a value of 1 if the
comparison is true, or 0 if it is false.  For matrix values, they all
work on an element-by-element basis.  For example,

@example
@group
[1, 2; 3, 4] == [1, 3; 2, 4]
     @result{}  1  0
         0  1
@end group
@end example

If one operand is a scalar and the other is a matrix, the scalar is
compared to each element of the matrix in turn, and the result is the
same size as the matrix.

@table @code
@item @var{x} < @var{y}
@opindex <
True if @var{x} is less than @var{y}.

@item @var{x} <= @var{y}
@opindex <=
True if @var{x} is less than or equal to @var{y}.

@item @var{x} == @var{y}
@opindex ==
True if @var{x} is equal to @var{y}.

@item @var{x} >= @var{y}
@opindex >=
True if @var{x} is greater than or equal to @var{y}.

@item @var{x} > @var{y}
@opindex >
True if @var{x} is greater than @var{y}.

@item @var{x} != @var{y}
@itemx @var{x} ~= @var{y}
@itemx @var{x} <> @var{y}
@opindex !=
@opindex ~=
@opindex <>
True if @var{x} is not equal to @var{y}.
@end table

String comparisons may also be performed with the @code{strcmp}
function, not with the comparison operators listed above.
@xref{Strings}.

@node Boolean Expressions, Assignment Ops, Comparison Ops, Expressions
@section Boolean Expressions
@cindex expressions, boolean
@cindex boolean expressions
@cindex expressions, logical
@cindex logical expressions
@cindex operators, boolean
@cindex boolean operators
@cindex logical operators
@cindex operators, logical
@cindex and operator
@cindex or operator
@cindex not operator

@menu
* Element-by-element Boolean Operators::  
* Short-circuit Boolean Operators::  
@end menu

@node Element-by-element Boolean Operators, Short-circuit Boolean Operators, Boolean Expressions, Boolean Expressions
@subsection Element-by-element Boolean Operators
@cindex element-by-element evaluation

An @dfn{element-by-element boolean expression} is a combination of
comparison expressions using the boolean
operators ``or'' (@samp{|}), ``and'' (@samp{&}), and ``not'' (@samp{!}),
along with parentheses to control nesting.  The truth of the boolean
expression is computed by combining the truth values of the
corresponding elements of the component expressions.  A value is
considered to be false if it is zero, and true otherwise.

Element-by-element boolean expressions can be used wherever comparison
expressions can be used.  They can be used in @code{if} and @code{while}
statements.  However, if a matrix value used as the condition in an
@code{if} or @code{while} statement is only true if @emph{all} of its
elements are nonzero.

Like comparison operations, each element of an element-by-element
boolean expression also has a numeric value (1 if true, 0 if false) that
comes into play if the result of the boolean expression is stored in a
variable, or used in arithmetic.

Here are descriptions of the three element-by-element boolean operators.

@table @code
@item @var{boolean1} & @var{boolean2}
@opindex &
Elements of the result are true if both corresponding elements of
@var{boolean1} and @var{boolean2} are true.

@item @var{boolean1} | @var{boolean2}
@opindex |
Elements of the result are true if either of the corresponding elements
of @var{boolean1} or @var{boolean2} is true.

@item ! @var{boolean}
@itemx ~ @var{boolean}
@opindex ~
@opindex !
Each element of the result is true if the corresponding element of
@var{boolean} is false.
@end table

For matrix operands, these operators work on an element-by-element
basis.  For example, the expression

@example
[1, 0; 0, 1] & [1, 0; 2, 3]
@end example

@noindent
returns a two by two identity matrix.

For the binary operators, the dimensions of the operands must conform if
both are matrices.  If one of the operands is a scalar and the other a
matrix, the operator is applied to the scalar and each element of the
matrix.

For the binary element-by-element boolean operators, both subexpressions
@var{boolean1} and @var{boolean2} are evaluated before computing the
result.  This can make a difference when the expressions have side
effects.  For example, in the expression

@example
a & b++
@end example

@noindent
the value of the variable @var{b} is incremented even if the variable
@var{a} is zero.

This behavior is necessary for the boolean operators to work as
described for matrix-valued operands.

@node Short-circuit Boolean Operators,  , Element-by-element Boolean Operators, Boolean Expressions
@subsection Short-circuit Boolean Operators
@cindex short-circuit evaluation

Combined with the implicit conversion to scalar values in @code{if} and
@code{while} conditions, Octave's element-by-element boolean operators
are often sufficient for performing most logical operations.  However,
it is sometimes desirable to stop evaluating a boolean expression as
soon as the overall truth value can be determined.  Octave's
@dfn{short-circuit} boolean operators work this way.

@table @code
@item @var{boolean1} && @var{boolean2}
@opindex &&
The expression @var{boolean1} is evaluated and converted to a scalar
using the equivalent of the operation @code{all (all (@var{boolean1}))}.
If it is false, the result of the overall expression is 0.  If it is
true, the expression @var{boolean2} is evaluated and converted to a
scalar using the equivalent of the operation @code{all (all
(@var{boolean1}))}.  If it is true, the result of the overall expression
is 1.  Otherwise, the result of the overall expression is 0.

@item @var{boolean1} || @var{boolean2}
@opindex ||
The expression @var{boolean1} is evaluated and converted to a scalar
using the equivalent of the operation @code{all (all (@var{boolean1}))}.
If it is true, the result of the overall expression is 1.  If it is
false, the expression @var{boolean2} is evaluated and converted to a
scalar using the equivalent of the operation @code{all (all
(@var{boolean1}))}.  If it is true, the result of the overall expression
is 1.  Otherwise, the result of the overall expression is 0.
@end table

The fact that both operands may not be evaluated before determining the
overall truth value of the expression can be important.  For example, in
the expression

@example
a && b++
@end example

@noindent
the value of the variable @var{b} is only incremented if the variable
@var{a} is nonzero.

This can be used to write somewhat more concise code.  For example, it
is possible write

@example
@group
function f (a, b, c)
  if (nargin > 2 && isstr (c))
    @dots{}
@end group
@end example

@noindent
instead of having to use two @code{if} statements to avoid attempting to
evaluate an argument that doesn't exist.  For example, without the
short-circuit feature, it would be necessary to write

@example
@group
function f (a, b, c)
  if (nargin > 2)
    if (isstr (c))
      @dots{}
@end group
@end example

Writing

@example
@group
function f (a, b, c)
  if (nargin > 2 & isstr (c))
    @dots{}
@end group
@end example

@noindent
would result in an error if @code{f} were called with one or two
arguments because Octave would be forced to try to evaluate both of the
operands for the operator @samp{&}.

@node Assignment Ops, Increment Ops, Boolean Expressions, Expressions
@section Assignment Expressions
@cindex assignment expressions
@cindex assignment operators
@cindex operators, assignment
@cindex expressions, assignment

@opindex =

An @dfn{assignment} is an expression that stores a new value into a
variable.  For example, the following expression assigns the value 1 to
the variable @code{z}:

@example
z = 1
@end example

After this expression is executed, the variable @code{z} has the value 1.
Whatever old value @code{z} had before the assignment is forgotten.
The @samp{=} sign is called an @dfn{assignment operator}.

Assignments can store string values also.  For example, the following
expression would store the value @code{"this food is good"} in the
variable @code{message}:

@example
@group
thing = "food"
predicate = "good"
message = [ "this " , thing , " is " , predicate ]
@end group
@end example

@noindent
(This also illustrates concatenation of strings.)

@cindex side effect
Most operators (addition, concatenation, and so on) have no effect
except to compute a value.  If you ignore the value, you might as well
not use the operator.  An assignment operator is different.  It does
produce a value, but even if you ignore the value, the assignment still
makes itself felt through the alteration of the variable.  We call this
a @dfn{side effect}.

@cindex lvalue
The left-hand operand of an assignment need not be a variable
(@pxref{Variables}).  It can also be an element of a matrix
(@pxref{Index Expressions}) or a list of return values
(@pxref{Calling Functions}).  These are all called @dfn{lvalues}, which
means they can appear on the left-hand side of an assignment operator.
The right-hand operand may be any expression.  It produces the new value
which the assignment stores in the specified variable, matrix element,
or list of return values.

It is important to note that variables do @emph{not} have permanent types.
The type of a variable is simply the type of whatever value it happens
to hold at the moment.  In the following program fragment, the variable
@code{foo} has a numeric value at first, and a string value later on:

@example
@group
octave:13> foo = 1
foo = 1
octave:13> foo = "bar"
foo = bar
@end group
@end example

@noindent
When the second assignment gives @code{foo} a string value, the fact that
it previously had a numeric value is forgotten.

Assignment of a scalar to an indexed matrix sets all of the elements
that are referenced by the indices to the scalar value.  For example, if
@code{a} is a matrix with at least two columns,

@example
@group
a(:, 2) = 5
@end group
@end example

@noindent
sets all the elements in the second column of @code{a} to 5.

Assigning an empty matrix @samp{[]} works in most cases to allow you to
delete rows or columns of matrices and vectors.  @xref{Empty Matrices}.
For example, given a 4 by 5 matrix @var{A}, the assignment

@example
A (3, :) = []
@end example

@noindent
deletes the third row of @var{A}, and the assignment

@example
A (:, 1:2:5) = []
@end example

@noindent
deletes the first, third, and fifth columns.

An assignment is an expression, so it has a value.  Thus, @code{z = 1}
as an expression has the value 1.  One consequence of this is that you
can write multiple assignments together:

@example
x = y = z = 0
@end example

@noindent
stores the value 0 in all three variables.  It does this because the
value of @code{z = 0}, which is 0, is stored into @code{y}, and then
the value of @code{y = z = 0}, which is 0, is stored into @code{x}.

This is also true of assignments to lists of values, so the following is
a valid expression

@example
[a, b, c] = [u, s, v] = svd (a)
@end example

@noindent
that is exactly equivalent to

@example
@group
[u, s, v] = svd (a)
a = u
b = s
c = v
@end group
@end example

In expressions like this, the number of values in each part of the
expression need not match.  For example, the expression

@example
[a, b, c, d] = [u, s, v] = svd (a)
@end example

@noindent
is equivalent to the expression above, except that the value of the
variable @samp{d} is left unchanged, and the expression

@example
[a, b] = [u, s, v] = svd (a)
@end example

@noindent
is equivalent to 

@example
@group
[u, s, v] = svd (a)
a = u
b = s
@end group
@end example

You can use an assignment anywhere an expression is called for.  For
example, it is valid to write @code{x != (y = 1)} to set @code{y} to 1
and then test whether @code{x} equals 1.  But this style tends to make
programs hard to read.  Except in a one-shot program, you should rewrite
it to get rid of such nesting of assignments.  This is never very hard.

@cindex increment operator
@cindex decrement operator
@cindex operators, increment
@cindex operators, decrement

@node Increment Ops, Operator Precedence, Assignment Ops, Expressions
@section Increment Operators

@emph{Increment operators} increase or decrease the value of a variable
by 1.  The operator to increment a variable is written as @samp{++}.  It
may be used to increment a variable either before or after taking its
value.

For example, to pre-increment the variable @var{x}, you would write
@code{++@var{x}}.  This would add one to @var{x} and then return the new
value of @var{x} as the result of the expression.  It is exactly the
same as the expression @code{@var{x} = @var{x} + 1}.

To post-increment a variable @var{x}, you would write @code{@var{x}++}.
This adds one to the variable @var{x}, but returns the value that
@var{x} had prior to incrementing it.  For example, if @var{x} is equal
to 2, the result of the expression @code{@var{x}++} is 2, and the new
value of @var{x} is 3.

For matrix and vector arguments, the increment and decrement operators
work on each element of the operand.

Here is a list of all the increment and decrement expressions.

@table @code
@item ++@var{x}
@opindex ++
This expression increments the variable @var{x}.  The value of the
expression is the @emph{new} value of @var{x}.  It is equivalent to the
expression @code{@var{x} = @var{x} + 1}.

@item --@var{x}
@opindex @code{--}
This expression decrements the variable @var{x}.  The value of the
expression is the @emph{new} value of @var{x}.  It is equivalent to the
expression @code{@var{x} = @var{x} - 1}.

@item @var{x}++
@opindex ++
This expression causes the variable @var{x} to be incremented.  The
value of the expression is the @emph{old} value of @var{x}.

@item @var{x}--
@opindex @code{--}
This expression causes the variable @var{x} to be decremented.  The
value of the expression is the @emph{old} value of @var{x}.
@end table

It is not currently possible to increment index expressions.  For
example, you might expect that the expression @code{@var{v}(4)++} would
increment the fourth element of the vector @var{v}, but instead it
results in a parse error.  This problem may be fixed in a future
release of Octave.

@node Operator Precedence,  , Increment Ops, Expressions
@section Operator Precedence
@cindex operator precedence

@dfn{Operator precedence} determines how operators are grouped, when
different operators appear close by in one expression.  For example,
@samp{*} has higher precedence than @samp{+}.  Thus, the expression
@code{a + b * c} means to multiply @code{b} and @code{c}, and then add
@code{a} to the product (i.e., @code{a + (b * c)}).

You can overrule the precedence of the operators by using parentheses.
You can think of the precedence rules as saying where the parentheses
are assumed if you do not write parentheses yourself.  In fact, it is
wise to use parentheses whenever you have an unusual combination of
operators, because other people who read the program may not remember
what the precedence is in this case.  You might forget as well, and then
you too could make a mistake.  Explicit parentheses will help prevent
any such mistake.

When operators of equal precedence are used together, the leftmost
operator groups first, except for the assignment and exponentiation
operators, which group in the opposite order.  Thus, the expression
@code{a - b + c} groups as @code{(a - b) + c}, but the expression
@code{a = b = c} groups as @code{a = (b = c)}.

The precedence of prefix unary operators is important when another
operator follows the operand.  For example, @code{-x^2} means
@code{-(x^2)}, because @samp{-} has lower precedence than @samp{^}.

Here is a table of the operators in Octave, in order of increasing
precedence.

@table @code
@item statement separators
@samp{;}, @samp{,}.

@item assignment
@samp{=}.  This operator groups right to left.

@item logical "or" and "and"
@samp{||}, @samp{&&}.

@item element-wise "or" and "and"
@samp{|}, @samp{&}.

@item relational
@samp{<}, @samp{<=}, @samp{==}, @samp{>=}, @samp{>}, @samp{!=},
@samp{~=}, @samp{<>}.

@item colon
@samp{:}.

@item add, subtract
@samp{+}, @samp{-}.

@item multiply, divide
@samp{*}, @samp{/}, @samp{\}, @samp{.\}, @samp{.*}, @samp{./}.

@item transpose
@samp{'}, @samp{.'}

@item unary plus, minus, increment, decrement, and ``not''
@samp{+}, @samp{-}, @samp{++}, @samp{--}, @samp{!}, @samp{~}.

@item exponentiation
@samp{^}, @samp{**}, @samp{.^}, @samp{.**}.
@end table
