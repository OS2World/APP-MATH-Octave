@c Copyright (C) 1996, 1997 John W. Eaton
@c This is part of the Octave manual.
@c For copying conditions, see the file gpl.tex.

@node Statistics, Sets, Optimization, Top
@chapter Statistics

I hope that someday Octave will include more statistics functions.  If
you would like to help improve Octave in this area, please contact
@email{bug-octave@@bevo.che.wisc.edu}.

@deftypefn {Function File} {} mean (@var{x})
If @var{x} is a vector, compute the mean of the elements of @var{x}
@iftex
@tex
$$ {\rm mean}(x) = \bar{x} = {1\over N} \sum_{i=1}^N x_i $$
@end tex
@end iftex
@ifinfo

@example
mean (x) = SUM_i x(i) / N
@end example
@end ifinfo
If @var{x} is a matrix, compute the mean for each column and return them
in a row vector.
@end deftypefn

@deftypefn {Function File} {} median (@var{x})
If @var{x} is a vector, compute the median value of the elements of
@var{x}.
@iftex
@tex
$$
{\rm median} (x) =
  \cases{x(\lceil N/2\rceil), & $N$ odd;\cr
          (x(N/2)+x(N/2+1))/2, & $N$ even.}
$$
@end tex
@end iftex
@ifinfo

@example
@group
            x(ceil(N/2)),             N odd
median(x) = 
            (x(N/2) + x((N/2)+1))/2,  N even
@end group
@end example
@end ifinfo
If @var{x} is a matrix, compute the median value for each
column and return them in a row vector.
@end deftypefn

@deftypefn {Function File} {} std (@var{x})
If @var{x} is a vector, compute the standard deviation of the elements
of @var{x}.
@iftex
@tex
$$
{\rm std} (x) = \sigma (x) = \sqrt{{\sum_{i=1}^N (x_i - \bar{x}) \over N - 1}}
$$
@end tex
@end iftex
@ifinfo

@example
@group
std (x) = sqrt (sumsq (x - mean (x)) / (n - 1))
@end group
@end example
@end ifinfo
If @var{x} is a matrix, compute the standard deviation for
each column and return them in a row vector.
@end deftypefn

@deftypefn {Function File} {} cov (@var{x}, @var{y})
If each row of @var{x} and @var{y} is an observation and each column is
a variable, the (@var{i},@var{j})-th entry of
@code{cov (@var{x}, @var{y})} is the covariance between the @var{i}-th
variable in @var{x} and the @var{j}-th variable in @var{y}.  If called
with one argument, compute @code{cov (@var{x}, @var{x})}.
@end deftypefn

@deftypefn {Function File} {} corrcoef (@var{x}, @var{y})
If each row of @var{x} and @var{y} is an observation and each column is
a variable, the (@var{i},@var{j})-th entry of
@code{corrcoef (@var{x}, @var{y})} is the correlation between the
@var{i}-th variable in @var{x} and the @var{j}-th variable in @var{y}.
If called with one argument, compute @code{corrcoef (@var{x}, @var{x})}.
@end deftypefn

@deftypefn {Function File} {} kurtosis (@var{x})
If @var{x} is a vector of length @var{N}, return the kurtosis
@iftex
@tex
$$
 {\rm kurtosis} (x) = {1\over N \sigma(x)^4} \sum_{i=1}^N (x_i-\bar{x})^4 - 3
$$
@end tex
@end iftex
@ifinfo

@example
kurtosis (x) = N^(-1) std(x)^(-4) sum ((x - mean(x)).^4) - 3
@end example
@end ifinfo

@noindent
of @var{x}.  If @var{x} is a matrix, return the row vector containing
the kurtosis of each column.
@end deftypefn

@deftypefn {Function File} {} mahalanobis (@var{x}, @var{y})
Return the Mahalanobis' D-square distance between the multivariate
samples @var{x} and @var{y}, which must have the same number of
components (columns), but may have a different number of observations
(rows).
@end deftypefn

@deftypefn {Function File} {} skewness (@var{x})
If @var{x} is a vector of length @var{N}, return the skewness
@iftex
@tex
$$
{\rm skewness} (x) = {1\over N \sigma(x)^3} \sum_{i=1}^N (x_i-\bar{x})^3
$$
@end tex
@end iftex
@ifinfo

@example
skewness (x) = N^(-1) std(x)^(-3) sum ((x - mean(x)).^3)
@end example
@end ifinfo

@noindent
of @var{x}.  If @var{x} is a matrix, return the row vector containing
the skewness of each column.
@end deftypefn
