% Copyright (C) 1996, 1997 John W. Eaton
% This is part of the Octave manual.
% For copying conditions, see the file gpl.tex.
% Modified by Klaus Gebhardt, 1997

\input texinfo
@setfilename octave.

@ifinfo
@format
START-INFO-DIR-ENTRY
* Octave: (octave).	Interactive language for numerical computations.
END-INFO-DIR-ENTRY
@end format

@macro email{text}
(\text\)
@end macro

@macro url{text}
(\text\)
@end macro
@end ifinfo

@c Settings for printing on 8-1/2 by 11 inch paper:
@c -----------------------------------------------

@setchapternewpage odd

@c Settings for small book format:
@c ------------------------------

@ignore
@smallbook
@setchapternewpage odd
@finalout
@iftex
@cropmarks
@end iftex
@end ignore

@defindex op

@c Things like the Octave version number are defined in conf.tex.
@c This file doesn't include a chapter, so it must not be included
@c if you want to run the Emacs function texinfo-multiple-files-update.

@include ../conf.tex

@settitle GNU Octave

@ifinfo

Copyright (C) 1996, 1997 John W. Eaton.

Permission is granted to make and distribute verbatim copies of
this manual provided the copyright notice and this permission notice
are preserved on all copies.

@ignore
Permission is granted to process this file through Tex and print the
results, provided the printed document carries copying permission
notice identical to this one except for the removal of this paragraph
(this paragraph not being relevant to the printed manual).

@end ignore
Permission is granted to copy and distribute modified versions of
this manual under the conditions for verbatim copying, provided that
the entire resulting derived work is distributed under the terms of
a permission notice identical to this one.

Permission is granted to copy and distribute translations of this
manual into another language, under the above conditions for
modified versions.
@end ifinfo

@titlepage
@title GNU Octave
@subtitle A high-level interactive language for numerical computations
@subtitle Edition 3 for Octave version @value{VERSION}
@subtitle February 1997
@author John W. Eaton
@page
@vskip 0pt plus 1filll
Copyright @copyright{} 1996, 1997 John W. Eaton.

This is the third edition of the Octave documentation, and is consistent
with version @value{VERSION} of Octave.

Permission is granted to make and distribute verbatim copies of
this manual provided the copyright notice and this permission notice
are preserved on all copies.

Permission is granted to copy and distribute modified versions of this
manual under the conditions for verbatim copying, provided that the entire
resulting derived work is distributed under the terms of a permission
notice identical to this one.

Permission is granted to copy and distribute translations of this manual
into another language, under the same conditions as for modified versions.

Portions of this document have been adapted from the @code{gawk},
@code{readline}, @code{gcc}, and C library manuals, published by the Free
Software Foundation, 59 Temple Place---Suite 330, Boston, MA
02111--1307, USA.
@end titlepage

@ifinfo
@node Top, Preface, (dir), (dir)
@top

This manual documents how to run, install and port GNU Octave, as well
as its new features and incompatibilities, and how to report bugs.
It corresponds to GNU Octave version @value{VERSION}.
@end ifinfo

@menu
* Preface::                     
* Introduction::                A brief introduction to Octave.
* Getting Started::             
* Data Types::                  
* Numeric Data Types::          
* Strings::                     
* Data Structures::             
* Variables::                   
* Expressions::                 Expressions.
* Evaluation::                  
* Statements::                  Looping and program flow control.
* Functions and Scripts::       
* Error Handling::              
* Input and Output::            
* Plotting::                    
* Matrix Manipulation::         
* Arithmetic::                  
* Linear Algebra::              
* Nonlinear Equations::         
* Quadrature::                  
* Differential Equations::      
* Optimization::                
* Statistics::                  
* Sets::                        
* Polynomial Manipulations::    
* Control Theory::              
* Signal Processing::           
* Image Processing::            
* Audio Processing::            
* System Utilities::            
* Tips::                        
* Trouble::                     If you have trouble installing Octave.
* Installation::                How to configure, compile and install Octave.
* Emacs::                       
* Grammar::                     
* Copying::                     The GNU General Public License.
* Concept Index::               An item for each concept.
* Variable Index::              An item for each documented variable.
* Function Index::              An item for each documented function.
* Operator Index::              An item for each documented operator.


 --- The Detailed Node Listing ---

Preface

* Acknowledgements::            
* How You Can Contribute to Octave::  
* Distribution::                

A Brief Introduction to Octave

* Running Octave::              
* Simple Examples::             
* Conventions::                 

Conventions

* Fonts::                       
* Evaluation Notation::         
* Printing Notation::           
* Error Messages::              
* Format of Descriptions::      

Format of Descriptions

* A Sample Function Description::  
* A Sample Command Description::  
* A Sample Variable Description::  

Getting Started

* Invoking Octave::             
* Quitting Octave::             
* Getting Help::                
* Command Line Editing::        
* Errors::                      
* Executable Octave Programs::  
* Comments::                    

Invoking Octave

* Command Line Options::        
* Startup Files::               

Command Line Editing

* Cursor Motion::               
* Killing and Yanking::         
* Commands For Text::           
* Commands For Completion::     
* Commands For History::        
* Customizing the Prompt::      
* Diary and Echo Commands::     

Data Types

* Built-in Data Types::         
* User-defined Data Types::     
* Object Sizes::                

Built-in Data Types

* Numeric Objects::             
* String Objects::              
* Data Structure Objects::      

Numeric Data Types

* Matrices::                    
* Ranges::                      
* Predicates for Numeric Objects::  

Matrices

* Empty Matrices::              

Strings

* Creating Strings::            
* Searching and Replacing::     
* String Conversions::          
* Character Class Functions::   

Variables

* Global Variables::            
* Status of Variables::         
* Summary of Built-in Variables::  
* Defaults from the Environment::  

Expressions

* Index Expressions::           
* Calling Functions::           
* Arithmetic Ops::              
* Comparison Ops::              
* Boolean Expressions::         
* Assignment Ops::              
* Increment Ops::               
* Operator Precedence::         

Calling Functions

* Call by Value::               
* Recursion::                   

Boolean Expressions

* Element-by-element Boolean Operators::  
* Short-circuit Boolean Operators::  

Statements

* The if Statement::            
* The switch Statement::        
* The while Statement::         
* The for Statement::           
* The break Statement::         
* The continue Statement::      
* The unwind_protect Statement::  
* The try Statement::           
* Continuation Lines::          

The @code{for} Statement

* Looping Over Structure Elements::  

Functions and Script Files

* Defining Functions::          
* Multiple Return Values::      
* Variable-length Argument Lists::  
* Variable-length Return Lists::  
* Returning From a Function::   
* Function Files::              
* Script Files::                
* Dynamically Linked Functions::  
* Organization of Functions::   

Input and Output

* Basic Input and Output::      
* C-Style I/O Functions::       

Basic Input and Output

* Terminal Output::             
* Terminal Input::              
* Simple File I/O::             

C-Style I/O Functions

* Opening and Closing Files::   
* Simple Output::               
* Line-Oriented Input::         
* Formatted Output::            
* Output Conversion for Matrices::  
* Output Conversion Syntax::    
* Table of Output Conversions::  
* Integer Conversions::         
* Floating-Point Conversions::  Other Output Conversions::    
* Other Output Conversions::    
* Formatted Input::             
* Input Conversion Syntax::     
* Table of Input Conversions::  
* Numeric Input Conversions::   
* String Input Conversions::    
* Binary I/O::                  
* Temporary Files::             
* EOF and Errors::              
* File Positioning::            

Plotting

* Two-Dimensional Plotting::    
* Specialized Two-Dimensional Plots::  
* Three-Dimensional Plotting::  
* Plot Annotations::            
* Multiple Plots on One Page::  

Matrix Manipulation

* Finding Elements and Checking Conditions::  
* Rearranging Matrices::        
* Special Utility Matrices::    
* Famous Matrices::             

Arithmetic

* Utility Functions::           
* Complex Arithmetic::          
* Trigonometry::                
* Sums and Products::           
* Special Functions::           
* Mathematical Constants::      

Linear Algebra

* Basic Matrix Functions::      
* Matrix Factorizations::       
* Functions of a Matrix::       

Quadrature

* Functions of One Variable::   
* Orthogonal Collocation::      

Differential Equations

* Ordinary Differential Equations::  
* Differential-Algebraic Equations::  

Optimization

* Quadratic Programming::       
* Nonlinear Programming::       
* Linear Least Squares::        

System Utilities

* Timing Utilities::            
* Filesystem Utilities::        
* Controlling Subprocesses::    
* Process ID Information::      
* Environment Variables::       
* Current Working Directory::   
* Password Database Functions::  
* Group Database Functions::    
* System Information::          

Tips and Standards

* Style Tips::                  Writing clean and robust programs.
* Coding Tips::                 Making code run faster.
* Documentation Tips::          Writing readable documentation strings.
* Comment Tips::                Conventions for writing comments.
* Function Headers::            Standard headers for functions.

Known Causes of Trouble with Octave

* Actual Bugs::                 Bugs we will fix later.
* Reporting Bugs::              
* Bug Criteria::                
* Bug Lists::                   
* Bug Reporting::               
* Sending Patches::             
* Service::                     

Reporting Bugs

* Bug Criteria::                
* Where: Bug Lists.             Where to send your bug report.
* Reporting: Bug Reporting.     How to report a bug effectively.
* Patches: Sending Patches.     How to send a patch for Octave.

Installing Octave

* Notes::                       
* Installation Problems::       
* Binary Distributions::        

Binary Distributions

* Installing Octave from a Binary Distribution::  
* Creating a Binary Distribution::  

Emacs Octave Support

* Installing EOS::              
* Using Octave Mode::           
* Running Octave From Within Emacs::  
* Using the Emacs Info Reader for Octave::  

Grammar

* Keywords::                    
@end menu

@include preface.tex
@include intro.tex
@include basics.tex
@include data.tex
@include numbers.tex
@include strings.tex
@include struct.tex
@include var.tex
@include expr.tex
@include eval.tex
@include stmt.tex
@include func.tex
@include errors.tex
@include io.tex
@include plot.tex
@include matrix.tex
@include arith.tex
@include linalg.tex
@include nonlin.tex
@include quad.tex
@include diffeq.tex
@include optim.tex
@include stats.tex
@include set.tex
@include poly.tex
@include control.tex
@include signal.tex
@include image.tex
@include audio.tex
@include system.tex

@c maybe add again later, if anyone every writes any really interesting
@c fun stuff for Octave.
@c
@c @include amuse.tex

@c Appendices start here.  Installation and bugs have to go before the
@c readline and Info appendices because we want to have separate indices
@c for them, and there appears to be no way to go back to the original
@c set of indices once a redirection has taken place.

@include tips.tex
@include bugs.tex
@include install.tex
@include emacs.tex
@include grammar.tex
@include gpl.tex

@include cp-idx.tex
@include vr-idx.tex
@include fn-idx.tex
@include op-idx.tex

@contents

@bye
