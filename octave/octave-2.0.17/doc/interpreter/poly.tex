@c Copyright (C) 1996, 1997 John W. Eaton
@c This is part of the Octave manual.
@c For copying conditions, see the file gpl.tex.

@node Polynomial Manipulations, Control Theory, Sets, Top
@chapter Polynomial Manipulations

In Octave, a polynomial is represented by its coefficients (arranged
in descending order).  For example, a vector
@iftex
 $c$
@end iftex
@ifinfo
 @var{c}
@end ifinfo
of length
@iftex
@tex
 $N+1$
@end tex
@end iftex
@ifinfo
 @var{N+1}
@end ifinfo
 corresponds to the following polynomial of order
@iftex
@tex
 $N$
$$
 p (x) = c_1 x^N + ... + c_N x + c_{N+1}.
$$
@end tex
@end iftex
@ifinfo
 @var{N}

@example
p(x) = @var{c}(1) x^@var{N} + ... + @var{c}(@var{N}) x + @var{c}(@var{N}+1).
@end example
@end ifinfo

@deftypefn {Function File} {} compan (@var{c})
Compute the companion matrix corresponding to polynomial coefficient
vector @var{c}.

The companion matrix is
@iftex
@tex
$$
A = \left[\matrix{
 -c_2/c_1 & -c_3/c_1 & \cdots & -c_N/c_1 & -c_{N+1}/c_1\cr  
     1    &     0    & \cdots &     0    &         0   \cr
     0    &     1    & \cdots &     0    &         0   \cr
  \vdots  &   \vdots & \ddots &  \vdots  &      \vdots \cr
     0    &     0    & \cdots &     1    &         0}\right].
$$
@end tex
@end iftex
@ifinfo

@smallexample
     _                                                        _
    |  -c(2)/c(1)   -c(3)/c(1)  ...  -c(N)/c(1)  -c(N+1)/c(1)  |
    |       1            0      ...       0             0      |
    |       0            1      ...       0             0      |
A = |       .            .   .            .             .      |
    |       .            .       .        .             .      |
    |       .            .           .    .             .      |
    |_      0            0      ...       1             0     _|
@end smallexample
@end ifinfo

The eigenvalues of the companion matrix are equal to the roots of the
polynomial.
@end deftypefn

@deftypefn {Function File} {} conv (@var{a}, @var{b})
Convolve two vectors.

@code{y = conv (a, b)} returns a vector of length equal to
@code{length (a) + length (b) - 1}.
If @var{a} and @var{b} are polynomial coefficient vectors, @code{conv}
returns the coefficients of the product polynomial.
@end deftypefn

@deftypefn {Function File} {} deconv (@var{y}, @var{a})
Deconvolve two vectors.

@code{[b, r] = deconv (y, a)} solves for @var{b} and @var{r} such that
@code{y = conv (a, b) + r}.

If @var{y} and @var{a} are polynomial coefficient vectors, @var{b} will
contain the coefficients of the polynomial quotient and @var{r} will be
a remander polynomial of lowest order.
@end deftypefn

@deftypefn {Function File} {} poly (@var{a})
If @var{a} is a square @var{N}-by-@var{N} matrix, @code{poly (@var{a})}
is the row vector of the coefficients of @code{det (z * eye (N) - a)},
the characteristic polynomial of @var{a}.  If @var{x} is a vector,
@code{poly (@var{x})} is a vector of coefficients of the polynomial
whose roots are the elements of @var{x}.
@end deftypefn

@deftypefn {Function File} {} polyderiv (@var{c})
Return the coefficients of the derivative of the polynomial whose
coefficients are given by vector @var{c}.
@end deftypefn

@deftypefn {Function File} {[@var{p}, @var{yf}] =} polyfit (@var{x}, @var{y}, @var{n})
Return the coefficients of a polynomial @var{p}(@var{x}) of degree
@var{n} that minimizes 
@iftex
@tex
$$
\sum_{i=1}^N (p(x_i) - y_i)^2
$$
@end tex
@end iftex
@ifinfo
@code{sumsq (p(x(i)) - y(i))},
@end ifinfo
 to best fit the data in the least squares sense.
@end deftypefn

If two output arguments are requested, the second contains the values of
the polynomial for each value of @var{x}.

@deftypefn {Function File} {} polyinteg (@var{c})
Return the coefficients of the integral of the polynomial whose
coefficients are represented by the vector @var{c}.

The constant of integration is set to zero.
@end deftypefn

@deftypefn {Function File} {} polyreduce (@var{c})
Reduces a polynomial coefficient vector to a minimum number of terms by
stripping off any leading zeros.
@end deftypefn

@deftypefn {Function File} {} polyval (@var{c}, @var{x})
Evaluate a polynomial.

@code{polyval (@var{c}, @var{x})} will evaluate the polynomial at the
specified value of @var{x}.

If @var{x} is a vector or matrix, the polynomial is evaluated at each of
the elements of @var{x}.
@end deftypefn

@deftypefn {Function File} {} polyvalm (@var{c}, @var{x})
Evaluate a polynomial in the matrix sense.

@code{polyvalm (@var{c}, @var{x})} will evaluate the polynomial in the
matrix sense, i.e. matrix multiplication is used instead of element by
element multiplication as is used in polyval.

The argument @var{x} must be a square matrix.
@end deftypefn

@deftypefn {Function File} {} residue (@var{b}, @var{a}, @var{tol})
If @var{b} and @var{a} are vectors of polynomial coefficients, then
residue calculates the partial fraction expansion corresponding to the
ratio of the two polynomials.
@cindex partial fraction expansion

The function @code{residue} returns @var{r}, @var{p}, @var{k}, and
@var{e}, where the vector @var{r} contains the residue terms, @var{p}
contains the pole values, @var{k} contains the coefficients of a direct
polynomial term (if it exists) and @var{e} is a vector containing the
powers of the denominators in the partial fraction terms.

Assuming @var{b} and @var{a} represent polynomials
@iftex
@tex
$P(s)$ and $Q(s)$
@end tex
@end iftex
@ifinfo
 P (s) and Q(s)
@end ifinfo
 we have:
@iftex
@tex
$$
{P(s)\over Q(s)} = \sum_{m=1}^M {r_m\over (s-p_m)^e_m}
  + \sum_{i=1}^N k_i s^{N-i}.
$$
@end tex
@end iftex
@ifinfo

@example
 P(s)    M       r(m)         N
 ---- = SUM -------------  + SUM k(i)*s^(N-i)
 Q(s)   m=1 (s-p(m))^e(m)    i=1
@end example
@end ifinfo

@noindent
where @var{M} is the number of poles (the length of the @var{r},
@var{p}, and @var{e} vectors) and @var{N} is the length of the @var{k}
vector.

The argument @var{tol} is optional, and if not specified, a default
value of 0.001 is assumed.  The tolerance value is used to determine
whether poles with small imaginary components are declared real.  It is
also used to determine if two poles are distinct.  If the ratio of the
imaginary part of a pole to the real part is less than @var{tol}, the
imaginary part is discarded.  If two poles are farther apart than
@var{tol} they are distinct.  For example,

@example
@group
 b = [1, 1, 1];
 a = [1, -5, 8, -4];
 [r, p, k, e] = residue (b, a);
     @result{} r = [-2, 7, 3]
     @result{} p = [2, 2, 1]
     @result{} k = [](0x0)
     @result{} e = [1, 2, 1]
@end group
@end example

@noindent
which implies the following partial fraction expansion
@iftex
@tex
$$
{s^2+s+1\over s^3-5s^2+8s-4} = {-2\over s-2} + {7\over (s-2)^2} + {3\over s-1}
$$
@end tex
@end iftex
@ifinfo

@example
        s^2 + s + 1       -2        7        3
   ------------------- = ----- + ------- + -----
   s^3 - 5s^2 + 8s - 4   (s-2)   (s-2)^2   (s-1)
@end example
@end ifinfo
@end deftypefn

@deftypefn {Function File} {} roots (@var{v})

For a vector @var{v} with @var{N} components, return
the roots of the polynomial
@iftex
@tex
$$
v_1 z^{N-1} + \cdots + v_{N-1} z + v_N.
$$
@end tex
@end iftex
@ifinfo

@example
v(1) * z^(N-1) + ... + v(N-1) * z + v(N).
@end example
@end ifinfo
@end deftypefn
