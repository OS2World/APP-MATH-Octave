@c Copyright (C) 1996, 1997 John W. Eaton
@c This is part of the Octave manual.
@c For copying conditions, see the file gpl.tex.

@node Tips, Trouble, System Utilities, Top
@appendix Tips and Standards
@cindex tips
@cindex standards of coding style
@cindex coding standards

This chapter describes no additional features of Octave.  Instead it
gives advice on making effective use of the features described in the
previous chapters.

@menu
* Style Tips::                  Writing clean and robust programs.
* Coding Tips::                 Making code run faster.
* Documentation Tips::          Writing readable documentation strings.
* Comment Tips::                Conventions for writing comments.
* Function Headers::            Standard headers for functions.
@end menu

@node Style Tips, Coding Tips, Tips, Tips
@section Writing Clean Octave Programs

Here are some tips for avoiding common errors in writing Octave code
intended for widespread use:

@itemize @bullet
@item
Since all global variables share the same name space, and all functions
share another name space, you should choose a short word to distinguish
your program from other Octave programs.  Then take care to begin the
names of all global variables, constants, and functions with the chosen
prefix.  This helps avoid name conflicts.

If you write a function that you think ought to be added to Octave under
a certain name, such as @code{fiddle_matrix}, don't call it by that name
in your program.  Call it @code{mylib_fiddle_matrix} in your program,
and send mail to @email{bug-octave@@bevo.che.wisc.edu} suggesting that it
be added to Octave.  If and when it is, the name can be changed easily
enough.

If one prefix is insufficient, your package may use two or three
alternative common prefixes, so long as they make sense.

Separate the prefix from the rest of the symbol name with an underscore
@samp{_}.  This will be consistent with Octave itself and with most
Octave programs.

@item
When you encounter an error condition, call the function @code{error}
(or @code{usage}).  The @code{error} and @code{usage} functions do not
return.
@xref{Errors}.

@item
Please put a copyright notice on the file if you give copies to anyone.
Use the same lines that appear at the top of the function files
distributed with Octave.  If you have not signed papers to assign the
copyright to anyone else, then place your name in the copyright notice.
@end itemize

@node Coding Tips, Documentation Tips, Style Tips, Tips
@section Tips for Making Code Run Faster.
@cindex execution speed
@cindex speedups

Here are some ways of improving the execution speed of Octave programs.

@itemize @bullet
@item
Avoid looping wherever possible.

@item
Use iteration rather than recursion whenever possible.
Function calls are slow in Octave.

@item
Avoid resizing matrices unnecessarily.  When building a single result
matrix from a series of calculations, set the size of the result matrix
first, then insert values into it.  Write

@example
@group
result = zeros (big_n, big_m)
for i = over:and_over
  r1 = @dots{}
  r2 = @dots{}
  result (r1, r2) = new_value ();
endfor
@end group
@end example

@noindent
instead of

@example
@group
result = [];
for i = ever:and_ever
  result = [ result, new_value() ];
endfor
@end group
@end example

@item
Avoid calling @code{eval} or @code{feval} whenever possible, because
they require Octave to parse input or look up the name of a function in
the symbol table.

If you are using @code{eval} as an exception handling mechanism and not
because you need to execute some arbitrary text, use the @code{try}
statement instead.  @xref{The try Statement}.

@item
If you are calling lots of functions but none of them will need to
change during your run, set the variable
@code{ignore_function_time_stamp} to @code{"all"} so that Octave doesn't
waste a lot of time checking to see if you have updated your function
files.
@end itemize

@node Documentation Tips, Comment Tips, Coding Tips, Tips
@section Tips for Documentation Strings

Here are some tips for the writing of documentation strings.

@itemize @bullet
@item
Every command, function, or variable intended for users to know about
should have a documentation string.

@item
An internal variable or subroutine of an Octave program might as well have
a documentation string.

@item
The first line of the documentation string should consist of one or two
complete sentences that stand on their own as a summary.

The documentation string can have additional lines that expand on the
details of how to use the function or variable.  The additional lines
should also be made up of complete sentences.

@item
For consistency, phrase the verb in the first sentence of a
documentation string as an infinitive with ``to'' omitted.  For
instance, use ``Return the frob of A and B.'' in preference to ``Returns
the frob of A and B@.''  Usually it looks good to do likewise for the
rest of the first paragraph.  Subsequent paragraphs usually look better
if they have proper subjects.

@item
Write documentation strings in the active voice, not the passive, and in
the present tense, not the future.  For instance, use ``Return a list
containing A and B.'' instead of ``A list containing A and B will be
returned.''

@item
Avoid using the word ``cause'' (or its equivalents) unnecessarily.
Instead of, ``Cause Octave to display text in boldface,'' write just
``Display text in boldface.''

@item
Do not start or end a documentation string with whitespace.

@item
Format the documentation string so that it fits in an Emacs window on an
80-column screen.  It is a good idea for most lines to be no wider than
60 characters.

However, rather than simply filling the entire documentation string, you
can make it much more readable by choosing line breaks with care.
Use blank lines between topics if the documentation string is long.
 
@item
@strong{Do not} indent subsequent lines of a documentation string so
that the text is lined up in the source code with the text of the first
line.  This looks nice in the source code, but looks bizarre when users
view the documentation.  Remember that the indentation before the
starting double-quote is not part of the string!

@item
The documentation string for a variable that is a yes-or-no flag should
start with words such as ``Nonzero means@dots{}'', to make it clear that
all nonzero values are equivalent and indicate explicitly what zero and
nonzero mean.

@item
When a function's documentation string mentions the value of an argument
of the function, use the argument name in capital letters as if it were
a name for that value.  Thus, the documentation string of the operator
@code{/} refers to its second argument as @samp{DIVISOR}, because the
actual argument name is @code{divisor}.

Also use all caps for meta-syntactic variables, such as when you show
the decomposition of a list or vector into subunits, some of which may
vary.
@end itemize

@node Comment Tips, Function Headers, Documentation Tips, Tips
@section Tips on Writing Comments

Here are the conventions to follow when writing comments.

@table @samp
@item #
Comments that start with a single sharp-sign, @samp{#}, should all be
aligned to the same column on the right of the source code.  Such
comments usually explain how the code on the same line does its job.  In
the Emacs mode for Octave, the @kbd{M-;} (@code{indent-for-comment})
command automatically inserts such a @samp{#} in the right place, or
aligns such a comment if it is already present.

@item ##
Comments that start with two semicolons, @samp{##}, should be aligned to
the same level of indentation as the code.  Such comments usually
describe the purpose of the following lines or the state of the program
at that point.
@end table

@noindent
The indentation commands of the Octave mode in Emacs, such as @kbd{M-;}
(@code{indent-for-comment}) and @kbd{TAB} (@code{octave-indent-line})
automatically indent comments according to these conventions,
depending on the number of semicolons.  @xref{Comments,,
Manipulating Comments, emacs, The GNU Emacs Manual}.

@node Function Headers,  , Comment Tips, Tips
@section Conventional Headers for Octave Functions
@cindex header comments

Octave has conventions for using special comments in function files
to give information such as who wrote them.  This section explains these
conventions.

The top of the file should contain a copyright notice, followed by a
block of comments that can be used as the help text for the function.
Here is an example:

@example
## Copyright (C) 1996, 1997 John W. Eaton
##
## This file is part of Octave.
##
## Octave is free software; you can redistribute it and/or
## modify it under the terms of the GNU General Public
## License as published by the Free Software Foundation;
## either version 2, or (at your option) any later version.
##
## Octave is distributed in the hope that it will be useful,
## but WITHOUT ANY WARRANTY; without even the implied
## warranty of MERCHANTABILITY or FITNESS FOR A PARTICULAR
## PURPOSE.  See the GNU General Public License for more
## details.
##
## You should have received a copy of the GNU General Public
## License along with Octave; see the file COPYING.  If not,
## write to the Free Software Foundation, 59 Temple Place -
## Suite 330, Boston, MA 02111-1307, USA.

## usage: [IN, OUT, PID] = popen2 (COMMAND, ARGS)
##
## Start a subprocess with two-way communication.  COMMAND
## specifies the name of the command to start.  ARGS is an
## array of strings containing options for COMMAND.  IN and
## OUT are the file ids of the input and streams for the
## subprocess, and PID is the process id of the subprocess,
## or -1 if COMMAND could not be executed.
##
## Example:
##
##  [in, out, pid] = popen2 ("sort", "-nr");
##  fputs (in, "these\nare\nsome\nstrings\n");
##  fclose (in);
##  while (isstr (s = fgets (out)))
##    fputs (stdout, s);
##  endwhile
##  fclose (out);
@end example

Octave uses the first block of comments in a function file that do not
appear to be a copyright notice as the help text for the file.  For
Octave to recognize the first comment block as a copyright notice, it
must match the regular expression

@example
^ Copyright (C).*\n\n This file is part of Octave.
@end example

@noindent
or

@example
^ Copyright (C).*\n\n This program is free softwar
@end example

@noindent
(after stripping the leading comment characters).  This is a fairly
strict requirement, and may be relaxed somewhat in the future.

After the copyright notice and help text come several @dfn{header
comment} lines, each beginning with @samp{## @var{header-name}:}.  For
example,

@example
@group
## Author: jwe
## Keywords: subprocesses input-output
## Maintainer: jwe
@end group
@end example

Here is a table of the conventional possibilities for @var{header-name}:

@table @samp
@item Author
This line states the name and net address of at least the principal
author of the library.

@smallexample
## Author: John W. Eaton <jwe@@bevo.che.wisc.edu>
@end smallexample

@item Maintainer
This line should contain a single name/address as in the Author line, or
an address only, or the string @samp{jwe}.  If there is no maintainer
line, the person(s) in the Author field are presumed to be the
maintainers.  The example above is mildly bogus because the maintainer
line is redundant.

The idea behind the @samp{Author} and @samp{Maintainer} lines is to make
possible a function to ``send mail to the maintainer'' without
having to mine the name out by hand.

Be sure to surround the network address with @samp{<@dots{}>} if
you include the person's full name as well as the network address.

@item Created
This optional line gives the original creation date of the
file.  For historical interest only.

@item Version
If you wish to record version numbers for the individual Octave program,
put them in this line.

@item Adapted-By
In this header line, place the name of the person who adapted the
library for installation (to make it fit the style conventions, for
example).

@item Keywords
This line lists keywords.  Eventually, it will be used by an apropos
command to allow people will find your package when they're looking for
things by topic area.  To separate the keywords, you can use spaces,
commas, or both.
@end table

Just about every Octave function ought to have the @samp{Author} and
@samp{Keywords} header comment lines.  Use the others if they are
appropriate.  You can also put in header lines with other header
names---they have no standard meanings, so they can't do any harm.
