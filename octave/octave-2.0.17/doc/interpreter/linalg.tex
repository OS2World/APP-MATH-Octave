@c Copyright (C) 1996, 1997 John W. Eaton
@c This is part of the Octave manual.
@c For copying conditions, see the file gpl.tex.

@node Linear Algebra, Nonlinear Equations, Arithmetic, Top
@chapter Linear Algebra

This chapter documents the linear algebra functions of Octave.
Reference material for many of these functions may be found in
Golub and Van Loan, @cite{Matrix Computations, 2nd Ed.}, Johns Hopkins,
1989, and in @cite{@sc{Lapack} Users' Guide}, SIAM, 1992.

@menu
* Basic Matrix Functions::      
* Matrix Factorizations::       
* Functions of a Matrix::       
@end menu

@node Basic Matrix Functions, Matrix Factorizations, Linear Algebra, Linear Algebra
@section Basic Matrix Functions

@deftypefn {Loadable Function} {@var{aa} =} balance (@var{a}, @var{opt})
@deftypefnx {Loadable Function} {[@var{dd}, @var{aa}] =} balance (@var{a}, @var{opt})
@deftypefnx {Loadable Function} {[@var{cc}, @var{dd}, @var{aa}, @var{bb]} =} balance (@var{a}, @var{b}, @var{opt})

@code{[dd, aa] = balance (a)} returns @code{aa = dd \ a * dd}.
@code{aa} is a matrix whose row and column norms are roughly equal in
magnitude, and @code{dd} = @code{p * d}, where @code{p} is a permutation
matrix and @code{d} is a diagonal matrix of powers of two.  This allows
the equilibration to be computed without roundoff.  Results of
eigenvalue calculation are typically improved by balancing first.

@code{[cc, dd, aa, bb] = balance (a, b)} returns @code{aa = cc*a*dd} and
@code{bb = cc*b*dd)}, where @code{aa} and @code{bb} have non-zero
elements of approximately the same magnitude and @code{cc} and @code{dd}
are permuted diagonal matrices as in @code{dd} for the algebraic
eigenvalue problem.

The eigenvalue balancing option @code{opt} is selected as follows:

@table @asis
@item @code{"N"}, @code{"n"}
No balancing; arguments copied, transformation(s) set to identity.

@item @code{"P"}, @code{"p"}
Permute argument(s) to isolate eigenvalues where possible.

@item @code{"S"}, @code{"s"}
Scale to improve accuracy of computed eigenvalues.

@item @code{"B"}, @code{"b"}
Permute and scale, in that order. Rows/columns of a (and b)
that are isolated by permutation are not scaled.  This is the default
behavior.
@end table

Algebraic eigenvalue balancing uses standard @sc{Lapack} routines.

Generalized eigenvalue problem balancing uses Ward's algorithm
(SIAM Journal on Scientific and Statistical Computing, 1981).
@end deftypefn

@deftypefn {} {} cond (@var{a})
Compute the (two-norm) condition number of a matrix. @code{cond (a)} is
defined as @code{norm (a) * norm (inv (a))}, and is computed via a
singular value decomposition.
@end deftypefn

@deftypefn {Loadable Function} {} det (@var{a})
Compute the determinant of @var{a} using @sc{Linpack}.
@end deftypefn

@deftypefn {Loadable Function} {@var{lambda} =} eig (@var{a})
@deftypefnx {Loadable Function} {[@var{v}, @var{lambda}] =} eig (@var{a})
The eigenvalues (and eigenvectors) of a matrix are computed in a several
step process which begins with a Hessenberg decomposition, followed by a
Schur decomposition, from which the eigenvalues are apparent.  The
eigenvectors, when desired, are computed by further manipulations of the
Schur decomposition.
@end deftypefn

@deftypefn {Loadable Function} {@var{G} =} givens (@var{x}, @var{y})
@deftypefnx {Loadable Function} {[@var{c}, @var{s}] =} givens (@var{x}, @var{y})
@iftex
@tex
Return a $2\times 2$ orthogonal matrix
$$
 G = \left[\matrix{c & s\cr -s'& c\cr}\right]
$$
such that
$$
 G \left[\matrix{x\cr y}\right] = \left[\matrix{\ast\cr 0}\right]
$$
with $x$ and $y$ scalars.
@end tex
@end iftex
@ifinfo
Return a 2 by 2 orthogonal matrix
@code{@var{G} = [@var{c} @var{s}; -@var{s}' @var{c}]} such that
@code{@var{G} [@var{x}; @var{y}] = [*; 0]} with @var{x} and @var{y} scalars.
@end ifinfo

For example,

@example
@group
givens (1, 1)
     @result{}   0.70711   0.70711
         -0.70711   0.70711
@end group
@end example
@end deftypefn

@deftypefn {Loadable Function} {} inv (@var{a})
@deftypefnx {Loadable Function} {} inverse (@var{a})
Compute the inverse of the square matrix @var{a}.
@end deftypefn

@deftypefn {Function File} {} norm (@var{a}, @var{p})
Compute the p-norm of the matrix @var{a}.  If the second argument is
missing, @code{p = 2} is assumed.

If @var{a} is a matrix:

@table @asis
@item @var{p} = @code{1}
1-norm, the largest column sum of @var{a}.

@item @var{p} = @code{2}
Largest singular value of @var{a}.

@item @var{p} = @code{Inf}
@cindex infinity norm
Infinity norm, the largest row sum of @var{a}.

@item @var{p} = @code{"fro"}
@cindex Frobenius norm
Frobenius norm of @var{a}, @code{sqrt (sum (diag (@var{a}' * @var{a})))}.
@end table

If @var{a} is a vector or a scalar:

@table @asis
@item @var{p} = @code{Inf}
@code{max (abs (@var{a}))}.

@item @var{p} = @code{-Inf}
@code{min (abs (@var{a}))}.

@item other
p-norm of @var{a}, @code{(sum (abs (@var{a}) .^ @var{p})) ^ (1/@var{p})}.
@end table
@end deftypefn

@deftypefn {Function File} {} null (@var{a}, @var{tol})
Return an orthonormal basis of the null space of @var{a}.

The dimension of the null space is taken as the number of singular
values of @var{a} not greater than @var{tol}.  If the argument @var{tol}
is missing, it is computed as

@example
max (size (@var{a})) * max (svd (@var{a})) * eps
@end example
@end deftypefn

@deftypefn {Function File} {} orth (@var{a}, @var{tol})
Return an orthonormal basis of the range space of @var{a}.

The dimension of the range space is taken as the number of singular
values of @var{a} greater than @var{tol}.  If the argument @var{tol} is
missing, it is computed as

@example
max (size (@var{a})) * max (svd (@var{a})) * eps
@end example
@end deftypefn

@deftypefn {Function File} {} pinv (@var{x}, @var{tol})
Return the pseudoinverse of @var{x}.  Singular values less than
@var{tol} are ignored. 

If the second argument is omitted, it is assumed that

@example
tol = max (size (@var{x})) * sigma_max (@var{x}) * eps,
@end example

@noindent
where @code{sigma_max (@var{x})} is the maximal singular value of @var{x}.
@end deftypefn

@deftypefn {Function File} {} rank (@var{a}, @var{tol})
Compute the rank of @var{a}, using the singular value decomposition.
The rank is taken to be the number  of singular values of @var{a} that
are greater than the specified tolerance @var{tol}.  If the second
argument is omitted, it is taken to be

@example
tol = max (size (@var{a})) * sigma (1) * eps;
@end example

@noindent
where @code{eps} is machine precision and @code{sigma} is the largest
singular value of @var{a}.
@end deftypefn

@deftypefn {Function File} {} trace (@var{a})
Compute the trace of @var{a}, @code{sum (diag (@var{a}))}.
@end deftypefn

@node Matrix Factorizations, Functions of a Matrix, Basic Matrix Functions, Linear Algebra
@section Matrix Factorizations

@deftypefn {Loadable Function} {} chol (@var{a})
@cindex Cholesky factorization
Compute the Cholesky factor, @var{r}, of the symmetric positive definite
matrix @var{a}, where
@iftex
@tex
$ R^T R = A $.
@end tex
@end iftex
@ifinfo

@example
r' * r = a.
@end example
@end ifinfo
@end deftypefn

@deftypefn {Loadable Function} {@var{h} =} hess (@var{a})
@deftypefnx {Loadable Function} {[@var{p}, @var{h}] =} hess (@var{a})
@cindex Hessenberg decomposition
Compute the Hessenberg decomposition of the matrix @var{a}.

The Hessenberg decomposition is usually used as the first step in an
eigenvalue computation, but has other applications as well (see Golub,
Nash, and Van Loan, IEEE Transactions on Automatic Control, 1979.  The
Hessenberg decomposition is
@iftex
@tex
$$
A = PHP^T
$$
where $P$ is a square unitary matrix ($P^HP = I$), and $H$
is upper Hessenberg ($H_{i,j} = 0, \forall i \ge j+1$).
@end tex
@end iftex
@ifinfo
@code{p * h * p' = a} where @code{p} is a square unitary matrix
(@code{p' * p = I}, using complex-conjugate transposition) and @code{h}
is upper Hessenberg (@code{i >= j+1 => h (i, j) = 0}).
@end ifinfo
@end deftypefn

@deftypefn {Loadable Function} {[@var{l}, @var{u}, @var{p}] =} lu (@var{a})
@cindex LU decomposition
Compute the LU decomposition of @var{a}, using subroutines from
@sc{Lapack}.  The result is returned in a permuted form, according to
the optional return value @var{p}.  For example, given the matrix
@code{a = [1, 2; 3, 4]},

@example
[l, u, p] = lu (a)
@end example

@noindent
returns

@example
l =

  1.00000  0.00000
  0.33333  1.00000

u =

  3.00000  4.00000
  0.00000  0.66667

p =

  0  1
  1  0
@end example
@end deftypefn

@deftypefn {Loadable Function} {[@var{q}, @var{r}, @var{p}] =} qr (@var{a})
@cindex QR factorization
Compute the QR factorization of @var{a}, using standard @sc{Lapack}
subroutines.  For example, given the matrix @code{a = [1, 2; 3, 4]},

@example
[q, r] = qr (a)
@end example

@noindent
returns

@example
q =

  -0.31623  -0.94868
  -0.94868   0.31623

r =

  -3.16228  -4.42719
   0.00000  -0.63246
@end example

The @code{qr} factorization has applications in the solution of least
squares problems
@iftex
@tex
$$
\min_x \left\Vert A x - b \right\Vert_2
$$
@end tex
@end iftex
@ifinfo

@example
@code{min norm(A x - b)}
@end example

@end ifinfo
for overdetermined systems of equations (i.e.,
@iftex
@tex
$A$
@end tex
@end iftex
@ifinfo
@code{a}
@end ifinfo
 is a tall, thin matrix).  The QR factorization is
@iftex
@tex
$QR = A$ where $Q$ is an orthogonal matrix and $R$ is upper triangular.
@end tex
@end iftex
@ifinfo
@code{q * r = a} where @code{q} is an orthogonal matrix and @code{r} is
upper triangular.
@end ifinfo

The permuted QR factorization @code{[@var{q}, @var{r}, @var{p}] =
qr (@var{a})} forms the QR factorization such that the diagonal
entries of @code{r} are decreasing in magnitude order.  For example,
given the matrix @code{a = [1, 2; 3, 4]},

@example
[q, r, pi] = qr(a)
@end example

@noindent
returns

@example
q = 

  -0.44721  -0.89443
  -0.89443   0.44721

r =

  -4.47214  -3.13050
   0.00000   0.44721

p =

   0  1
   1  0
@end example

The permuted @code{qr} factorization @code{[q, r, p] = qr (a)}
factorization allows the construction of an orthogonal basis of
@code{span (a)}.
@end deftypefn

@deftypefn {Loadable Function} {@var{s} =} schur (@var{a})
@deftypefnx {Loadable Function} {[@var{u}, @var{s}] =} schur (@var{a}, @var{opt})
@cindex Schur decomposition
The Schur decomposition is used to compute eigenvalues of a
square matrix, and has applications in the solution of algebraic
Riccati equations in control (see @code{are} and @code{dare}).
@code{schur} always returns
@iftex
@tex
$S = U^T A U$
@end tex
@end iftex
@ifinfo
@code{s = u' * a * u}
@end ifinfo
where
@iftex
@tex
$U$
@end tex
@end iftex
@ifinfo
@code{u}
@end ifinfo
 is a unitary matrix
@iftex
@tex
($U^T U$ is identity)
@end tex
@end iftex
@ifinfo
(@code{u'* u} is identity)
@end ifinfo
and
@iftex
@tex
$S$
@end tex
@end iftex
@ifinfo
@code{s}
@end ifinfo
is upper triangular.  The eigenvalues of
@iftex
@tex
$A$ (and $S$)
@end tex
@end iftex
@ifinfo
@code{a} (and @code{s})
@end ifinfo
are the diagonal elements of
@iftex
@tex
$S$
@end tex
@end iftex
@ifinfo
@code{s}
@end ifinfo
If the matrix
@iftex
@tex
$A$
@end tex
@end iftex
@ifinfo
@code{a}
@end ifinfo
is real, then the real Schur decomposition is computed, in which the
matrix
@iftex
@tex
$U$
@end tex
@end iftex
@ifinfo
@code{u}
@end ifinfo
is orthogonal and
@iftex
@tex
$S$
@end tex
@end iftex
@ifinfo
@code{s}
@end ifinfo
is block upper triangular
with blocks of size at most
@iftex
@tex
$2\times 2$
@end tex
@end iftex
@ifinfo
@code{2 x 2}
@end ifinfo
blocks along the diagonal.  The diagonal elements of
@iftex
@tex
$S$
@end tex
@end iftex
@ifinfo
@code{s}
@end ifinfo
(or the eigenvalues of the
@iftex
@tex
$2\times 2$
@end tex
@end iftex
@ifinfo
@code{2 x 2}
@end ifinfo
blocks, when
appropriate) are the eigenvalues of
@iftex
@tex
$A$
@end tex
@end iftex
@ifinfo
@code{a}
@end ifinfo
and
@iftex
@tex
$S$.
@end tex
@end iftex
@ifinfo
@code{s}.
@end ifinfo

The eigenvalues are optionally ordered along the diagonal according to
the value of @code{opt}.  @code{opt = "a"} indicates that all
eigenvalues with negative real parts should be moved to the leading
block of
@iftex
@tex
$S$
@end tex
@end iftex
@ifinfo
@code{s}
@end ifinfo
(used in @code{are}), @code{opt = "d"} indicates that all eigenvalues
with magnitude less than one should be moved to the leading block of
@iftex
@tex
$S$
@end tex
@end iftex
@ifinfo
@code{s}
@end ifinfo
(used in @code{dare}), and @code{opt = "u"}, the default, indicates that
no ordering of eigenvalues should occur.  The leading
@iftex
@tex
$k$
@end tex
@end iftex
@ifinfo
@code{k}
@end ifinfo
columns of
@iftex
@tex
$U$
@end tex
@end iftex
@ifinfo
@code{u}
@end ifinfo
always span the
@iftex
@tex
$A$-invariant
@end tex
@end iftex
@ifinfo
@code{a}-invariant
@end ifinfo
subspace corresponding to the
@iftex
@tex
$k$
@end tex
@end iftex
@ifinfo
@code{k}
@end ifinfo
leading eigenvalues of
@iftex
@tex
$S$.
@end tex
@end iftex
@ifinfo
@code{s}.
@end ifinfo
@end deftypefn

@deftypefn {Loadable Function} {@var{s} =} svd (@var{a})
@deftypefnx {Loadable Function} {[@var{u}, @var{s}, @var{v}] =} svd (@var{a})
@cindex singular value decomposition
Compute the singular value decomposition of @var{a}
@iftex
@tex
$$
 A = U\Sigma V^H
$$
@end tex
@end iftex
@ifinfo

@example
a = u * sigma * v'
@end example
@end ifinfo

The function @code{svd} normally returns the vector of singular values.
If asked for three return values, it computes
@iftex
@tex
$U$, $S$, and $V$.
@end tex
@end iftex
@ifinfo
U, S, and V.
@end ifinfo
For example,

@example
svd (hilb (3))
@end example

@noindent
returns

@example
ans =

  1.4083189
  0.1223271
  0.0026873
@end example

@noindent
and

@example
[u, s, v] = svd (hilb (3))
@end example

@noindent
returns

@example
u =

  -0.82704   0.54745   0.12766
  -0.45986  -0.52829  -0.71375
  -0.32330  -0.64901   0.68867

s =

  1.40832  0.00000  0.00000
  0.00000  0.12233  0.00000
  0.00000  0.00000  0.00269

v =

  -0.82704   0.54745   0.12766
  -0.45986  -0.52829  -0.71375
  -0.32330  -0.64901   0.68867
@end example

If given a second argument, @code{svd} returns an economy-sized
decomposition, eliminating the unnecessary rows or columns of @var{u} or
@var{v}.
@end deftypefn

@node Functions of a Matrix,  , Matrix Factorizations, Linear Algebra
@section Functions of a Matrix

@deftypefn {Loadable Function} {} expm (@var{a})
Return the exponential of a matrix, defined as the infinite Taylor
series
@iftex
@tex
$$
 \exp (A) = I + A + {A^2 \over 2!} + {A^3 \over 3!} + \cdots
$$
@end tex
@end iftex
@ifinfo

@example
expm(a) = I + a + a^2/2! + a^3/3! + ...
@end example

@end ifinfo
The Taylor series is @emph{not} the way to compute the matrix
exponential; see Moler and Van Loan, @cite{Nineteen Dubious Ways to
Compute the Exponential of a Matrix}, SIAM Review, 1978.  This routine
uses Ward's diagonal
@iftex
@tex
Pad\'e
@end tex
@end iftex
@ifinfo
Pade'
@end ifinfo
approximation method with three step preconditioning (SIAM Journal on
Numerical Analysis, 1977).  Diagonal
@iftex
@tex
Pad\'e
@end tex
@end iftex
@ifinfo
Pade'
@end ifinfo
 approximations are rational polynomials of matrices
@iftex
@tex
$D_q(a)^{-1}N_q(a)$
@end tex
@end iftex
@ifinfo

@example
     -1
D (a)   N (a)
@end example

@end ifinfo
 whose Taylor series matches the first
@iftex
@tex
$2 q + 1 $
@end tex
@end iftex
@ifinfo
@code{2q+1}
@end ifinfo
terms of the Taylor series above; direct evaluation of the Taylor series
(with the same preconditioning steps) may be desirable in lieu of the
@iftex
@tex
Pad\'e
@end tex
@end iftex
@ifinfo
Pade'
@end ifinfo
approximation when
@iftex
@tex
$D_q(a)$
@end tex
@end iftex
@ifinfo
@code{Dq(a)}
@end ifinfo
is ill-conditioned.
@end deftypefn

@deftypefn {Loadable Function} {} logm (@var{a})
Compute the matrix logarithm of the square matrix @var{a}.  Note that
this is currently implemented in terms of an eigenvalue expansion and
needs to be improved to be more robust.
@end deftypefn

@deftypefn {Loadable Function} {} sqrtm (@var{a})
Compute the matrix square root of the square matrix @var{a}.  Note that
this is currently implemented in terms of an eigenvalue expansion and
needs to be improved to be more robust.
@end deftypefn

@deftypefn {Function File} {} kron (@var{a}, @var{b})
Form the kronecker product of two matrices, defined block by block as

@example
x = [a(i, j) b]
@end example

For example,

@example
@group
kron (1:4, ones (3, 1))
     @result{}  1  2  3  4
         1  2  3  4
         1  2  3  4
@end group
@end example
@end deftypefn

@deftypefn {Function File} {[@var{aa}, @var{bb}, @var{q}, @var{z}] =} qzhess (@var{a}, @var{b})
Compute the Hessenberg-triangular decomposition of the matrix pencil
@code{(@var{a}, @var{b})}, returning
@code{@var{aa} = @var{q} * @var{a} * @var{z}}, 
@code{@var{bb} = @var{q} * @var{b} * @var{z}}, with @var{q} and @var{z}
orthogonal.  For example,

@example
@group
[aa, bb, q, z] = qzhess ([1, 2; 3, 4], [5, 6; 7, 8])
     @result{} aa = [ -3.02244, -4.41741;  0.92998,  0.69749 ]
     @result{} bb = [ -8.60233, -9.99730;  0.00000, -0.23250 ]
     @result{}  q = [ -0.58124, -0.81373; -0.81373,  0.58124 ]
     @result{}  z = [ 1, 0; 0, 1 ]
@end group
@end example

The Hessenberg-triangular decomposition is the first step in
Moler and Stewart's QZ decomposition algorithm.

Algorithm taken from Golub and Van Loan, @cite{Matrix Computations, 2nd
edition}.
@end deftypefn

@deftypefn {Loadable Function} {} qzval (@var{a}, @var{b})
Compute generalized eigenvalues of the matrix pencil 
@iftex
@tex
$a - \lambda b$.
@end tex
@end iftex
@ifinfo
@code{@var{a} - lambda @var{b}}.
@end ifinfo

The arguments @var{a} and @var{b} must be real matrices.
@end deftypefn

@deftypefn {Loadable Function} {@var{x} =} syl (@var{a}, @var{b}, @var{c})
Solve the Sylvester equation
@iftex
@tex
$$
 A X + X B + C = 0
$$
@end tex
@end iftex
@ifinfo

@example
A X + X B + C = 0
@end example
@end ifinfo
using standard @sc{Lapack} subroutines.  For example,

@example
@group
syl ([1, 2; 3, 4], [5, 6; 7, 8], [9, 10; 11, 12])
     @result{} [ -0.50000, -0.66667; -0.66667, -0.50000 ]
@end group
@end example
@end deftypefn
