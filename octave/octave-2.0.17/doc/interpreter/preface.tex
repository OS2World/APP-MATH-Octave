@c Copyright (C) 1996, 1997 John W. Eaton
@c This is part of the Octave manual.
@c For copying conditions, see the file gpl.tex.

@node Preface, Introduction, Top, Top
@unnumbered Preface
@cindex contributors
@cindex history

Octave was originally intended to be companion software for an
undergraduate-level textbook on chemical reactor design being written by
James B. Rawlings of the University of Wisconsin-Madison and John
G. Ekerdt of the University of Texas.

Clearly, Octave is now much more than just another `courseware' package
with limited utility beyond the classroom.  Although our initial goals
were somewhat vague, we knew that we wanted to create something that
would enable students to solve realistic problems, and that they could
use for many things other than chemical reactor design problems.

There are those who would say that we should be teaching the students
Fortran instead, because that is the computer language of engineering,
but every time we have tried that, the students have spent far too much
time trying to figure out why their Fortran code crashes and not enough
time learning about chemical engineering.  With Octave, most students
pick up the basics quickly, and are using it confidently in just a few
hours.

Although it was originally intended to be used to teach reactor design,
it has been used in several other undergraduate and graduate
courses in the Chemical Engineering Department at the University of
Texas, and the math department at the University of Texas has been using
it for teaching differential equations and linear algebra as well.  If
you find it useful, please let us know.  We are always interested to
find out how Octave is being used in other places.

Virtually everyone thinks that the name Octave has something to do with
music, but it is actually the name of a former professor of mine who
wrote a famous textbook on chemical reaction engineering, and who was
also well known for his ability to do quick `back of the envelope'
calculations.  We hope that this software will make it possible for many
people to do more ambitious computations just as easily.

Everyone is encouraged to share this software with others under the
terms of the GNU General Public License (@pxref{Copying}) as described
at the beginning of this manual.  You are also encouraged to help make
Octave more useful by writing and contributing additional functions for
it, and by reporting any problems you may have.

@menu
* Acknowledgements::            
* How You Can Contribute to Octave::  
* Distribution::                
@end menu

@node Acknowledgements, How You Can Contribute to Octave, Preface, Preface
@unnumberedsec Acknowledgements
@cindex acknowledgements

Many people have already contributed to Octave's development.  In
addition to John W. Eaton, the following people have helped write parts
of Octave or helped out in various other ways.

@c Once lp_solve has been added, don't forget to include the lp_solve
@c author, and Kantor, for providing an example.

@itemize @bullet
@item
Thomas Baier @email{baier@@ci.tuwien.ac.at} wrote the original versions
of @code{popen}, @code{pclose}, @code{execute}, @code{sync_system}, and
@code{async_system}.

@item
Karl Berry @email{karl@@cs.umb.edu} wrote the @code{kpathsea} library
that allows Octave to recursively search directory paths for function
and script files.

@item
Georg Beyerle @email{gbeyerle@@awi-potsdam.de} contributed code to save
values in @sc{Matlab}'s @file{.mat}-file format, and has provided many
useful bug reports and suggestions.

@item
John Campbell @email{jcc@@bevo.che.wisc.edu} wrote most of the file and
C-style input and output functions.

@item
Brian Fox @email{bfox@@gnu.org} wrote the @code{readline} library
used for command history editing, and the portion of this manual that
documents it.

@item
Klaus Gebhardt @email{gebhardt@@crunch.ikp.physik.th-darmstadt.de}
ported Octave to OS/2.

@item
A. Scottedward Hodel @email{A.S.Hodel@@eng.auburn.edu} contributed a number
of functions including @code{expm}, @code{qzval}, @code{qzhess},
@code{syl}, @code{lyap}, and @code{balance}.

@item
Kurt Hornik @email{Kurt.Hornik@@ci.tuwien.ac.at} provided the
@code{corrcoef}, @code{cov}, @code{fftconv}, @code{fftfilt}, @code{gcd},
@code{lcd}, @code{kurtosis}, @code{null}, @code{orth}, @code{poly},
@code{polyfit}, @code{roots}, and @code{skewness} functions, supplied
documentation for these and numerous other functions, rewrote the Emacs
mode for editing Octave code and provided its documentation, and has
helped tremendously with testing.  He has also been a constant source of
new ideas for improving Octave.

@item
Phil Johnson @email{johnsonp@@nicco.sscnet.ucla.edu} has helped to make
Linux releases available.

@item
Friedrich Leisch @email{leisch@@ci.tuwien.ac.at} provided the
@code{mahalanobis} function.

@item
Ken Neighbors @email{wkn@@leland.stanford.edu} has provided many useful
bug reports and comments on @sc{Matlab} compatibility.

@item
Rick Niles @email{niles@@axp745.gsfc.nasa.gov} rewrote Octave's plotting
functions to add line styles and the ability to specify an unlimited
number of lines in a single call.  He also continues to track down odd
incompatibilities and bugs.

@item
Mark Odegard @email{meo@@sugarland.unocal.com} provided the initial
implementation of @code{fread}, @code{fwrite}, @code{feof}, and
@code{ferror}.

@item
Tony Richardson @email{arichard@@stark.cc.oh.us} wrote Octave's
image processing functions as well as most of the original polynomial
functions.

@item
R. Bruce Tenison @email{Bruce.Tenison@@eng.auburn.edu} wrote the
@code{hess} and @code{schur} functions.

@item
Teresa Twaroch @email{twaroch@@ci.tuwien.ac.at} provided the functions
@code{gls} and @code{ols}.

@item
Andreas Weingessel @email{Andreas.Weingessel@@ci.tuwien.ac.at} wrote the
audio functions @code{lin2mu}, @code{loadaudio}, @code{mu2lin},
@code{playaudio}, @code{record}, @code{saveaudio}, and @code{setaudio}.

@item
Fook Fah Yap @email{ffy@@eng.cam.ac.uk} provided the @code{fft} and
@code{ifft} functions and valuable bug reports for early versions.
@end itemize

Special thanks to the following people and organizations for
supporting the development of Octave:

@itemize @bullet
@item
Digital Equipment Corporation, for an equipment grant as part of their
External Research Program.

@item
Sun Microsystems, Inc., for an Academic Equipment grant.

@item
International Business Machines, Inc., for providing equipment as part
of a grant to the University of Texas College of Engineering.

@item
Texaco Chemical Company, for providing funding to continue the
development of this software.

@item
The University of Texas College of Engineering, for providing a
Challenge for Excellence Research Supplement, and for providing an
Academic Development Funds grant.

@item
The State of Texas, for providing funding through the Texas
Advanced Technology Program under Grant No. 003658-078.

@item
Noel Bell, Senior Engineer, Texaco Chemical Company, Austin Texas.

@item
James B. Rawlings, Professor, University of Wisconsin-Madison,
Department of Chemical Engineering.

@item
Richard Stallman, for writing GNU.
@end itemize

This project would not have been possible without the GNU software used
in and used to produce Octave.

@node How You Can Contribute to Octave, Distribution, Acknowledgements, Preface
@unnumberedsec How You Can Contribute to Octave
@cindex contributing to Octave
@cindex funding Octave development

There are a number of ways that you can contribute to help make Octave a
better system.  Perhaps the most important way to contribute is to write
high-quality code for solving new problems, and to make your code freely
available for others to use.

If you find Octave useful, consider providing additional funding to
continue its development.  Even a modest amount of additional funding
could make a significant difference in the amount of time that is
available for development and support.

If you cannot provide funding or contribute code, you can still help
make Octave better and more reliable by reporting any bugs you find and
by offering suggestions for ways to improve Octave.  @xref{Trouble}, for
tips on how to write useful bug reports.

@node Distribution,  , How You Can Contribute to Octave, Preface
@unnumberedsec Distribution
@cindex distribution of Octave

Octave is @dfn{free} software.  This means that everyone is free to
use it and free to redistribute it on certain conditions.  Octave is not
in the public domain.  It is copyrighted and there are restrictions on
its distribution, but the restrictions are designed to ensure that
others will have the same freedom to use and redistribute Octave that
you have.  The precise conditions can be found in the GNU General Public
License that comes with Octave and that also appears in @ref{Copying}.

Octave is available on CD-ROM with various collections of other free
software, and from the Free Software Foundation.  Ordering a copy of
Octave from the Free Software Foundation helps to fund the development
of more free software.  For more information, write to

@quotation
Free Software Foundation@*
59 Temple Place---Suite 330@*
Boston, MA 02111--1307@*
USA
@end quotation
 
Octave is also available on the Internet from
@url{ftp://ftp.che.wisc.edu/pub/octave}, and additional information is
available from @url{http://www.che.wisc.edu/octave}.
