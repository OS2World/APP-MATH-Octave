@c Copyright (C) 1996, 1997 John W. Eaton
@c This is part of the Octave manual.
@c For copying conditions, see the file gpl.tex.

@node Image Processing, Audio Processing, Signal Processing, Top
@chapter Image Processing

Octave can display images with the X Window System using the
@code{xloadimage} program.  You do not need to be running X in order to
manipulate images, however, so some of these functions may be useful
even if you are not able to view the results.

Loading images only works with Octave's image format (a file with a
matrix containing the image data, and a matrix containing the
colormap).  Contributions of robust, well-written functions to read
other image formats are welcome.  If you can provide them, or would like
to improve Octave's image processing capabilities in other ways, please
contact @email{bug-octave@@bevo.che.wisc.edu}.

@deftypefn {Function File} {} colormap (@var{map})
@deftypefnx {Function File} {} colormap ("default")
Set the current colormap.

@code{colormap (@var{map})} sets the current colormap to @var{map}.  The
color map should be an @var{n} row by 3 column matrix.  The columns
contain red, green, and blue intensities respectively.  All entries
should be between 0 and 1 inclusive.  The new colormap is returned.

@code{colormap ("default")} restores the default colormap (a gray scale
colormap with 64 entries).  The default colormap is returned.

With no arguments, @code{colormap} returns the current color map.
@end deftypefn

@deftypefn {Function File} {} gray (@var{n})
Return a gray colormap with @var{n} entries corresponding to values from
0 to @var{n}-1.  The argument @var{n} should be a scalar.  If it is
omitted, 64 is assumed.
@end deftypefn

@deftypefn {Function File} {[@var{img}, @var{map}] =} gray2ind (@var{})
Convert a gray scale intensity image to an Octave indexed image.
@end deftypefn

@deftypefn {Function File} {} image (@var{x}, @var{zoom})
Display a matrix as a color image.  The elements of @var{x} are indices
into the current colormap and should have values between 1 and the
length of the colormap.  If @var{zoom} is omitted, a value of 4 is
assumed. 
@end deftypefn

@deftypefn {Function File} {} imagesc (@var{x}, @var{zoom})
Display a scaled version of the matrix @var{x} as a color image.  The
matrix is scaled so that its entries are indices into the current
colormap.  The scaled matrix is returned.  If @var{zoom} is omitted, a
value of 4 is assumed.
@end deftypefn

@deftypefn {Function File} {} imshow (@var{x}, @var{map})
@deftypefnx {Function File} {} imshow (@var{x}, @var{n})
@deftypefnx {Function File} {} imshow (@var{i}, @var{n})
@deftypefnx {Function File} {} imshow (@var{r}, @var{g}, @var{b})
Display images.

@code{imshow (@var{x})} displays an indexed image using the current
colormap.

@code{imshow (@var{x}, @var{map})} displays an indexed image using the
specified colormap.

@code{imshow (@var{i}, @var{n})} displays a gray scale intensity image.

@code{imshow (@var{r}, @var{g}, @var{b})} displays an RGB image.
@end deftypefn

@deftypefn {Function File} {} ind2gray (@var{x}, @var{map})
Convert an Octave indexed image to a gray scale intensity image.
If @var{map} is omitted, the current colormap is used to determine the
intensities.
@end deftypefn

@deftypefn {Function File} {[@var{r}, @var{g}, @var{b}] =} ind2rgb (@var{x}, @var{map})
Convert an indexed image to red, green, and blue color components.
If @var{map} is omitted, the current colormap is used for the conversion.
@end deftypefn

@deftypefn {Function File} {[@var{x}, @var{map}] =} loadimage (@var{file})
Load an image file and it's associated color map from the specified
@var{file}.  The image must be stored in Octave's image format.
@end deftypefn

@deftypefn {Function File} {} rgb2ntsc (@var{rgb})
Image format conversion.
@end deftypefn

@deftypefn {Function File} {} ntsc2rgb (@var{yiq})
Image format conversion.
@end deftypefn

@deftypefn {Function File} {} ocean (@var{n})
Create color colormap.  The argument @var{n} should be a scalar.  If it
is omitted, 64 is assumed.
@end deftypefn

@deftypefn {Function File} {[@var{x}, @var{map}] =} rgb2ind (@var{r}, @var{g}, @var{b})
Convert and RGB image to an Octave indexed image.
@end deftypefn

@deftypefn {Function File} {} saveimage (@var{file}, @var{x}, @var{fmt}, @var{map})
Save the matrix @var{x} to @var{file} in image format @var{fmt}.  Valid
values for @var{fmt} are

@table @code
@item "img"
Octave's image format.  The current colormap is also saved in the file.

@item "ppm"
Portable pixmap format.

@item "ps"
PostScript format.  Note that images saved in PostScript format can not
be read back into Octave with loadimage.
@end table

If the fourth argument is supplied, the specified colormap will also be
saved along with the image.

Note: if the colormap contains only two entries and these entries are
black and white, the bitmap ppm and PostScript formats are used.  If the
image is a gray scale image (the entries within each row of the colormap
are equal) the gray scale ppm and PostScript image formats are used,
otherwise the full color formats are used.
@end deftypefn

@defvr {Built-in Variable} IMAGEPATH			
A colon separated list of directories in which to search for image
files.
@end defvr
