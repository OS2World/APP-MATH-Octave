@c Copyright (C) 1996, 1997 John W. Eaton
@c This is part of the Octave manual.
@c For copying conditions, see the file gpl.tex.

@node Data Types, Numeric Data Types, Getting Started, Top
@chapter Data Types
@cindex data types

All versions of Octave include a number of built-in data types,
including real and complex scalars and matrices, character strings, and
a data structure type.

It is also possible to define new specialized data types by writing a
small amount of C++ code.  On some systems, new data types can be loaded
dynamically while Octave is running, so it is not necessary to recompile
all of Octave just to add a new type.  @xref{Dynamically Linked
Functions} for more information about Octave's dynamic linking
capabilities.  @ref{User-defined Data Types} describes what you must do
to define a new data type for Octave.

@menu
* Built-in Data Types::         
* User-defined Data Types::     
* Object Sizes::                
@end menu

@node Built-in Data Types, User-defined Data Types, Data Types, Data Types
@section Built-in Data Types
@cindex data types, built-in
@cindex built-in data types

The standard built-in data types are real and complex scalars and
matrices, ranges, character strings, and a data structure type.
Additional built-in data types may be added in future versions.  If you
need a specialized data type that is not currently provided as a
built-in type, you are encouraged to write your own user-defined data
type and contribute it for distribution in a future release of Octave.

@menu
* Numeric Objects::             
* String Objects::              
* Data Structure Objects::      
@end menu

@node Numeric Objects, String Objects, Built-in Data Types, Built-in Data Types
@subsection Numeric Objects
@cindex numeric constant
@cindex numeric value

Octave's built-in numeric objects include real and complex scalars and
matrices.  All built-in numeric data is currently stored as double
precision numbers.  On systems that use the IEEE floating point format,
values in the range of approximately
@iftex
@tex
 $2.2251\times10^{-308}$ to $1.7977\times10^{308}$
@end tex
@end iftex
@ifinfo
 2.2251e-308 to 1.7977e+308
@end ifinfo
 can be stored, and the relative precision is approximately
@iftex
@tex
 $2.2204\times10^{-16}$.
@end tex
@end iftex
@ifinfo
 2.2204e-16.
@end ifinfo
The exact values are given by the variables @code{realmin},
@code{realmax}, and @code{eps}, respectively.

Matrix objects can be of any size, and can be dynamically reshaped and
resized.  It is easy to extract individual rows, columns, or submatrices
using a variety of powerful indexing features.  @xref{Index Expressions}.

@xref{Numeric Data Types}, for more information.

@node String Objects, Data Structure Objects, Numeric Objects, Built-in Data Types
@subsection String Objects
@cindex strings
@cindex character strings
@opindex "
@opindex '

A character string in Octave consists of a sequence of characters
enclosed in either double-quote or single-quote marks.  Internally,
Octave currently stores strings as matrices of characters.  All the
indexing operations that work for matrix objects also work for strings.

@xref{Strings}, for more information.

@node Data Structure Objects,  , String Objects, Built-in Data Types
@subsection Data Structure Objects
@cindex structures
@cindex data structures

Octave's data structure type can help you to organize related objects of
different types.  The current implementation uses an associative array
with indices limited to strings, but the syntax is more like C-style
structures.

@xref{Data Structures}, for more information.

@node User-defined Data Types, Object Sizes, Built-in Data Types, Data Types
@section User-defined Data Types
@cindex user-defined data types
@cindex data types, user-defined

Someday I hope to expand this to include a complete description of
Octave's mechanism for managing user-defined data types.  Until this
feature is documented here, you will have to make do by reading the code
in the @file{ov.h}, @file{ops.h}, and related files from Octave's
@file{src} directory.

@node Object Sizes,  , User-defined Data Types, Data Types
@section Object Sizes

The following functions allow you to determine the size of a variable or
expression.  These functions are defined for all objects.  They return
@minus{}1 when the operation doesn't make sense.  For example, Octave's
data structure type doesn't have rows or columns, so the @code{rows} and
@code{columns} functions return @minus{}1 for structure arguments.

@deftypefn {Function File} {} columns (@var{a})
Return the number of columns of @var{a}.
@end deftypefn

@deftypefn {Function File} {} rows (@var{a})
Return the number of rows of @var{a}.
@end deftypefn

@deftypefn {Function File} {} length (@var{a})
Return the number of rows of @var{a} or the number of columns of
@var{a}, whichever is larger.
@end deftypefn

@deftypefn {Function File} {} size (@var{a}, @var{n})
Return the number rows and columns of @var{a}.

With one input argument and one output argument, the result is returned
in a 2 element row vector.  If there are two output arguments, the
number of rows is assigned to the first, and the number of columns to
the second.  For example,

@example
@group
size ([1, 2; 3, 4; 5, 6])
     @result{} [ 3, 2 ]

[nr, nc] = size ([1, 2; 3, 4; 5, 6])
     @result{} nr = 3
     @result{} nc = 2
@end group
@end example

If given a second argument of either 1 or 2, @code{size} will return
only the row or column dimension.  For example

@example
size ([1, 2; 3, 4; 5, 6], 2)
     @result{} 2
@end example

@noindent
returns the number of columns in the given matrix.
@end deftypefn

@deftypefn {Function File} {} isempty (@var{a})
Return 1 if @var{a} is an empty matrix (either the number of rows, or
the number of columns, or both are zero).  Otherwise, return 0.
@end deftypefn

