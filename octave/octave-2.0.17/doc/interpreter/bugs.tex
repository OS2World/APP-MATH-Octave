@c Copyright (C) 1996, 1997 John W. Eaton
@c This is part of the Octave manual.
@c For copying conditions, see the file gpl.tex.

@c The text of this file will eventually appear in the file BUGS
@c in the Octave distribution, as well as in the Octave manual.

@ifclear BUGSONLY
@node Trouble, Installation, Tips, Top
@appendix Known Causes of Trouble
@end ifclear

@ifset BUGSONLY
@include conf.tex

This file documents known bugs in Octave and describes where and how to
report any bugs that you may find.

Copyright (C) 1996, 1997 John W. Eaton.  You may copy, distribute, and
modify it freely as long as you preserve this copyright notice and
permission notice.

@node Trouble,  , Trouble
@chapter Known Causes of Trouble with Octave
@end ifset

@cindex bugs, known
@cindex installation trouble
@cindex known causes of trouble
@cindex troubleshooting

This section describes known problems that affect users of Octave.  Most
of these are not Octave bugs per se---if they were, we would fix them.
But the result for a user may be like the result of a bug.

Some of these problems are due to bugs in other software, some are
missing features that are too much work to add, and some are places
where people's opinions differ as to what is best.

@menu
* Actual Bugs::                 Bugs we will fix later.
* Reporting Bugs::              
* Bug Criteria::                
* Bug Lists::                   
* Bug Reporting::               
* Sending Patches::             
* Service::                     
@end menu

@node Actual Bugs, Reporting Bugs, Trouble, Trouble
@appendixsec Actual Bugs We Haven't Fixed Yet

@itemize @bullet
@item
Output that comes directly from Fortran functions is not sent through
the pager and may appear out of sequence with other output that is sent
through the pager.  One way to avoid this is to force pending output to
be flushed before calling a function that will produce output from
within Fortran functions.  To do this, use the command

@example
fflush (stdout)
@end example

Another possible workaround is to use the command

@example
page_screen_output = "false"
@end example

@noindent
to turn the pager off.

@item
If you get messages like

@example
Input line too long
@end example

when trying to plot many lines on one graph, you have probably generated
a plot command that is too larger for @code{gnuplot}'s fixed-length
buffer for commands.  Splitting up the plot command doesn't help because
replot is implemented in gnuplot by simply appending the new plotting
commands to the old command line and then evaluating it again.

You can demonstrate this `feature' by running gnuplot and doing
something like

@example
  plot sin (x), sin (x), sin (x), ... lots more ..., sin (x)
@end example

@noindent
and then

@example
  replot sin (x), sin (x), sin (x), ... lots more ..., sin (x)
@end example

@noindent
after repeating the replot command a few times, gnuplot will give you
an error.

Also, it doesn't help to use backslashes to enter a plot command over
several lines, because the limit is on the overall command line
length, once the backslashed lines are all pasted together.

Because of this, Octave tries to use as little of the command-line
length as possible by using the shortest possible abbreviations for
all the plot commands and options.  Unfortunately, the length of the
temporary file names is probably what is taking up the most space on
the command line.

You can buy a little bit of command line space by setting the
environment variable @code{TMPDIR} to be "." before starting Octave, or
you can increase the maximum command line length in gnuplot by changing
the following limits in the file plot.h in the gnuplot distribution and
recompiling gnuplot.

@example
#define MAX_LINE_LEN 32768  /* originally 1024 */
#define MAX_TOKENS 8192     /* originally 400 */
@end example  

Of course, this doesn't really fix the problem, but it does make it
much less likely that you will run into trouble unless you are putting
a very large number of lines on a given plot.
@end itemize

A list of ideas for future enhancements is distributed with Octave.  See
the file @file{PROJECTS} in the top level directory in the source
distribution.

@node Reporting Bugs, Bug Criteria, Actual Bugs, Trouble
@appendixsec Reporting Bugs
@cindex bugs
@cindex reporting bugs

Your bug reports play an essential role in making Octave reliable.

When you encounter a problem, the first thing to do is to see if it is
already known.  @xref{Trouble}.  If it isn't known, then you should
report the problem.

Reporting a bug may help you by bringing a solution to your problem, or
it may not.  In any case, the principal function of a bug report is
to help the entire community by making the next version of Octave work
better.  Bug reports are your contribution to the maintenance of Octave.

In order for a bug report to serve its purpose, you must include the
information that makes it possible to fix the bug.

@findex bug_report

If you have Octave working at all, the easiest way to prepare a complete
bug report is to use the Octave function @code{bug_report}.  When you
execute this function, Octave will prompt you for a subject and then
invoke the editor on a file that already contains all the configuration
information.  When you exit the editor, Octave will mail the bug report
for you.

@menu
* Bug Criteria::                
* Where: Bug Lists.             Where to send your bug report.
* Reporting: Bug Reporting.     How to report a bug effectively.
* Patches: Sending Patches.     How to send a patch for Octave.
@end menu

@node Bug Criteria, Bug Lists, Reporting Bugs, Trouble
@appendixsec Have You Found a Bug?
@cindex bug criteria

If you are not sure whether you have found a bug, here are some guidelines:

@itemize @bullet
@cindex fatal signal
@cindex core dump
@item
If Octave gets a fatal signal, for any input whatever, that is
a bug.  Reliable interpreters never crash.

@cindex incorrect output
@cindex incorrect results
@cindex results, incorrect
@cindex answers, incorrect
@cindex erroneous results
@cindex wrong answers
@item
If Octave produces incorrect results, for any input whatever,
that is a bug.

@cindex undefined behavior
@cindex undefined function value
@item
Some output may appear to be incorrect when it is in fact due to a
program whose behavior is undefined, which happened by chance to give
the desired results on another system.  For example, the range operator
may produce different results because of differences in the way floating
point arithmetic is handled on various systems.

@cindex erroneous messages
@cindex incorrect error messages
@cindex error messages, incorrect
@item
If Octave produces an error message for valid input, that is a bug.

@cindex invalid input
@item
If Octave does not produce an error message for invalid input, that is
a bug.  However, you should note that your idea of ``invalid input''
might be my idea of ``an extension'' or ``support for traditional
practice''.

@cindex improving Octave
@cindex suggestions
@item
If you are an experienced user of programs like Octave, your suggestions
for improvement are welcome in any case.
@end itemize

@node Bug Lists, Bug Reporting, Bug Criteria, Trouble
@appendixsec Where to Report Bugs
@cindex bug report mailing lists
@cindex reporting bugs
@cindex bugs, reporting

@findex bug_report

If you have Octave working at all, the easiest way to prepare a complete
bug report is to use the Octave function @code{bug_report}.  When you
execute this function, Octave will prompt you for a subject and then
invoke the editor on a file that already contains all the configuration
information.  When you exit the editor, Octave will mail the bug report
for you.

If for some reason you cannot use Octave's @code{bug_report} function,
send bug reports for Octave to @email{bug-octave@@bevo.che.wisc.edu}.

@strong{Do not send bug reports to @samp{help-octave}}.  Most users of
Octave do not want to receive bug reports.  Those that do have asked to
be on the mailing list.

As a last resort, send bug reports on paper to:

@example
Octave Bugs c/o John W. Eaton
University of Wisconsin-Madison
Department of Chemical Engineering
1415 Engineering Drive
Madison, Wisconsin 53706  USA
@end example

@node Bug Reporting, Sending Patches, Bug Lists, Trouble
@appendixsec How to Report Bugs
@cindex bugs, reporting

Send bug reports for Octave to one of the addresses listed in
@ref{Bug Lists}.

The fundamental principle of reporting bugs usefully is this:
@strong{report all the facts}.  If you are not sure whether to state a
fact or leave it out, state it!

Often people omit facts because they think they know what causes the
problem and they conclude that some details don't matter.  Thus, you might
assume that the name of the variable you use in an example does not matter.
Well, probably it doesn't, but one cannot be sure.  Perhaps the bug is a
stray memory reference which happens to fetch from the location where that
name is stored in memory; perhaps, if the name were different, the contents
of that location would fool the interpreter into doing the right thing
despite the bug.  Play it safe and give a specific, complete example.

Keep in mind that the purpose of a bug report is to enable someone to
fix the bug if it is not known. Always write your bug reports on
the assumption that the bug is not known.

Sometimes people give a few sketchy facts and ask, ``Does this ring a
bell?''  This cannot help us fix a bug.  It is better to send a complete
bug report to begin with.

Try to make your bug report self-contained.  If we have to ask you for
more information, it is best if you include all the previous information
in your response, as well as the information that was missing.

To enable someone to investigate the bug, you should include all these
things:

@itemize @bullet
@item
The version of Octave.  You can get this by noting the version number
that is printed when Octave starts, or running it with the @samp{-v}
option.

@item
A complete input file that will reproduce the bug.

A single statement may not be enough of an example---the bug might
depend on other details that are missing from the single statement where
the error finally occurs.

@item
The command arguments you gave Octave to execute that example
and observe the bug.  To guarantee you won't omit something important,
list all the options. 

If we were to try to guess the arguments, we would probably guess wrong
and then we would not encounter the bug.

@item
The type of machine you are using, and the operating system name and
version number.

@item
The command-line arguments you gave to the @code{configure} command when
you installed the interpreter.

@item
A complete list of any modifications you have made to the interpreter
source.

Be precise about these changes---show a context diff for them.

@item
Details of any other deviations from the standard procedure for installing
Octave.

@cindex incorrect output
@cindex incorrect results
@cindex results, incorrect
@cindex answers, incorrect
@cindex erroneous results
@cindex wrong answers
@item
A description of what behavior you observe that you believe is
incorrect.  For example, "The interpreter gets a fatal signal," or, "The
output produced at line 208 is incorrect."

Of course, if the bug is that the interpreter gets a fatal signal, then
one can't miss it.  But if the bug is incorrect output, we might not
notice unless it is glaringly wrong.

Even if the problem you experience is a fatal signal, you should still
say so explicitly.  Suppose something strange is going on, such as, your
copy of the interpreter is out of synch, or you have encountered a bug
in the C library on your system.  Your copy might crash and the copy
here would not.  If you said to expect a crash, then when the
interpreter here fails to crash, we would know that the bug was not
happening.  If you don't say to expect a crash, then we would not know
whether the bug was happening.  We would not be able to draw any
conclusion from our observations.

Often the observed symptom is incorrect output when your program is run.
Unfortunately, this is not enough information unless the program is
short and simple.  It is very helpful if you can include an explanation
of the expected output, and why the actual output is incorrect.

@item
If you wish to suggest changes to the Octave source, send them as
context diffs.  If you even discuss something in the Octave source,
refer to it by context, not by line number, because the line numbers in
the development sources probably won't match those in your sources.
@end itemize

Here are some things that are not necessary:

@itemize @bullet
@cindex bugs, investigating
@item
A description of the envelope of the bug.

Often people who encounter a bug spend a lot of time investigating which
changes to the input file will make the bug go away and which changes
will not affect it.  Such information is usually not necessary to enable
us to fix bugs in Octave, but if you can find a simpler example to
report @emph{instead} of the original one, that is a convenience.
Errors in the output will be easier to spot, running under the debugger
will take less time, etc. Most Octave bugs involve just one function, so
the most straightforward way to simplify an example is to delete all the
function definitions except the one in which the bug occurs.

However, simplification is not vital; if you don't want to do
this, report the bug anyway and send the entire test case you
used.

@item
A patch for the bug.  Patches can be helpful, but if you find a bug, you
should report it, even if you cannot send a fix for the problem.
@end itemize

@node Sending Patches, Service, Bug Reporting, Trouble
@appendixsec Sending Patches for Octave
@cindex improving Octave
@cindex diffs, submitting
@cindex patches, submitting
@cindex submitting diffs
@cindex submitting patches

If you would like to write bug fixes or improvements for Octave, that is
very helpful.  When you send your changes, please follow these
guidelines to avoid causing extra work for us in studying the patches.

If you don't follow these guidelines, your information might still be
useful, but using it will take extra work.  Maintaining Octave is a lot
of work in the best of circumstances, and we can't keep up unless you do
your best to help.

@itemize @bullet
@item
Send an explanation with your changes of what problem they fix or what
improvement they bring about.  For a bug fix, just include a copy of the
bug report, and explain why the change fixes the bug.

@item
Always include a proper bug report for the problem you think you have
fixed.  We need to convince ourselves that the change is right before
installing it.  Even if it is right, we might have trouble judging it if
we don't have a way to reproduce the problem.

@item
Include all the comments that are appropriate to help people reading the
source in the future understand why this change was needed.

@item
Don't mix together changes made for different reasons.
Send them @emph{individually}.

If you make two changes for separate reasons, then we might not want to
install them both.  We might want to install just one.

@item
Use @samp{diff -c} to make your diffs.  Diffs without context are hard
for us to install reliably.  More than that, they make it hard for us to
study the diffs to decide whether we want to install them.  Unidiff
format is better than contextless diffs, but not as easy to read as
@samp{-c} format.

If you have GNU diff, use @samp{diff -cp}, which shows the name of the
function that each change occurs in.

@item
Write the change log entries for your changes.

Read the @file{ChangeLog} file to see what sorts of information to put
in, and to learn the style that we use.  The purpose of the change log
is to show people where to find what was changed.  So you need to be
specific about what functions you changed; in large functions, it's
often helpful to indicate where within the function the change was made.

On the other hand, once you have shown people where to find the change,
you need not explain its purpose. Thus, if you add a new function, all
you need to say about it is that it is new.  If you feel that the
purpose needs explaining, it probably does---but the explanation will be
much more useful if you put it in comments in the code.

If you would like your name to appear in the header line for who made
the change, send us the header line.
@end itemize

@node Service,  , Sending Patches, Trouble
@appendixsec How To Get Help with Octave
@cindex help, where to find

The mailing list @email{help-octave@@bevo.che.wisc.edu} exists for the
discussion of matters related to using and installing Octave.  If would
like to join the discussion, please send a short note to
@email{help-octave@strong{-request}@@bevo.che.wisc.edu}.

@strong{Please do not} send requests to be added or removed from the
mailing list, or other administrative trivia to the list itself.

If you think you have found a bug in the installation procedure,
however, you should send a complete bug report for the problem to
@email{bug-octave@@bevo.che.wisc.edu}.  @xref{Bug Reporting} for
information that will help you to submit a useful report.
