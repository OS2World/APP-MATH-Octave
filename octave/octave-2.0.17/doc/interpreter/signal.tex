@c Copyright (C) 1996, 1997 John W. Eaton
@c This is part of the Octave manual.
@c For copying conditions, see the file gpl.tex.

@node Signal Processing, Image Processing, Control Theory, Top
@chapter Signal Processing

I hope that someday Octave will include more signal processing
functions.  If you would like to help improve Octave in this area,
please contact @email{bug-octave@@bevo.che.wisc.edu}.

@deftypefn {Function File} {} detrend (@var{x}, @var{p})
If @var{x} is a vector, @code{detrend (@var{x}, @var{p})} removes the
best fit of a polynomial of order @var{p} from the data @var{x}.

If @var{x} is a matrix, @code{detrend (@var{x}, @var{p})} does the same
for each column in @var{x}.

The second argument is optional.  If it is not specified, a value of 1
is assumed.  This corresponds to removing a linear trend.
@end deftypefn

@deftypefn {Function} {} fft (@var{a}, @var{n})
Compute the FFT of @var{a} using subroutines from @sc{Fftpack}.  If @var{a}
is a matrix, @code{fft} computes the FFT for each column of @var{a}.

If called with two arguments, @var{n} is expected to be an integer
specifying the number of elements of @var{a} to use.  If @var{a} is a
matrix, @var{n} specifies the number of rows of @var{a} to use.  If
@var{n} is larger than the size of @var{a}, @var{a} is resized and
padded with zeros.
@end deftypefn

@deftypefn {Loadable Function} {} ifft (@var{a}, @var{n})
Compute the inverse FFT of @var{a} using subroutines from @sc{Fftpack}.  If
@var{a} is a matrix, @code{fft} computes the inverse FFT for each column
of @var{a}.

If called with two arguments, @var{n} is expected to be an integer
specifying the number of elements of @var{a} to use.  If @var{a} is a
matrix, @var{n} specifies the number of rows of @var{a} to use.  If
@var{n} is larger than the size of @var{a}, @var{a} is resized and
padded with zeros.
@end deftypefn

@deftypefn {Loadable Function} {} fft2 (@var{a}, @var{n}, @var{m})
Compute the two dimensional FFT of @var{a}.

The optional arguments @var{n} and @var{m} may be used specify the
number of rows and columns of @var{a} to use.  If either of these is
larger than the size of @var{a}, @var{a} is resized and padded with
zeros.
@end deftypefn

@deftypefn {Loadable Function} {} ifft2 (@var{a}, @var{n}, @var{m})
Compute the two dimensional inverse FFT of @var{a}.

The optional arguments @var{n} and @var{m} may be used specify the
number of rows and columns of @var{a} to use.  If either of these is
larger than the size of @var{a}, @var{a} is resized and padded with
zeros.
@end deftypefn

@deftypefn {Built-in Function} {} fftconv (@var{a}, @var{b}, @var{n})
Return the convolution of the vectors @var{a} and @var{b}, as a vector
with length equal to the @code{length (a) + length (b) - 1}.  If @var{a}
and @var{b} are the coefficient vectors of two polynomials, the returned
value is the coefficient vector of the product polynomial.

The computation uses the FFT by calling the function @code{fftfilt}.  If
the optional argument @var{n} is specified, an N-point FFT is used.
@end deftypefn

@deftypefn {Function File} {} fftfilt (@var{b}, @var{x}, @var{n})

With two arguments, @code{fftfilt} filters @var{x} with the FIR filter
@var{b} using the FFT.

Given the optional third argument, @var{n}, @code{fftfilt} uses the
overlap-add method to filter @var{x} with @var{b} using an N-point FFT.
@end deftypefn

@deftypefn {Loadable Function} {y =} filter (@var{b}, @var{a}, @var{x})
Return the solution to the following linear, time-invariant difference
equation:
@iftex
@tex
$$
\sum_{k=0}^N a_{k+1} y_{n-k} = \sum_{k=0}^M b_{k+1} x_{n-k}, \qquad
 1 \le n \le P
$$
@end tex
@end iftex
@ifinfo

@smallexample
   N                   M
  SUM a(k+1) y(n-k) = SUM b(k+1) x(n-k)      for 1<=n<=length(x)
  k=0                 k=0
@end smallexample
@end ifinfo

@noindent
where
@ifinfo
 N=length(a)-1 and M=length(b)-1.
@end ifinfo
@iftex
@tex
 $a \in \Re^{N-1}$, $b \in \Re^{M-1}$, and $x \in \Re^P$.
@end tex
@end iftex
An equivalent form of this equation is:
@iftex
@tex
$$
y_n = -\sum_{k=1}^N c_{k+1} y_{n-k} + \sum_{k=0}^M d_{k+1} x_{n-k}, \qquad
 1 \le n \le P
$$
@end tex
@end iftex
@ifinfo

@smallexample
            N                   M
  y(n) = - SUM c(k+1) y(n-k) + SUM d(k+1) x(n-k)  for 1<=n<=length(x)
           k=1                 k=0
@end smallexample
@end ifinfo

@noindent
where
@ifinfo
 c = a/a(1) and d = b/a(1).
@end ifinfo
@iftex
@tex
$c = a/a_1$ and $d = b/a_1$.
@end tex
@end iftex

In terms of the z-transform, y is the result of passing the discrete-
time signal x through a system characterized by the following rational
system function:
@iftex
@tex
$$
H(z) = {\displaystyle\sum_{k=0}^M d_{k+1} z^{-k}
        \over 1 + \displaystyle\sum_{k+1}^N c_{k+1} z^{-k}}
$$
@end tex
@end iftex
@ifinfo

@example
             M
            SUM d(k+1) z^(-k)
            k=0
  H(z) = ----------------------
               N
          1 + SUM c(k+1) z(-k)
              k=1
@end example
@end ifinfo
@end deftypefn

@deftypefn {Loadable Function} {[@var{y}, @var{sf}] =} filter (@var{b}, @var{a}, @var{x}, @var{si})
This is the same as the @code{filter} function described above, except
that @var{si} is taken as the initial state of the system and the final
state is returned as @var{sf}.  The state vector is a column vector
whose length is equal to the length of the longest coefficient vector
minus one.  If @var{si} is not set, the initial state vector is set to
all zeros.
@end deftypefn

@deftypefn {Function File} {[@var{h}, @var{w}] =} freqz (@var{b}, @var{a}, @var{n}, "whole")
Return the complex frequency response @var{h} of the rational IIR filter
whose numerator and denominator coefficients are @var{b} and @var{a},
respectively.  The response is evaluated at @var{n} angular frequencies
between 0 and
@ifinfo
 2*pi.
@end ifinfo
@iftex
@tex
 $2\pi$.
@end tex
@end iftex

@noindent
The output value @var{w} is a vector of the frequencies.

If the fourth argument is omitted, the response is evaluated at
frequencies between 0 and
@ifinfo
 pi.
@end ifinfo
@iftex
@tex
 $\pi$.
@end tex
@end iftex

If @var{n} is omitted, a value of 512 is assumed.

If @var{a} is omitted, the denominator is assumed to be 1 (this
corresponds to a simple FIR filter).

For fastest computation, @var{n} should factor into a small number of
small primes.
@end deftypefn

@deftypefn {Function File} {} sinc (@var{x})
Return
@iftex
@tex
$ \sin (\pi x)/(\pi x)$.
@end tex
@end iftex
@ifinfo
 sin(pi*x)/(pi*x).
@end ifinfo
@end deftypefn
