@c Copyright (C) 1996, 1997 John W. Eaton
@c This is part of the Octave manual.
@c For copying conditions, see the file gpl.tex.

@node Functions and Scripts, Error Handling, Statements, Top
@chapter Functions and Script Files
@cindex defining functions
@cindex user-defined functions
@cindex functions, user-defined
@cindex script files

Complicated Octave programs can often be simplified by defining
functions.  Functions can be defined directly on the command line during
interactive Octave sessions, or in external files, and can be called just
like built-in functions.

@menu
* Defining Functions::          
* Multiple Return Values::      
* Variable-length Argument Lists::  
* Variable-length Return Lists::  
* Returning From a Function::   
* Function Files::              
* Script Files::                
* Dynamically Linked Functions::  
* Organization of Functions::   
@end menu

@node Defining Functions, Multiple Return Values, Functions and Scripts, Functions and Scripts
@section Defining Functions
@cindex @code{function} statement
@cindex @code{endfunction} statement

In its simplest form, the definition of a function named @var{name}
looks like this:

@example
@group
function @var{name}
  @var{body}
endfunction
@end group
@end example

@noindent
A valid function name is like a valid variable name: a sequence of
letters, digits and underscores, not starting with a digit.  Functions
share the same pool of names as variables.

The function @var{body} consists of Octave statements.  It is the
most important part of the definition, because it says what the function
should actually @emph{do}.

For example, here is a function that, when executed, will ring the bell
on your terminal (assuming that it is possible to do so):

@example
@group
function wakeup
  printf ("\a");
endfunction
@end group
@end example

The @code{printf} statement (@pxref{Input and Output}) simply tells
Octave to print the string @code{"\a"}.  The special character @samp{\a}
stands for the alert character (ASCII 7).  @xref{Strings}.

Once this function is defined, you can ask Octave to evaluate it by
typing the name of the function.

Normally, you will want to pass some information to the functions you
define.  The syntax for passing parameters to a function in Octave is

@example
@group
function @var{name} (@var{arg-list})
  @var{body}
endfunction
@end group
@end example

@noindent
where @var{arg-list} is a comma-separated list of the function's
arguments.  When the function is called, the argument names are used to
hold the argument values given in the call.  The list of arguments may
be empty, in which case this form is equivalent to the one shown above.

To print a message along with ringing the bell, you might modify the
@code{beep} to look like this:

@example
@group
function wakeup (message)
  printf ("\a%s\n", message);
endfunction
@end group
@end example

Calling this function using a statement like this

@example
wakeup ("Rise and shine!");
@end example

@noindent
will cause Octave to ring your terminal's bell and print the message
@samp{Rise and shine!}, followed by a newline character (the @samp{\n}
in the first argument to the @code{printf} statement).

In most cases, you will also want to get some information back from the
functions you define.  Here is the syntax for writing a function that
returns a single value:

@example
@group
function @var{ret-var} = @var{name} (@var{arg-list})
  @var{body}
endfunction
@end group
@end example

@noindent
The symbol @var{ret-var} is the name of the variable that will hold the
value to be returned by the function.  This variable must be defined
before the end of the function body in order for the function to return
a value.

Variables used in the body of a function are local to the
function.  Variables named in @var{arg-list} and @var{ret-var} are also
local to the function.  @xref{Global Variables}, for information about
how to access global variables inside a function.

For example, here is a function that computes the average of the
elements of a vector:

@example
@group
function retval = avg (v)
  retval = sum (v) / length (v);
endfunction
@end group
@end example

If we had written @code{avg} like this instead,

@example
@group
function retval = avg (v)
  if (is_vector (v))
    retval = sum (v) / length (v);
  endif
endfunction
@end group
@end example

@noindent
and then called the function with a matrix instead of a vector as the
argument, Octave would have printed an error message like this:

@example
@group
error: `retval' undefined near line 1 column 10
error: evaluating index expression near line 7, column 1
@end group
@end example

@noindent
because the body of the @code{if} statement was never executed, and
@code{retval} was never defined.  To prevent obscure errors like this,
it is a good idea to always make sure that the return variables will
always have values, and to produce meaningful error messages when
problems are encountered.  For example, @code{avg} could have been
written like this:

@example
@group
function retval = avg (v)
  retval = 0;
  if (is_vector (v))
    retval = sum (v) / length (v);
  else
    error ("avg: expecting vector argument");
  endif
endfunction
@end group
@end example

There is still one additional problem with this function.  What if it is
called without an argument?  Without additional error checking, Octave
will probably print an error message that won't really help you track
down the source of the error.  To allow you to catch errors like this,
Octave provides each function with an automatic variable called
@code{nargin}.  Each time a function is called, @code{nargin} is
automatically initialized to the number of arguments that have actually
been passed to the function.  For example, we might rewrite the
@code{avg} function like this:

@example
@group
function retval = avg (v)
  retval = 0;
  if (nargin != 1)
    usage ("avg (vector)");
  endif
  if (is_vector (v))
    retval = sum (v) / length (v);
  else
    error ("avg: expecting vector argument");
  endif
endfunction
@end group
@end example

Although Octave does not automatically report an error if you call a
function with more arguments than expected, doing so probably indicates
that something is wrong.  Octave also does not automatically report an
error if a function is called with too few arguments, but any attempt to
use a variable that has not been given a value will result in an error.
To avoid such problems and to provide useful messages, we check for both
possibilities and issue our own error message.

@defvr {Automatic Variable} nargin
When a function is called, this local variable is automatically
initialized to the number of arguments passed to the function.  At the
top level, @code{nargin} holds the number of command line arguments that
were passed to Octave.
@end defvr

@defvr {Built-in Variable} silent_functions
If the value of @code{silent_functions} is nonzero, internal output
from a function is suppressed.  Otherwise, the results of expressions
within a function body that are not terminated with a semicolon will
have their values printed.  The default value is 0.

For example, if the function

@example
function f ()
  2 + 2
endfunction
@end example

@noindent
is executed, Octave will either print @samp{ans = 4} or nothing
depending on the value of @code{silent_functions}.
@end defvr

@defvr {Built-in Variable} warn_missing_semicolon
If the value of this variable is nonzero, Octave will warn when
statements in function definitions don't end in semicolons.  The default
value is 0.
@end defvr

@node Multiple Return Values, Variable-length Argument Lists, Defining Functions, Functions and Scripts
@section Multiple Return Values

Unlike many other computer languages, Octave allows you to define
functions that return more than one value.  The syntax for defining
functions that return multiple values is

@example
function [@var{ret-list}] = @var{name} (@var{arg-list})
  @var{body}
endfunction
@end example

@noindent
where @var{name}, @var{arg-list}, and @var{body} have the same meaning
as before, and @var{ret-list} is a comma-separated list of variable
names that will hold the values returned from the function.  The list of
return values must have at least one element.  If @var{ret-list} has
only one element, this form of the @code{function} statement is
equivalent to the form described in the previous section.

Here is an example of a function that returns two values, the maximum
element of a vector and the index of its first occurrence in the vector.

@example
@group
function [max, idx] = vmax (v)
  idx = 1;
  max = v (idx);
  for i = 2:length (v)
    if (v (i) > max)
      max = v (i);
      idx = i;
    endif
  endfor
endfunction
@end group
@end example

In this particular case, the two values could have been returned as
elements of a single array, but that is not always possible or
convenient.  The values to be returned may not have compatible
dimensions, and it is often desirable to give the individual return
values distinct names.

In addition to setting @code{nargin} each time a function is called,
Octave also automatically initializes @code{nargout} to the number of
values that are expected to be returned.  This allows you to write
functions that behave differently depending on the number of values that
the user of the function has requested.  The implicit assignment to the
built-in variable @code{ans} does not figure in the count of output
arguments, so the value of @code{nargout} may be zero.

The @code{svd} and @code{lu} functions are examples of built-in
functions that behave differently depending on the value of
@code{nargout}.

It is possible to write functions that only set some return values.  For
example, calling the function

@example
function [x, y, z] = f ()
  x = 1;
  z = 2;
endfunction
@end example

@noindent
as

@example
[a, b, c] = f ()
@end example

@noindent
produces:

@example
a = 1

b = [](0x0)

c = 2
@end example

@noindent
provided that the built-in variable @code{define_all_return_values} is
nonzero and the value of @code{default_return_value} is @samp{[]}.
@xref{Summary of Built-in Variables}.

@defvr {Automatic Variable} nargout
When a function is called, this local variable is automatically
initialized to the number of arguments expected to be returned.  For
example, 

@example
f ()
@end example

@noindent
will result in @code{nargout} being set to 0 inside the function
@code{f} and

@example
[s, t] = f ()
@end example

@noindent
will result in @code{nargout} being set to 2 inside the function
@code{f}.

At the top level, @code{nargout} is undefined.
@end defvr

@defvr {Built-in Variable} default_return_value
The value given to otherwise uninitialized return values if
@code{define_all_return_values} is nonzero.  The default value is
@code{[]}.
@end defvr

@defvr {Built-in Variable} define_all_return_values
If the value of @code{define_all_return_values} is nonzero, Octave
will substitute the value specified by @code{default_return_value} for
any return values that remain undefined when a function returns.  The
default value is 0.
@end defvr

@deftypefn {Function File} {} nargchk (@var{nargin_min}, @var{nargin_max}, @var{n})
If @var{n} is in the range @var{nargin_min} through @var{nargin_max}
inclusive, return the empty matrix.  Otherwise, return a message
indicating whether @var{n} is too large or too small.

This is useful for checking to see that the number of arguments supplied
to a function is within an acceptable range.
@end deftypefn

@node Variable-length Argument Lists, Variable-length Return Lists, Multiple Return Values, Functions and Scripts
@section Variable-length Argument Lists
@cindex Variable-length argument lists
@cindex @code{...}

Octave has a real mechanism for handling functions that take an
unspecified number of arguments, so it is not necessary to place an
upper bound on the number of optional arguments that a function can
accept.

@c XXX FIXME XXX -- should we add a note about why this feature is not
@c compatible with Matlab 5?

Here is an example of a function that uses the new syntax to print a
header followed by an unspecified number of values:

@example
function foo (heading, ...)
  disp (heading);
  va_start ();
  ## Pre-decrement to skip `heading' arg.
  while (--nargin)
    disp (va_arg ());
  endwhile
endfunction
@end example

The ellipsis that marks the variable argument list may only appear once
and must be the last element in the list of arguments.

@deftypefn {Built-in Function} {} va_start ()
Position an internal pointer to the first unnamed argument and allows
you to cycle through the arguments more than once.  It is not necessary
to call @code{va_start} if you do not plan to cycle through the
arguments more than once.  This function may only be called inside
functions that have been declared to accept a variable number of input
arguments.
@end deftypefn

@deftypefn {Built-in Function} {} va_arg ()
Return the value of the next available argument and move the internal
pointer to the next argument.  It is an error to call @code{va_arg()}
when there are no more arguments available.
@end deftypefn

Sometimes it is useful to be able to pass all unnamed arguments to
another function.  The keyword @var{all_va_args} makes this very easy to
do.  For example,

@example
@group
function f (...)
  while (nargin--)
    disp (va_arg ())
  endwhile
endfunction

function g (...)
  f ("begin", all_va_args, "end")
endfunction

g (1, 2, 3)

     @print{} begin
     @print{} 1
     @print{} 2
     @print{} 3
     @print{} end
@end group
@end example

@defvr {Keyword} all_va_args
This keyword stands for the entire list of optional argument, so it is
possible to use it more than once within the same function without
having to call @code{va_start}.  It can only be used within functions
that take a variable number of arguments.  It is an error to use it in
other contexts.
@end defvr

@node Variable-length Return Lists, Returning From a Function, Variable-length Argument Lists, Functions and Scripts
@section Variable-length Return Lists
@cindex Variable-length return lists
@cindex @code{...}

Octave also has a real mechanism for handling functions that return an
unspecified number of values, so it is no longer necessary to place an
upper bound on the number of outputs that a function can produce.

Here is an example of a function that uses a variable-length return list
to produce @var{n} values:

@example
@group
function [...] = f (n, x)
  for i = 1:n
    vr_val (i * x);
  endfor
endfunction

[dos, quatro] = f (2, 2)
     @result{} dos = 2
     @result{} quatro = 4
@end group
@end example

As with variable argument lists, the ellipsis that marks the variable
return list may only appear once and must be the last element in the
list of returned values.

@deftypefn {Built-in Function} {} vr_val (@var{val})
Each time this function is called, it places the value of its argument
at the end of the list of values to return from the current function.
Once @code{vr_val} has been called, there is no way to go back to the
beginning of the list and rewrite any of the return values.  This
function may only be called within functions that have been declared to
return an unspecified number of output arguments (by using the special
ellipsis notation described above).
@end deftypefn

@node Returning From a Function, Function Files, Variable-length Return Lists, Functions and Scripts
@section Returning From a Function

The body of a user-defined function can contain a @code{return} statement.
This statement returns control to the rest of the Octave program.  It
looks like this:

@example
return
@end example

Unlike the @code{return} statement in C, Octave's @code{return}
statement cannot be used to return a value from a function.  Instead,
you must assign values to the list of return variables that are part of
the @code{function} statement.  The @code{return} statement simply makes
it easier to exit a function from a deeply nested loop or conditional
statement.

Here is an example of a function that checks to see if any elements of a
vector are nonzero.

@example
@group
function retval = any_nonzero (v)
  retval = 0;
  for i = 1:length (v)
    if (v (i) != 0)
      retval = 1;
      return;
    endif
  endfor
  printf ("no nonzero elements found\n");
endfunction
@end group
@end example

Note that this function could not have been written using the
@code{break} statement to exit the loop once a nonzero value is found
without adding extra logic to avoid printing the message if the vector
does contain a nonzero element.

@defvr {Keyword} return
When Octave encounters the keyword @code{return} inside a function or
script, it returns control to be caller immediately.  At the top level,
the return statement is ignored.  A @code{return} statement is assumed
at the end of every function definition.
@end defvr

@defvr {Built-in Variable} return_last_computed_value
If the value of @code{return_last_computed_value} is true, and a
function is defined without explicitly specifying a return value, the
function will return the value of the last expression.  Otherwise, no
value will be returned.  The default value is 0.

For example, the function

@example
function f ()
  2 + 2;
endfunction
@end example

@noindent
will either return nothing, if the value of
@code{return_last_computed_value} is 0, or 4, if the value of
@code{return_last_computed_value} is nonzero.
@end defvr

@node Function Files, Script Files, Returning From a Function, Functions and Scripts
@section Function Files
@cindex function file

Except for simple one-shot programs, it is not practical to have to
define all the functions you need each time you need them.  Instead, you
will normally want to save them in a file so that you can easily edit
them, and save them for use at a later time.

Octave does not require you to load function definitions from files
before using them.  You simply need to put the function definitions in a
place where Octave can find them.

When Octave encounters an identifier that is undefined, it first looks
for variables or functions that are already compiled and currently
listed in its symbol table.  If it fails to find a definition there, it
searches the list of directories specified by the built-in variable
@code{LOADPATH} for files ending in @file{.m} that have the same base
name as the undefined identifier.@footnote{The @samp{.m} suffix was
chosen for compatibility with @sc{Matlab}.}  Once Octave finds a file
with a name that matches, the contents of the file are read.  If it
defines a @emph{single} function, it is compiled and executed.
@xref{Script Files}, for more information about how you can define more
than one function in a single file.

When Octave defines a function from a function file, it saves the full
name of the file it read and the time stamp on the file.  After
that, it checks the time stamp on the file every time it needs the
function.  If the time stamp indicates that the file has changed since
the last time it was read, Octave reads it again.

Checking the time stamp allows you to edit the definition of a function
while Octave is running, and automatically use the new function
definition without having to restart your Octave session.  Checking the
time stamp every time a function is used is rather inefficient, but it
has to be done to ensure that the correct function definition is used.

To avoid degrading performance unnecessarily by checking the time stamps
on functions that are not likely to change, Octave assumes that function
files in the directory tree
@file{@var{octave-home}/share/octave/@var{version}/m}
will not change, so it doesn't have to check their time stamps every time the
functions defined in those files are used.  This is normally a very good
assumption and provides a significant improvement in performance for the
function files that are distributed with Octave.

If you know that your own function files will not change while you are
running Octave, you can improve performance by setting the variable
@code{ignore_function_time_stamp} to @code{"all"}, so that Octave will
ignore the time stamps for all function files.  Setting it to
@code{"system"} gives the default behavior.  If you set it to anything
else, Octave will check the time stamps on all function files.

@c XXX FIXME XXX -- note about time stamps on files in NFS environments?

@defvr {Built-in Variable} DEFAULT_LOADPATH
A colon separated list of directories in which to search for function
files by default.  The value of this variable is also automatically
substituted for leading, trailing, or doubled colons that appear in the
built-in variable @code{LOADPATH}.
@end defvr

@defvr {Built-in Variable} LOADPATH
A colon separated list of directories in which to search for function
files.  @xref{Functions and Scripts}.  The value of @code{LOADPATH}
overrides the environment variable @code{OCTAVE_PATH}.  @xref{Installation}.

@code{LOADPATH} is now handled in the same way as @TeX{} handles
@code{TEXINPUTS}.  Leading, trailing, or doubled colons that appear in
@code{LOADPATH} are replaced by the value of @code{DEFAULT_LOADPATH}.
The default value of @code{LOADPATH} is @code{":"}, which tells Octave
to search in the directories specified by @code{DEFAULT_LOADPATH}.

In addition, if any path element ends in @samp{//}, that directory and
all subdirectories it contains are searched recursively for function
files.  This can result in a slight delay as Octave caches the lists of
files found in the @code{LOADPATH} the first time Octave searches for a
function.  After that, searching is usually much faster because Octave
normally only needs to search its internal cache for files.

To improve performance of recursive directory searching, it is best for
each directory that is to be searched recursively to contain
@emph{either} additional subdirectories @emph{or} function files, but
not a mixture of both.

@xref{Organization of Functions} for a description of the function file
directories that are distributed with Octave.
@end defvr

@defvr {Built-in Variable} ignore_function_time_stamp
This variable can be used to prevent Octave from making the system call
@code{stat} each time it looks up functions defined in function files.
If @code{ignore_function_time_stamp} to @code{"system"}, Octave will not
automatically recompile function files in subdirectories of
@file{@var{octave-home}/lib/@var{version}} if they have changed since
they were last compiled, but will recompile other function files in the
@code{LOADPATH} if they change.  If set to @code{"all"}, Octave will not
recompile any function files unless their definitions are removed with
@code{clear}.  For any other value of @code{ignore_function_time_stamp},
Octave will always check to see if functions defined in function files
need to recompiled.  The default value of @code{ignore_function_time_stamp} is
@code{"system"}.
@end defvr

@defvr {Built-in Variable} warn_function_name_clash
If the value of @code{warn_function_name_clash} is nonzero, a warning is
issued when Octave finds that the name of a function defined in a
function file differs from the name of the file.  (If the names
disagree, the name declared inside the file is ignored.)  If the value
is 0, the warning is omitted.  The default value is 1.
@end defvr

@node Script Files, Dynamically Linked Functions, Function Files, Functions and Scripts
@section Script Files

A script file is a file containing (almost) any sequence of Octave
commands.  It is read and evaluated just as if you had typed each
command at the Octave prompt, and provides a convenient way to perform a
sequence of commands that do not logically belong inside a function.

Unlike a function file, a script file must @emph{not} begin with the
keyword @code{function}.  If it does, Octave will assume that it is a
function file, and that it defines a single function that should be
evaluated as soon as it is defined.

A script file also differs from a function file in that the variables
named in a script file are not local variables, but are in the same
scope as the other variables that are visible on the command line.

Even though a script file may not begin with the @code{function}
keyword, it is possible to define more than one function in a single
script file and load (but not execute) all of them at once.  To do 
this, the first token in the file (ignoring comments and other white
space) must be something other than @code{function}.  If you have no
other statements to evaluate, you can use a statement that has no
effect, like this:

@example
@group
# Prevent Octave from thinking that this
# is a function file:

1;

# Define function one:

function one ()
  ...
@end group
@end example

To have Octave read and compile these functions into an internal form,
you need to make sure that the file is in Octave's @code{LOADPATH}, then
simply type the base name of the file that contains the commands.
(Octave uses the same rules to search for script files as it does to
search for function files.)

If the first token in a file (ignoring comments) is @code{function},
Octave will compile the function and try to execute it, printing a
message warning about any non-whitespace characters that appear after
the function definition.

Note that Octave does not try to look up the definition of any identifier
until it needs to evaluate it.  This means that Octave will compile the
following statements if they appear in a script file, or are typed at
the command line,

@example
@group
# not a function file:
1;
function foo ()
  do_something ();
endfunction
function do_something ()
  do_something_else ();
endfunction
@end group
@end example

@noindent
even though the function @code{do_something} is not defined before it is
referenced in the function @code{foo}.  This is not an error because
Octave does not need to resolve all symbols that are referenced by a
function until the function is actually evaluated.

Since Octave doesn't look for definitions until they are needed, the
following code will always print @samp{bar = 3} whether it is typed
directly on the command line, read from a script file, or is part of a
function body, even if there is a function or script file called
@file{bar.m} in Octave's @code{LOADPATH}.

@example
@group
eval ("bar = 3");
bar
@end group
@end example

Code like this appearing within a function body could fool Octave if
definitions were resolved as the function was being compiled.  It would
be virtually impossible to make Octave clever enough to evaluate this
code in a consistent fashion.  The parser would have to be able to
perform the call to @code{eval} at compile time, and that would be
impossible unless all the references in the string to be evaluated could
also be resolved, and requiring that would be too restrictive (the
string might come from user input, or depend on things that are not
known until the function is evaluated).

Although Octave normally executes commands from script files that have
the name @file{@var{file}.m}, you can use the function @code{source} to
execute commands from any file.

@deftypefn {Built-in Function} {} source (@var{file})
Parse and execute the contents of @var{file}.  This is equivalent to
executing commands from a script file, but without requiring the file to
be named @file{@var{file}.m}.
@end deftypefn

@node Dynamically Linked Functions, Organization of Functions, Script Files, Functions and Scripts
@section Dynamically Linked Functions
@cindex dynamic linking

On some systems, Octave can dynamically load and execute functions
written in C++.  Octave can only directly call functions written in C++,
but you can also load functions written in other languages
by calling them from a simple wrapper function written in C++.

Here is an example of how to write a C++ function that Octave can load,
with commentary.  The source for this function is included in the source
distributions of Octave, in the file @file{examples/oregonator.cc}.  It
defines the same set of differential equations that are used in the
example problem of @ref{Ordinary Differential Equations}.  By running
that example and this one, we can compare the execution times to see
what sort of increase in speed you can expect by using dynamically
linked functions.

The function defined in @file{oregonator.cc} contains just 8 statements,
and is not much different than the code defined in the corresponding
M-file (also distributed with Octave in the file
@file{examples/oregonator.m}).

Here is the complete text of @file{oregonator.cc}:

just

@example
@group
#include <octave/oct.h>

DEFUN_DLD (oregonator, args, ,
  "The `oregonator'.")
@{
  ColumnVector dx (3);

  ColumnVector x = args(0).vector_value ();

  dx(0) = 77.27 * (x(1) - x(0)*x(1) + x(0)
                   - 8.375e-06*pow (x(0), 2));

  dx(1) = (x(2) - x(0)*x(1) - x(1)) / 77.27;

  dx(2) = 0.161*(x(0) - x(2));

  return octave_value (dx);
@}
@end group
@end example

The first line of the file,

@example
#include <octave/oct.h>
@end example

@noindent
includes declarations for all of Octave's internal functions that you
will need.  If you need other functions from the standard C++ or C
libraries, you can include the necessary headers here.

The next two lines
@example
@group
DEFUN_DLD (oregonator, args, ,
  "The `oregonator'.")
@end group
@end example

@noindent
declares the function.  The macro @code{DEFUN_DLD} and the macros that
it depends on are defined in the files @file{defun-dld.h},
@file{defun.h}, and @file{defun-int.h} (these files are included in the
header file @file{octave/oct.h}).

Note that the third parameter to @code{DEFUN_DLD} (@code{nargout}) is
not used, so it is omitted from the list of arguments to in order to
avoid  the warning from gcc about an unused function parameter.

@noindent
simply declares an object to store the right hand sides of the
differential equation, and

The statement

@example
ColumnVector x = args(0).vector_value ();
@end example

@noindent
extracts a column vector from the input arguments.  The variable
@code{args} is passed to functions defined with @code{DEFUN_DLD} as an
@code{octave_value_list} object, which includes methods for getting the
length of the list and extracting individual elements.

In this example, we don't check for errors, but that is not difficult.
All of the Octave's built-in functions do some form of checking on their
arguments, so you can check the source code for those functions for
examples of various strategies for verifying that the correct number and
types of arguments have been supplied.

The next statements

@example
@group
ColumnVector dx (3);

dx(0) = 77.27 * (x(1) - x(0)*x(1) + x(0)
                 - 8.375e-06*pow (x(0), 2));

dx(1) = (x(2) - x(0)*x(1) - x(1)) / 77.27;

dx(2) = 0.161*(x(0) - x(2));
@end group
@end example

@noindent
define the right hand side of the differential equation.  Finally, we
can return @code{dx}:

@example
return octave_value (dx);
@end example

@noindent
The actual return type is @code{octave_value_list}, but it is only
necessary to convert the return type to an @code{octave_value} because
there is a default constructor that can automatically create an object
of that type from an @code{octave_value} object, so we can just use that
instead.

To use this file, your version of Octave must support dynamic linking.
To find out if it does, type the command
@kbd{octave_config_info ("dld")} at the Octave prompt.  Support for
dynamic linking is included if this command returns 1.

To compile the example file, type the command @samp{mkoctfile
oregonator.cc} at the shell prompt.  The script @code{mkoctfile} should
have been installed along with Octave.  Running it will create a file
called @file{oregonator.oct} that can be loaded by Octave.  To test the
@file{oregonator.oct} file, start Octave and type the command

@example
oregonator ([1, 2, 3], 0)
@end example

@noindent
at the Octave prompt.  Octave should respond by printing

@example
@group
ans =

   77.269353
   -0.012942
   -0.322000
@end group
@end example

You can now use the @file{oregonator.oct} file just as you would the
@code{oregonator.m} file to solve the set of differential equations.

On a 133 MHz Pentium running Linux, Octave can solve the problem shown
in @ref{Ordinary Differential Equations} in about 1.4 second using the
dynamically linked function, compared to about 19 seconds using the
M-file.  Similar decreases in execution time can be expected for other
functions, particularly those that rely on functions like @code{lsode}
that require user-supplied functions.

Just as for M-files, Octave will automatically reload dynamically linked
functions when the files that define them are more recent than the last
time that the function was loaded.  Two variables are available to
control how Octave behaves when dynamically linked functions are cleared
or reloaded.

@defvr {Built-in Variable} auto_unload_dot_oct_files
If the value of @code{auto_unload_dot_oct_files} is nonzero, Octave will
automatically unload any @file{.oct} files when there are no longer any
functions in the symbol table that reference them.
@end defvr

@defvr {Built-in Variable} warn_reload_forces_clear
If several functions have been loaded from the same file, Octave must
clear all the functions before any one of them can be reloaded.  If
@code{warn_reload_forces_clear}, Octave will warn you when this happens,
and print a list of the additional functions that it is forced to clear.
@end defvr

Additional examples for writing dynamically linked functions are
available in the files in the @file{src} directory of the Octave
distribution.  Currently, this includes the files

@example
@group
balance.cc   fft2.cc      inv.cc       qzval.cc
chol.cc      filter.cc    log.cc       schur.cc
colloc.cc    find.cc      lsode.cc     sort.cc 
dassl.cc     fsolve.cc    lu.cc        svd.cc
det.cc       givens.cc    minmax.cc    syl.cc
eig.cc       hess.cc      pinv.cc      
expm.cc      ifft.cc      qr.cc     
fft.cc       ifft2.cc     quad.cc
@end group
@end example

@noindent
These files use the macro @code{DEFUN_DLD_BUILTIN} instead of
@code{DEFUN_DLD}.  The difference between these two macros is just that
@code{DEFUN_DLD_BUILTIN} can define a built-in function that is not
dynamically loaded if the operating system does not support dynamic
linking.  To define your own dynamically linked functions you should use
@code{DEFUN_DLD}.

There is currently no detailed description of all the functions that you
can call in a built-in function.  For the time being, you will have to
read the source code for Octave.

@node Organization of Functions,  , Dynamically Linked Functions, Functions and Scripts
@section Organization of Functions Distributed with Octave

Many of Octave's standard functions are distributed as function files.
They are loosely organized by topic, in subdirectories of
@file{@var{octave-home}/lib/octave/@var{version}/m}, to make it easier
to find them.

The following is a list of all the function file subdirectories, and the
types of functions you will find there.

@table @file
@item audio
Functions for playing and recording sounds.

@item control
Functions for design and simulation of automatic control systems.

@item elfun
Elementary functions.

@item general
Miscellaneous matrix manipulations, like @code{flipud}, @code{rot90},
and @code{triu}, as well as other basic functions, like
@code{is_matrix}, @code{nargchk}, etc.

@item image
Image processing tools.  These functions require the X Window System.

@item io
Input-ouput functions.

@item linear-algebra
Functions for linear algebra.

@item miscellaneous
Functions that don't really belong anywhere else.

@item plot
A set of functions that implement the @sc{Matlab}-like plotting functions.

@item polynomial
Functions for manipulating polynomials.

@item set
Functions for creating and manipulating sets of unique values.

@item signal
Functions for signal processing applications.

@item specfun
Special functions.

@item special-matrix
Functions that create special matrix forms.

@item startup
Octave's system-wide startup file.

@item statistics
Statistical functions.

@item strings
Miscellaneous string-handling functions.

@item time
Functions related to time keeping.
@end table
