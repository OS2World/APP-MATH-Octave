@c Copyright (C) 1996, 1997 John W. Eaton
@c This is part of the Octave manual.
@c For copying conditions, see the file gpl.tex.

@node Matrix Manipulation, Arithmetic, Plotting, Top
@chapter Matrix Manipulation

There are a number of functions available for checking to see if the
elements of a matrix meet some condition, and for rearranging the
elements of a matrix.  For example, Octave can easily tell you if all
the elements of a matrix are finite, or are less than some specified
value.  Octave can also rotate the elements, extract the upper- or
lower-triangular parts, or sort the columns of a matrix.

@menu
* Finding Elements and Checking Conditions::  
* Rearranging Matrices::        
* Special Utility Matrices::    
* Famous Matrices::             
@end menu

@node Finding Elements and Checking Conditions, Rearranging Matrices, Matrix Manipulation, Matrix Manipulation
@section Finding Elements and Checking Conditions

The functions @code{any} and @code{all} are useful for determining
whether any or all of the elements of a matrix satisfy some condition.
The @code{find} function is also useful in determining which elements of
a matrix meet a specified condition.

@deftypefn {Built-in Function} {} any (@var{x})
For a vector argument, return 1 if any element of the vector is
nonzero.

For a matrix argument, return a row vector of ones and
zeros with each element indicating whether any of the elements of the
corresponding column of the matrix are nonzero.  For example,

@example
@group
any (eye (2, 4))
     @result{} [ 1, 1, 0, 0 ]
@end group
@end example

To see if any of the elements of a matrix are nonzero, you can use a
statement like

@example
any (any (a))
@end example
@end deftypefn

@deftypefn {Built-in Function} {} all (@var{x})
The function @code{all} behaves like the function @code{any}, except
that it returns true only if all the elements of a vector, or all the
elements in a column of a matrix, are nonzero.
@end deftypefn

Since the comparison operators (@pxref{Comparison Ops}) return matrices
of ones and zeros, it is easy to test a matrix for many things, not just
whether the elements are nonzero.  For example, 

@example
@group
all (all (rand (5) < 0.9))
     @result{} 0
@end group
@end example

@noindent
tests a random 5 by 5 matrix to see if all of its elements are less
than 0.9.

Note that in conditional contexts (like the test clause of @code{if} and
@code{while} statements) Octave treats the test as if you had typed
@code{all (all (condition))}.

@deftypefn {Function File} {[@var{err}, @var{y1}, ...] =} common_size (@var{x1}, ...)
Determine if all input arguments are either scalar or of common
size.  If so, @var{err} is zero, and @var{yi} is a matrix of the
common size with all entries equal to @var{xi} if this is a scalar or
@var{xi} otherwise.  If the inputs cannot be brought to a common size,
errorcode is 1, and @var{yi} is @var{xi}.  For example,

@example
@group
[errorcode, a, b] = common_size ([1 2; 3 4], 5)
     @result{} errorcode = 0
     @result{} a = [ 1, 2; 3, 4 ]
     @result{} b = [ 5, 5; 5, 5 ]
@end group
@end example

@noindent
This is useful for implementing functions where arguments can either
be scalars or of common size.
@end deftypefn

@deftypefn {Function File} {} diff (@var{x}, @var{k})
If @var{x} is a vector of length @var{n}, @code{diff (@var{x})} is the
vector of first differences
@iftex
@tex
 $x_2 - x_1, \ldots{}, x_n - x_{n-1}$.
@end tex
@end iftex
@ifinfo
 @var{x}(2) - @var{x}(1), @dots{}, @var{x}(n) - @var{x}(n-1).
@end ifinfo

If @var{x} is a matrix, @code{diff (@var{x})} is the matrix of column
differences.

The second argument is optional.  If supplied, @code{diff (@var{x},
@var{k})}, where @var{k} is a nonnegative integer, returns the
@var{k}-th differences.
@end deftypefn

@deftypefn {Mapping Function} {} isinf (@var{x})
Return 1 for elements of @var{x} that are infinite and zero
otherwise. For example,

@example
@group
isinf ([13, Inf, NaN])
     @result{} [ 0, 1, 0 ]
@end group
@end example
@end deftypefn

@deftypefn {Mapping Function} {} isnan (@var{x})
Return 1 for elements of @var{x} that are NaN values and zero
otherwise. For example,

@example
@group
isnan ([13, Inf, NaN])
     @result{} [ 0, 0, 1 ]
@end group
@end example
@end deftypefn

@deftypefn {Mapping Function} {} finite (@var{x})
Return 1 for elements of @var{x} that are NaN values and zero
otherwise. For example,

@example
@group
finite ([13, Inf, NaN])
     @result{} [ 1, 0, 0 ]
@end group
@end example
@end deftypefn

@deftypefn {Loadable Function} {} find (@var{x})
Return a vector of indices of nonzero elements of a matrix.  To obtain a
single index for each matrix element, Octave pretends that the columns
of a matrix form one long vector (like Fortran arrays are stored).  For
example,

@example
@group
find (eye (2))
     @result{} [ 1; 4 ]
@end group
@end example

If two outputs are requested, @code{find} returns the row and column
indices of nonzero elements of a matrix.  For example,

@example
@group
[i, j] = find (2 * eye (2))
     @result{} i = [ 1; 2 ]
     @result{} j = [ 1; 2 ]
@end group
@end example

If three outputs are requested, @code{find} also returns a vector
containing the nonzero values.  For example,

@example
@group
[i, j, v] = find (3 * eye (2))
     @result{} i = [ 1; 2 ]
     @result{} j = [ 1; 2 ]
     @result{} v = [ 3; 3 ]
@end group
@end example
@end deftypefn
        
@node Rearranging Matrices, Special Utility Matrices, Finding Elements and Checking Conditions, Matrix Manipulation
@section Rearranging Matrices

@deftypefn {Function File} {} fliplr (@var{x})
Return a copy of @var{x} with the order of the columns reversed.  For
example, 

@example
@group
fliplr ([1, 2; 3, 4])
     @result{}  2  1
         4  3
@end group
@end example
@end deftypefn

@deftypefn {Function File} {} flipud (@var{x})
Return a copy of @var{x} with the order of the rows reversed.  For
example,

@example
@group
flipud ([1, 2; 3, 4])
     @result{}  3  4
         1  2
@end group
@end example
@end deftypefn

@deftypefn {Function File} {} rot90 (@var{x}, @var{n})
Return a copy of @var{x} with the elements rotated counterclockwise in
90-degree increments.  The second argument is optional, and specifies
how many 90-degree rotations are to be applied (the default value is 1).
Negative values of @var{n} rotate the matrix in a clockwise direction.
For example,

@example
@group
rot90 ([1, 2; 3, 4], -1)
     @result{}  3  1
         4  2
@end group
@end example

@noindent
rotates the given matrix clockwise by 90 degrees.  The following are all
equivalent statements:

@example
@group
rot90 ([1, 2; 3, 4], -1)
@equiv{}
rot90 ([1, 2; 3, 4], 3)
@equiv{}
rot90 ([1, 2; 3, 4], 7)
@end group
@end example
@end deftypefn

@deftypefn {Function File} {} reshape (@var{a}, @var{m}, @var{n})
Return a matrix with @var{m} rows and @var{n} columns whose elements are
taken from the matrix @var{a}.  To decide how to order the elements,
Octave pretends that the elements of a matrix are stored in column-major
order (like Fortran arrays are stored).

For example,

@example
@group
reshape ([1, 2, 3, 4], 2, 2)
     @result{}  1  3
         2  4
@end group
@end example

If the variable @code{do_fortran_indexing} is nonzero, the
@code{reshape} function is equivalent to

@example
@group
retval = zeros (m, n);
retval (:) = a;
@end group
@end example

@noindent
but it is somewhat less cryptic to use @code{reshape} instead of the
colon operator.  Note that the total number of elements in the original
matrix must match the total number of elements in the new matrix.
@end deftypefn

@deftypefn {Function File} {} shift (@var{x}, @var{b})
If @var{x} is a vector, perform a circular shift of length @var{b} of
the elements of @var{x}.

If @var{x} is a matrix, do the same for each column of @var{x}.
@end deftypefn

@deftypefn {Loadable Function} {[@var{s}, @var{i}] =} sort (@var{x})
Return a copy of @var{x} with the elements elements arranged in
increasing order.  For matrices, @code{sort} orders the elements in each
column.

For example,

@example
@group
sort ([1, 2; 2, 3; 3, 1])
     @result{}  1  1
         2  2
         3  3
@end group
@end example

The @code{sort} function may also be used to produce a matrix
containing the original row indices of the elements in the sorted
matrix.  For example,

@example
@group
[s, i] = sort ([1, 2; 2, 3; 3, 1])
     @result{} s = 1  1
            2  2
            3  3
     @result{} i = 1  3
            2  1
            3  2
@end group
@end example
@end deftypefn

Since the @code{sort} function does not allow sort keys to be specified,
it can't be used to order the rows of a matrix according to the values
of the elements in various columns@footnote{For example, to first sort
based on the values in column 1, and then, for any values that are
repeated in column 1, sort based on the values found in column 2, etc.}
in a single call.  Using the second output, however, it is possible to
sort all rows based on the values in a given column.  Here's an example
that sorts the rows of a matrix based on the values in the second
column.

@example
@group
a = [1, 2; 2, 3; 3, 1];
[s, i] = sort (a (:, 2));
a (i, :)
     @result{}  3  1
         1  2
         2  3
@end group
@end example

@deftypefn {Function File} {} tril (@var{a}, @var{k})
@deftypefnx {Function File} {} triu (@var{a}, @var{k})
Return a new matrix formed by extracting extract the lower (@code{tril})
or upper (@code{triu}) triangular part of the matrix @var{a}, and
setting all other elements to zero.  The second argument is optional,
and specifies how many diagonals above or below the main diagonal should
also be set to zero.

The default value of @var{k} is zero, so that @code{triu} and
@code{tril} normally include the main diagonal as part of the result
matrix.

If the value of @var{k} is negative, additional elements above (for
@code{tril}) or below (for @code{triu}) the main diagonal are also
selected.

The absolute value of @var{k} must not be greater than the number of
sub- or super-diagonals.

For example,

@example
@group
tril (ones (3), -1)
     @result{}  0  0  0
         1  0  0
         1  1  0
@end group
@end example

@noindent
and

@example
@group
tril (ones (3), 1)
     @result{}  1  1  0
         1  1  1
         1  1  1
@end group
@end example
@end deftypefn

@deftypefn {Function File} {} vec (@var{x})
Return the vector obtained by stacking the columns of the matrix @var{x}
one above the other.
@end deftypefn

@deftypefn {Function File} {} vech (@var{x})
Return the vector obtained by eliminating all supradiagonal elements of
the square matrix @var{x} and stacking the result one column above the
other.
@end deftypefn

@node Special Utility Matrices, Famous Matrices, Rearranging Matrices, Matrix Manipulation
@section Special Utility Matrices

@deftypefn {Built-in Function} {} eye (@var{x})
@deftypefnx {Built-in Function} {} eye (@var{n}, @var{m})
Return an identity matrix.  If invoked with a single scalar argument,
@code{eye} returns a square matrix with the dimension specified.  If you
supply two scalar arguments, @code{eye} takes them to be the number of
rows and columns.  If given a vector with two elements, @code{eye} uses
the values of the elements as the number of rows and columns,
respectively.  For example,

@example
@group
eye (3)
     @result{}  1  0  0
         0  1  0
         0  0  1
@end group
@end example

The following expressions all produce the same result:

@example
@group
eye (2)
@equiv{}
eye (2, 2)
@equiv{}
eye (size ([1, 2; 3, 4])
@end group
@end example

For compatibility with @sc{Matlab}, calling @code{eye} with no arguments
is equivalent to calling it with an argument of 1.
@end deftypefn

@deftypefn {Built-in Function} {} ones (@var{x})
@deftypefnx {Built-in Function} {} ones (@var{n}, @var{m})
Return a matrix whose elements are all 1.  The arguments are handled
the same as the arguments for @code{eye}.

If you need to create a matrix whose values are all the same, you should
use an expression like

@example
val_matrix = val * ones (n, m)
@end example
@end deftypefn

@deftypefn {Built-in Function} {} zeros (@var{x})
@deftypefnx {Built-in Function} {} zeros (@var{n}, @var{m})
Return a matrix whose elements are all 0.  The arguments are handled
the same as the arguments for @code{eye}.
@end deftypefn

@deftypefn {Loadable Function} {} rand (@var{x})
@deftypefnx {Loadable Function} {} rand (@var{n}, @var{m})
@deftypefnx {Loadable Function} {} rand (@code{"seed"}, @var{x})
Return a matrix with random elements uniformly distributed on the
interval (0, 1).  The arguments are handled the same as the arguments
for @code{eye}.  In
addition, you can set the seed for the random number generator using the
form

@example
rand ("seed", @var{x})
@end example

@noindent
where @var{x} is a scalar value.  If called as

@example
rand ("seed")
@end example

@noindent
@code{rand} returns the current value of the seed.
@end deftypefn

@deftypefn {Loadable Function} {} randn (@var{x})
@deftypefnx {Loadable Function} {} randn (@var{n}, @var{m})
@deftypefnx {Loadable Function} {} randn (@code{"seed"}, @var{x})
Return a matrix with normally distributed random elements.  The
arguments are handled the same as the arguments for @code{eye}.  In
addition, you can set the seed for the random number generator using the
form

@example
randn ("seed", @var{x})
@end example

@noindent
where @var{x} is a scalar value.  If called as

@example
randn ("seed")
@end example

@noindent
@code{randn} returns the current value of the seed.
@end deftypefn

The @code{rand} and @code{randn} functions use separate generators.
This ensures that

@example
@group
rand ("seed", 13);
randn ("seed", 13);
u = rand (100, 1);
n = randn (100, 1);
@end group
@end example

@noindent
and

@example
@group
rand ("seed", 13);
randn ("seed", 13);
u = zeros (100, 1);
n = zeros (100, 1);
for i = 1:100
  u(i) = rand ();
  n(i) = randn ();
end
@end group
@end example

@noindent
produce equivalent results.

Normally, @code{rand} and @code{randn} obtain their initial
seeds from the system clock, so that the sequence of random numbers is
not the same each time you run Octave.  If you really do need for to
reproduce a sequence of numbers exactly, you can set the seed to a
specific value.

If it is invoked without arguments, @code{rand} and @code{randn} return a
single element of a random sequence.

The @code{rand} and @code{randn} functions use Fortran code from
@sc{Ranlib}, a library of fortran routines for random number generation,
compiled by Barry W. Brown and James Lovato of the Department of
Biomathematics at The University of Texas, M.D. Anderson Cancer Center,
Houston, TX 77030.

@deftypefn {Built-in Function} {} diag (@var{v}, @var{k})
Return a diagonal matrix with vector @var{v} on diagonal @var{k}.  The
second argument is optional.  If it is positive, the vector is placed on
the @var{k}-th super-diagonal.  If it is negative, it is placed on the
@var{-k}-th sub-diagonal.  The default value of @var{k} is 0, and the
vector is placed on the main diagonal.  For example,

@example
@group
diag ([1, 2, 3], 1)
     @result{}  0  1  0  0
         0  0  2  0
         0  0  0  3
         0  0  0  0
@end group
@end example
@end deftypefn

@c XXX FIXME XXX -- is this really worth documenting?
@c
@c @defvr {Built-in Variable} ok_to_lose_imaginary_part
@c If the value of @code{ok_to_lose_imaginary_part} is nonzero, implicit
@c conversions of complex numbers to real numbers are allowed (for example,
@c by fsolve).  If the value is @code{"warn"}, the conversion is allowed,
@c but a warning is printed.  Otherwise, an error message is printed and
@c control is returned to the top level.  The default value is
@c @code{"warn"}.
@c 
@c XXX FIXME XXX -- this is here because it is used by @code{ones},
@c @code{zeros}, @code{rand}, etc.
@c @end defvr

The functions @code{linspace} and @code{logspace} make it very easy to
create vectors with evenly or logarithmically spaced elements.
@xref{Ranges}.

@deftypefn {Function File} {} linspace (@var{base}, @var{limit}, @var{n})
Return a row vector with @var{n} linearly spaced elements between
@var{base} and @var{limit}.  The number of elements, @var{n}, must be
greater than 1.  The @var{base} and @var{limit} are always included in
the range.  If @var{base} is greater than @var{limit}, the elements are
stored in decreasing order.  If the number of points is not specified, a
value of 100 is used.

The @code{linspace} function always returns a row vector, regardless of
the value of @code{prefer_column_vectors}.
@end deftypefn

@deftypefn {Function File} {} logspace (@var{base}, @var{limit}, @var{n})
Similar to @code{linspace} except that the values are logarithmically
spaced from
@iftex
@tex
$10^{base}$ to $10^{limit}$.
@end tex
@end iftex
@ifinfo
10^base to 10^limit.
@end ifinfo

If @var{limit} is equal to
@iftex
@tex
$\pi$,
@end tex
@end iftex
@ifinfo
pi,
@end ifinfo
the points are between
@iftex
@tex
$10^{base}$ and $\pi$,
@end tex
@end iftex
@ifinfo
10^base and pi,
@end ifinfo
@emph{not}
@iftex
@tex
$10^{base}$ and $10^{\pi}$,
@end tex
@end iftex
@ifinfo
10^base and 10^pi,
@end ifinfo
in order to  be compatible with the corresponding @sc{Matlab} function.
@end deftypefn

@defvr {Built-in Variable} treat_neg_dim_as_zero
If the value of @code{treat_neg_dim_as_zero} is nonzero, expressions
like

@example
eye (-1)
@end example

@noindent
produce an empty matrix (i.e., row and column dimensions are zero).
Otherwise, an error message is printed and control is returned to the
top level.  The default value is 0.
@end defvr

@node Famous Matrices,  , Special Utility Matrices, Matrix Manipulation
@section Famous Matrices

The following functions return famous matrix forms.

@deftypefn {Function File} {} hankel (@var{c}, @var{r})
Return the Hankel matrix constructed given the first column @var{c}, and
(optionally) the last row @var{r}.  If the last element of @var{c} is
not the same as the first element of @var{r}, the last element of
@var{c} is used.  If the second argument is omitted, the last row is
taken to be the same as the first column.

A Hankel matrix formed from an m-vector @var{c}, and an n-vector
@var{r}, has the elements
@iftex
@tex
$$
H (i, j) = \cases{c_{i+j-1},&$i+j-1\le m$;\cr r_{i+j-m},&otherwise.\cr}
$$
@end tex
@end iftex
@ifinfo

@example
@group
H (i, j) = c (i+j-1),  i+j-1 <= m;
H (i, j) = r (i+j-m),  otherwise
@end group
@end example
@end ifinfo
@end deftypefn

@deftypefn {Function File} {} hilb (@var{n})
Return the Hilbert matrix of order @var{n}.  The
@iftex
@tex
$i,\,j$
@end tex
@end iftex
@ifinfo
i, j
@end ifinfo
element of a Hilbert matrix is defined as
@iftex
@tex
$$
H (i, j) = {1 \over (i + j - 1)}
$$
@end tex
@end iftex
@ifinfo

@example
H (i, j) = 1 / (i + j - 1)
@end example
@end ifinfo
@end deftypefn

@deftypefn {Function File} {} invhilb (@var{n})
Return the inverse of a Hilbert matrix of order @var{n}.  This is exact.
Compare with the numerical calculation of @code{inverse (hilb (n))},
which suffers from the ill-conditioning of the Hilbert matrix, and the
finite precision of your computer's floating point arithmetic.
@end deftypefn

@deftypefn {Function File} {} sylvester_matrix (@var{k})
Return the Sylvester matrix of order
@iftex
@tex
$n = 2^k$.
@end tex
@end iftex
@ifinfo
n = 2^k.
@end ifinfo
@end deftypefn

@deftypefn {Function File} {} toeplitz (@var{c}, @var{r})
Return the Toeplitz matrix constructed given the first column @var{c},
and (optionally) the first row @var{r}.  If the first element of @var{c}
is not the same as the first element of @var{r}, the first element of
@var{c} is used.  If the second argument is omitted, the first row is
taken to be the same as the first column.

A square Toeplitz matrix has the form
@iftex
@tex
$$
\left[\matrix{c_0    & r_1     & r_2      & \ldots & r_n\cr
              c_1    & c_0     & r_1      &        & c_{n-1}\cr
              c_2    & c_1     & c_0      &        & c_{n-2}\cr
              \vdots &         &          &        & \vdots\cr
              c_n    & c_{n-1} & c_{n-2} & \ldots & c_0}\right].
$$
@end tex
@end iftex
@ifinfo

@example
@group
c(0)  r(1)   r(2)  ...  r(n)
c(1)  c(0)   r(1)      r(n-1)
c(2)  c(1)   c(0)      r(n-2)
 .                       .
 .                       .
 .                       .

c(n) c(n-1) c(n-2) ...  c(0)
@end group
@end example
@end ifinfo
@end deftypefn

@deftypefn {Function File} {} vander (@var{c})
Return the Vandermonde matrix whose next to last column is @var{c}.

A Vandermonde matrix has the form
@iftex
@tex
$$
\left[\matrix{c_0^n  & \ldots & c_0^2  & c_0    & 1\cr
              c_1^n  & \ldots & c_1^2  & c_1    & 1\cr
              \vdots &        & \vdots & \vdots & \vdots\cr
              c_n^n  & \ldots & c_n^2  & c_n    & 1}\right].
$$
@end tex
@end iftex
@ifinfo

@example
@group
c(0)^n ... c(0)^2  c(0)  1
c(1)^n ... c(1)^2  c(1)  1
 .           .      .    .
 .           .      .    .
 .           .      .    .
                 
c(n)^n ... c(n)^2  c(n)  1
@end group
@end example
@end ifinfo
@end deftypefn
