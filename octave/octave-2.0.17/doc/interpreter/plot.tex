@c Copyright (C) 1996, 1997 John W. Eaton
@c This is part of the Octave manual.
@c For copying conditions, see the file gpl.tex.

@node Plotting, Matrix Manipulation, Input and Output, Top
@chapter Plotting

All of Octave's plotting functions use @code{gnuplot} to handle the
actual graphics.  There are two low-level functions, @code{gplot} and
@code{gsplot}, that behave almost exactly like the corresponding
@code{gnuplot} functions @code{plot} and @code{splot}.  A number of
other higher level plotting functions, patterned after the graphics
functions found in @sc{Matlab} version 3.5, are also available.
These higher level functions are all implemented in terms of the two
low-level plotting functions.

@menu
* Two-Dimensional Plotting::    
* Specialized Two-Dimensional Plots::  
* Three-Dimensional Plotting::  
* Plot Annotations::            
* Multiple Plots on One Page::  
@end menu

@node Two-Dimensional Plotting, Specialized Two-Dimensional Plots, Plotting, Plotting
@section Two-Dimensional Plotting

@deffn {Command} gplot @var{ranges} @var{expression} @var{using} @var{title} @var{style}
Generate a 2-dimensional plot.

The @var{ranges}, @var{using}, @var{title}, and @var{style} arguments
are optional, and the @var{using}, @var{title} and @var{style}
qualifiers may appear in any order after the expression.  You may plot
multiple expressions with a single command by separating them with
commas.  Each expression may have its own set of qualifiers.

The optional item @var{ranges} has the syntax

@example
[ x_lo : x_up ] [ y_lo : y_up ]
@end example

@noindent
and may be used to specify the ranges for the axes of the plot,
independent of the actual range of the data.  The range for the y axes
and any of the individual limits may be omitted.  A range @code{[:]}
indicates that the default limits should be used.  This normally means
that a range just large enough to include all the data points will be
used.

The expression to be plotted must not contain any literal matrices
(e.g. @code{[ 1, 2; 3, 4 ]}) since it is nearly impossible to
distinguish a plot range from a matrix of data.

See the help for @code{gnuplot} for a description of the syntax for the
optional items.

By default, the @code{gplot} command plots the second column of a matrix
versus the first.  If the matrix only has one column, it is taken as a
vector of y-coordinates and the x-coordinate is taken as the element
index, starting with zero.  For example,

@example
gplot rand (100,1) with linespoints
@end example

@noindent
will plot 100 random values and connect them with lines.  When
@code{gplot} is used to plot a column vector, the indices of the
elements are taken as x values.

  If there are more than two columns, you can
choose which columns to plot with the @var{using} qualifier. For
example, given the data

@example
x = (-10:0.1:10)';
data = [x, sin(x), cos(x)];
@end example

@noindent
the command

@example
gplot [-11:11] [-1.1:1.1] \
  data with lines, data using 1:3 with impulses
@end example

@noindent
will plot two lines.  The first line is generated by the command
@code{data with lines}, and is a graph of the sine function over the
range @minus{}10 to 10.  The data is taken from the first two columns of
the matrix because columns to plot were not specified with the
@var{using} qualifier.

The clause @code{using 1:3} in the second part of this plot command
specifies that the first and third columns of the matrix @code{data}
should be taken as the values to plot.

In this example, the ranges have been explicitly specified to be a bit
larger than the actual range of the data so that the curves do not touch
the border of the plot.
@end deffn

@deffn {Command} gset options
@deffnx {Command} gshow options
@deffnx {Command} replot options
In addition to the basic plotting commands, the whole range of
@code{gset} and @code{gshow} commands from @code{gnuplot} are available,
as is @code{replot}.

@findex set
@findex show
Note that in Octave 2.0, the @code{set} and @code{show} commands were
renamed to @code{gset} and @code{gshow} in order to allow for
compatibility with the @sc{Matlab} graphics and GUI commands in a future
version of Octave.  (For now, the old @code{set} and @code{show}
commands do work, but they print an annoying warning message to try to
get people to switch to using @code{gset} and @code{gshow}.)

The @code{gset} and @code{gshow} commands allow you to set and show
@code{gnuplot} parameters.  For more information about the @code{gset}
and @code{gshow} commands, see the documentation for @code{set} and
@code{show} in the @code{gnuplot} user's guide (also available on line
if you run @code{gnuplot} directly, instead of running it from Octave).

The @code{replot} command allows you to force the plot to be
redisplayed.  This is useful if you have changed something about the
plot, such as the title or axis labels.  The @code{replot} command also
accepts the same arguments as @code{gplot} or @code{gsplot} (except for
data ranges) so you can add additional lines to existing plots.  

For example,

@example
gset term tek40
gset output "/dev/plotter"
gset title "sine with lines and cosine with impulses"
replot "sin (x) w l"
@end example

will change the terminal type for plotting, add a title to the current
plot, add a graph of
@iftex
@tex
$\sin(x)$
@end tex
@end iftex
@ifinfo
sin (x) 
@end ifinfo
to the plot, and force the new plot to be
sent to the plot device.  This last step is normally required in order
to update the plot.  This default is reasonable for slow terminals or
hardcopy output devices because even when you are adding additional
lines with a replot command, gnuplot always redraws the entire plot, and
you probably don't want to have a completely new plot generated every
time something as minor as an axis label changes.

@findex shg
The command @code{shg} is equivalent to executing @code{replot} without
any arguments.
@end deffn

@defvr {Built-in Variable} automatic_replot
You can tell Octave to redisplay the plot each time anything about it
changes by setting the value of the builtin variable
@code{automatic_replot} to a nonzero value.  Since this is fairly
inefficient, the default value is 0.
@end defvr

Note that NaN values in the plot data are automatically omitted, and
Inf values are converted to a very large value before calling gnuplot.

@c XXX FIXME XXX -- add info about what to do to get plots on remote X
@c terminals.  People often forget how to properly set DISPLAY and run
@c xhost.

@c XXX FIXME XXX -- add info about getting paper copies of plots.

The @sc{Matlab}-style two-dimensional plotting commands are:

@cindex plotting
@cindex graphics

@deftypefn {Function File} {} plot (@var{args})
This function produces two-dimensional plots.  Many different
combinations of arguments are possible.  The simplest form is

@example
plot (@var{y})
@end example

@noindent
where the argument is taken as the set of @var{y} coordinates and the
@var{x} coordinates are taken to be the indices of the elements,
starting with 1.

If more than one argument is given, they are interpreted as

@example
plot (@var{x}, @var{y}, @var{fmt} ...)
@end example

@noindent
where @var{y} and @var{fmt} are optional, and any number of argument
sets may appear.  The @var{x} and @var{y} values are
interpreted as follows:

@itemize @bullet
@item
If a single data argument is supplied, it is taken as the set of @var{y}
coordinates and the @var{x} coordinates are taken to be the indices of
the elements, starting with 1.

@item
If the first argument is a vector and the second is a matrix, the
the vector is plotted versus the columns (or rows) of the matrix.
(using whichever combination matches, with columns tried first.)

@item
If the first argument is a matrix and the second is a vector, the
the columns (or rows) of the matrix are plotted versus the vector.
(using whichever combination matches, with columns tried first.)

@item
If both arguments are vectors, the elements of @var{y} are plotted versus
the elements of @var{x}.

@item
If both arguments are matrices, the columns of @var{y} are plotted
versus the columns of @var{x}.  In this case, both matrices must have
the same number of rows and columns and no attempt is made to transpose
the arguments to make the number of rows match.

If both arguments are scalars, a single point is plotted.
@end itemize

The @var{fmt} argument, if present is interpreted as follows.  If
@var{fmt} is missing, the default gnuplot line style is assumed.

@table @samp
@item -
Set lines plot style (default).

@item .
Set dots plot style.

@item @@
Set points plot style.

@item -@@
Set linespoints plot style.

@item ^
Set impulses plot style.

@item L
Set steps plot style.

@item #
Set boxes plot style.

@item ~
Set errorbars plot style.

@item #~
Set boxerrorbars plot style.

@item @var{n}
Interpreted as the plot color if @var{n} is an integer in the range 1 to
6.

@item @var{nm}
If @var{nm} is a two digit integer and @var{m} is an integer in the
range 1 to 6, @var{m} is interpreted as the point style.  This is only
valid in combination with the @code{@@} or @code{-@@} specifiers.

@item @var{c}
If @var{c} is one of @code{"r"}, @code{"g"}, @code{"b"}, @code{"m"},
@code{"c"}, or @code{"w"}, it is interpreted as the plot color (red,
green, blue, magenta, cyan, or white).

@item +
@itemx *
@itemx o
@itemx x
Used in combination with the points or linespoints styles, set the point
style.
@end table

The color line styles have the following meanings on terminals that
support color.

@example
Number  Gnuplot colors  (lines)points style
  1       red                   *
  2       green                 +
  3       blue                  o
  4       magenta               x
  5       cyan                house
  6       brown            there exists
@end example

Here are some plot examples:

@example
plot (x, y, "@@12", x, y2, x, y3, "4", x, y4, "+")
@end example

This command will plot @code{y} with points of type 2 (displayed as
@samp{+}) and color 1 (red), @code{y2} with lines, @code{y3} with lines of
color 4 (magenta) and @code{y4} with points displayed as @samp{+}.

@example
plot (b, "*")
@end example

This command will plot the data in the variable @code{b} will be plotted
with points displayed as @samp{*}.
@end deftypefn

@deftypefn {Function File} {} hold @var{args}
Tell Octave to `hold' the current data on the plot when executing
subsequent plotting commands.  This allows you to execute a series of
plot commands and have all the lines end up on the same figure.  The
default is for each new plot command to clear the plot device first.
For example, the command

@example
hold on
@end example

@noindent
turns the hold state on.  An argument of @code{off} turns the hold state
off, and @code{hold} with no arguments toggles the current hold state.
@end deftypefn

@deftypefn {Function File} {} ishold
Return 1 if the next line will be added to the current plot, or 0 if
the plot device will be cleared before drawing the next line.
@end deftypefn

@deftypefn {Function File} {} clearplot
@deftypefnx {Function File} {} clg
Clear the plot window and any titles or axis labels.  The name
@code{clg} is aliased to @code{clearplot} for compatibility with @sc{Matlab}.

The commands @kbd{gplot clear}, @kbd{gsplot clear}, and @kbd{replot
clear} are equivalent to @code{clearplot}.  (Previously, commands like
@kbd{gplot clear} would evaluate @code{clear} as an ordinary expression
and clear all the visible variables.)
@end deftypefn

@deftypefn {Function File} {} closeplot
Close stream to the @code{gnuplot} subprocess.  If you are using X11,
this will close the plot window.
@end deftypefn

@deftypefn {Function File} {} purge_tmp_files
Delete the temporary files created by the plotting commands.

Octave creates temporary data files for @code{gnuplot} and then sends
commands to @code{gnuplot} through a pipe.  Octave will delete the
temporary files on exit, but if you are doing a lot of plotting you may
want to clean up in the middle of a session.

A future version of Octave will eliminate the need to use temporary
files to hold the plot data.
@end deftypefn

@deftypefn {Function File} {} axis (@var{limits})
Sets the axis limits for plots.

The argument @var{limits} should be a 2, 4, or 6 element vector.  The
first and second elements specify the lower and upper limits for the x
axis.  The third and fourth specify the limits for the y axis, and the
fifth and sixth specify the limits for the z axis.

With no arguments, @code{axis} turns autoscaling on.

If your plot is already drawn, then you need to use @code{replot} before
the new axis limits will take effect.  You can get this to happen
automatically by setting the built-in variable @code{automatic_replot}
to a nonzero value.
@end deftypefn

@node Specialized Two-Dimensional Plots, Three-Dimensional Plotting, Two-Dimensional Plotting, Plotting
@section Specialized Two-Dimensional Plots

@deftypefn {Function File} {} bar (@var{x}, @var{y})
Given two vectors of x-y data, @code{bar} produces a bar graph.

If only one argument is given, it is taken as a vector of y-values
and the x coordinates are taken to be the indices of the elements.

If two output arguments are specified, the data are generated but
not plotted.  For example,

@example
bar (x, y);
@end example

@noindent
and

@example
[xb, yb] = bar (x, y);
plot (xb, yb);
@end example

@noindent
are equivalent.
@end deftypefn

@deftypefn {Function File} {} contour (@var{z}, @var{n}, @var{x}, @var{y})
Make a contour plot of the three-dimensional surface described by
@var{z}.  Someone needs to improve @code{gnuplot}'s contour routines
before this will be very useful.
@end deftypefn

@deftypefn {Function File} {} hist (@var{y}, @var{x})
Produce histogram counts or plots.

With one vector input argument, plot a histogram of the values with
10 bins.  The range of the histogram bins is determined by the range
of the data.

Given a second scalar argument, use that as the number of bins.

Given a second vector argument, use that as the centers of the bins,
with the width of the bins determined from the adjacent values in
the vector.

Extreme values are lumped in the first and last bins.

With two output arguments, produce the values @var{nn} and @var{xx} such
that @code{bar (@var{xx}, @var{nn})} will plot the histogram.
@end deftypefn

@deftypefn {Function File} {} loglog (@var{args})
Make a two-dimensional plot using log scales for both axes.  See the
description of @code{plot} above for a description of the arguments that
@code{loglog} will accept.
@end deftypefn

@deftypefn {Function File} {} polar (@var{theta}, @var{rho})
Make a two-dimensional plot given polar the coordinates @var{theta} and
@var{rho}.
@end deftypefn

@deftypefn {Function File} {} semilogx (@var{args})
Make a two-dimensional plot using a log scale for the @var{x} axis.  See
the description of @code{plot} above for a description of the arguments
that @code{semilogx} will accept.
@end deftypefn

@deftypefn {Function File} {} semilogy (@var{args})
Make a two-dimensional plot using a log scale for the @var{y} axis.  See
the description of @code{plot} above for a description of the arguments
that @code{semilogy} will accept.
@end deftypefn

@deftypefn {Function File} {} stairs (@var{x}, @var{y})
Given two vectors of x-y data, bar produces a `stairstep' plot.

If only one argument is given, it is taken as a vector of y-values
and the x coordinates are taken to be the indices of the elements.

If two output arguments are specified, the data are generated but
not plotted.  For example,

@example
stairs (x, y);
@end example

@noindent
and

@example
[xs, ys] = stairs (x, y);
plot (xs, ys);
@end example

@noindent
are equivalent.
@end deftypefn

@node Three-Dimensional Plotting, Plot Annotations, Specialized Two-Dimensional Plots, Plotting
@section Three-Dimensional Plotting

@deffn {Command} gsplot @var{ranges} @var{expression} @var{using} @var{title} @var{style}
Generate a 3-dimensional plot.

The @var{ranges}, @var{using}, @var{title}, and @var{style} arguments
are optional, and the @var{using}, @var{title} and @var{style}
qualifiers may appear in any order after the expression.  You may plot
multiple expressions with a single command by separating them with
commas.  Each expression may have its own set of qualifiers.

The optional item @var{ranges} has the syntax

@example
[ x_lo : x_up ] [ y_lo : y_up ] [ z_lo : z_up ]
@end example

@noindent
and may be used to specify the ranges for the axes of the plot,
independent of the actual range of the data.  The range for the y and z
axes and any of the individual limits may be omitted.  A range
@code{[:]} indicates that the default limits should be used.  This
normally means that a range just large enough to include all the data
points will be used.

The expression to be plotted must not contain any literal matrices (e.g.
@code{[ 1, 2; 3, 4 ]}) since it is nearly impossible to distinguish a
plot range from a matrix of data.

See the help for @code{gnuplot} for a description of the syntax for the
optional items.

By default, the @code{gsplot} command plots each column of the
expression as the z value, using the row index as the x value, and the
column index as the y value.  The indices are counted from zero, not
one.  For example,

@example
gsplot rand (5, 2)
@end example

@noindent
will plot a random surface, with the x and y values taken from the row
and column indices of the matrix.

If parametric plotting mode is set (using the command
@kbd{gset parametric}, then @code{gsplot} takes the columns of the
matrix three at a time as the x, y and z values that define a line in
three space.  Any extra columns are ignored, and the x and y values are
expected to be sorted.  For example, with @code{parametric} set, it
makes sense to plot a matrix like
@iftex
@tex
$$
\left[\matrix{
1 & 1 & 3 & 2 & 1 & 6 & 3 & 1 & 9 \cr
1 & 2 & 2 & 2 & 2 & 5 & 3 & 2 & 8 \cr
1 & 3 & 1 & 2 & 3 & 4 & 3 & 3 & 7}\right]
$$
@end tex
@end iftex
@ifinfo

@example
1 1 3 2 1 6 3 1 9
1 2 2 2 2 5 3 2 8
1 3 1 2 3 4 3 3 7
@end example
@end ifinfo

@noindent
but not @code{rand (5, 30)}.
@end deffn

The @sc{Matlab}-style three-dimensional plotting commands are:

@deftypefn {Function File} {} mesh (@var{x}, @var{y}, @var{z})
Plot a mesh given matrices @code{x}, and @var{y} from @code{meshdom} and
a matrix @var{z} corresponding to the @var{x} and @var{y} coordinates of
the mesh.
@end deftypefn

@deftypefn {Function File} {} meshdom (@var{x}, @var{y})
Given vectors of @var{x} and @var{y} coordinates, return two matrices
corresponding to the @var{x} and @var{y} coordinates of the mesh.

See the file @file{sombrero.m} for an example of using @code{mesh} and
@code{meshdom}.
@end deftypefn

@defvr {Built-in Variable} gnuplot_binary
The name of the program invoked by the plot command.  The default value
is @code{"gnuplot"}.  @xref{Installation}.
@end defvr

@defvr {Built-in Variable} gnuplot_has_frames
If the value of this variable is nonzero, Octave assumes that your copy
of gnuplot has support for multiple frames that is included in recent
3.6beta releases.  It's initial value is determined by configure, but it
can be changed in your startup script or at the command line in case
configure got it wrong, or if you upgrade your gnuplot installation.
@end defvr

@deftypefn {Function File} {} figure (@var{n})
Set the current plot window to plot window @var{n}.  This function
currently requires X11 and a version of gnuplot that supports multiple
frames.
@end deftypefn

@defvr {Built-in Variable} gnuplot_has_multiplot
If the value of this variable is nonzero, Octave assumes that your copy
of gnuplot has the multiplot support that is included in recent
3.6beta releases.  It's initial value is determined by configure, but it
can be changed in your startup script or at the command line in case
configure got it wrong, or if you upgrade your gnuplot installation.
@end defvr

@node Plot Annotations, Multiple Plots on One Page, Three-Dimensional Plotting, Plotting
@section Plot Annotations

@deftypefn {Function File} {} grid
For two-dimensional plotting, force the display of a grid on the plot.
@end deftypefn

@deftypefn {Function File} {} title (@var{string})
Specify a title for the plot.  If you already have a plot displayed, use
the command @code{replot} to redisplay it with the new title.
@end deftypefn

@deftypefn {Function File} {} xlabel (@var{string})
@deftypefnx {Function File} {} ylabel (@var{string})
@deftypefnx {Function File} {} zlabel (@var{string})
Specify x, y, and z axis labels for the plot.  If you already have a plot
displayed, use the command @code{replot} to redisplay it with the new
labels.
@end deftypefn

@node Multiple Plots on One Page,  , Plot Annotations, Plotting
@section Multiple Plots on One Page

The following functions all require a version of @code{gnuplot} that
supports the multiplot feature.

@deftypefn {Function File} {} mplot (@var{x}, @var{y})
@deftypefnx {Function File} {} mplot (@var{x}, @var{y}, @var{fmt})
@deftypefnx {Function File} {} mplot (@var{x1}, @var{y1}, @var{x2}, @var{y2})
This is a modified version of the @code{plot} function that works with
the multiplot version of @code{gnuplot} to plot multiple plots per page. 
This plot version automatically advances to the next subplot position
after each set of arguments are processed.

See the description of the @var{plot} function for the various options.
@end deftypefn

@deftypefn {Function File} {} multiplot (@var{xn}, @var{yn})
Sets and resets multiplot mode.

If the arguments are non-zero, @code{multiplot} will set up multiplot
mode with @var{xn}, @var{yn} subplots along the @var{x} and @var{y}
axes.  If both arguments are zero, @code{multiplot} closes multiplot
mode.
@end deftypefn

@deftypefn {Function File} {} oneplot ()
If in multiplot mode, switches to single plot mode.
@end deftypefn

@deftypefn {Function File} {} plot_border (...)
Multiple arguments allowed to specify the sides on which the border
is shown.  Allowed arguments include:

@table @code
@item "blank"
No borders displayed.

@item "all"
All borders displayed

@item "north"
North Border

@item "south"
South Border

@item "east"
East Border

@item "west"
West Border
@end table

@noindent
The arguments may be abbreviated to single characters.  Without any
arguments, @code{plot_border} turns borders off.
@end deftypefn

@deftypefn {Function File} {} subplot (@var{rows}, @var{cols}, @var{index})
@deftypefnx {Function File} {} subplot (@var{rcn})
Sets @code{gnuplot} in multiplot mode and plots in location
given by index (there are @var{cols} by @var{rows} subwindows).

Input:

@table @var
@item rows
Number of rows in subplot grid.

@item columns
Number of columns in subplot grid.

@item index
Index of subplot where to make the next plot.
@end table

If only one argument is supplied, then it must be a three digit value
specifying the location in digits 1 (rows) and 2 (columns) and the plot
index in digit 3.

The plot index runs row-wise.  First all the columns in a row are filled
and then the next row is filled.

For example, a plot with 4 by 2 grid will have plot indices running as
follows:
@iftex
@tex
\vskip 10pt
\hfil\vbox{\offinterlineskip\hrule
\halign{\vrule#&&\qquad\hfil#\hfil\qquad\vrule\cr
height13pt&1&2&3&4\cr height12pt&&&&\cr\noalign{\hrule}
height13pt&5&6&7&8\cr height12pt&&&&\cr\noalign{\hrule}}}
\hfil
\vskip 10pt
@end tex
@end iftex
@ifinfo
@display
@group
+-----+-----+-----+-----+
|  1  |  2  |  3  |  4  |
+-----+-----+-----+-----+
|  5  |  6  |  7  |  8  |
+-----+-----+-----+-----+
@end group
@end display
@end ifinfo
@end deftypefn

@deftypefn {Function File} {} subwindow (@var{xn}, @var{yn})
Sets the subwindow position in multiplot mode for the next plot.  The
multiplot mode has to be previously initialized using the
@code{multiplot} function, otherwise this command just becomes an alias
to @code{multiplot}
@end deftypefn

@deftypefn {Function File} {} top_title (@var{string})
@deftypefnx {Function File} {} bottom_title (@var{string})
Makes a title with text @var{string} at the top (bottom) of the plot.
@end deftypefn
